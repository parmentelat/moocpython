    \hypertarget{les-property}{%
\section{\texorpdfstring{Les
\emph{property}}{Les property}}\label{les-property}}

    \textbf{Note}~: nous reviendrons largement sur cette notion de property
lorsque nous parlerons des \emph{property et descripteurs} en semaine 9.
Cependant, cette notion est suffisamment importante pour que nous vous
proposions un complément dès maintenant dessus.

    \hypertarget{compluxe9ment---niveau-intermuxe9diaire}{%
\subsection{Complément - niveau
intermédiaire}\label{compluxe9ment---niveau-intermuxe9diaire}}

    Comme on l'a vu dans le complément précédent, il est fréquent en Python
qu'une classe expose dans sa documentation un ou plusieurs attributs~;
c'est une pratique qui, en apparence seulement, paraît casser l'idée
d'une bonne encapsulation.\\

En réalité, grâce au mécanisme de \emph{property}, il n'en est rien.
Nous allons voir dans ce complément comment une classe peut en quelque
sorte intercepter les accès à ses attributs, et par là fournir une
encapsulation forte.\\

    Pour être concret, on va parler d'une classe \texttt{Temperature}. Au
lieu de proposer, comme ce serait l'usage dans d'autres langages, une
interface avec \texttt{get\_kelvin()} et \texttt{set\_kelvin()}, on va
se contenter d'exposer l'attribut \texttt{kelvin}, et malgré cela on va
pouvoir faire diverses vérifications et autres.

    \hypertarget{impluxe9mentation-nauxefve}{%
\subparagraph{Implémentation naïve\\\\}\label{impluxe9mentation-nauxefve}}

    Je vais commencer par une implémentation naïve, qui ne tire pas profit
des \emph{properties}~:

    \begin{Verbatim}[commandchars=\\\{\}]
{\color{incolor}In [{\color{incolor}1}]:} \PY{c+c1}{\PYZsh{} dans sa version la plus épurée, une classe}
        \PY{c+c1}{\PYZsh{} température pourrait ressembler à ça :}
        
        \PY{k}{class} \PY{n+nc}{Temperature1}\PY{p}{:}
            \PY{k}{def} \PY{n+nf}{\PYZus{}\PYZus{}init\PYZus{}\PYZus{}}\PY{p}{(}\PY{n+nb+bp}{self}\PY{p}{,} \PY{n}{kelvin}\PY{p}{)}\PY{p}{:}
                \PY{n+nb+bp}{self}\PY{o}{.}\PY{n}{kelvin} \PY{o}{=} \PY{n}{kelvin}
                
            \PY{k}{def} \PY{n+nf}{\PYZus{}\PYZus{}repr\PYZus{}\PYZus{}}\PY{p}{(}\PY{n+nb+bp}{self}\PY{p}{)}\PY{p}{:}
                \PY{k}{return} \PY{n}{f}\PY{l+s+s2}{\PYZdq{}}\PY{l+s+si}{\PYZob{}self.kelvin\PYZcb{}}\PY{l+s+s2}{K}\PY{l+s+s2}{\PYZdq{}}
\end{Verbatim}


    \begin{Verbatim}[commandchars=\\\{\}]
{\color{incolor}In [{\color{incolor}2}]:} \PY{c+c1}{\PYZsh{} créons une instance}
        \PY{n}{t1} \PY{o}{=} \PY{n}{Temperature1}\PY{p}{(}\PY{l+m+mi}{20}\PY{p}{)}
        \PY{n}{t1}
\end{Verbatim}


\begin{Verbatim}[commandchars=\\\{\}]
{\color{outcolor}Out[{\color{outcolor}2}]:} 20K
\end{Verbatim}
            
    \begin{Verbatim}[commandchars=\\\{\}]
{\color{incolor}In [{\color{incolor}3}]:} \PY{c+c1}{\PYZsh{} et pour accéder à la valeur numérique je peux faire}
        \PY{n}{t1}\PY{o}{.}\PY{n}{kelvin}
\end{Verbatim}


\begin{Verbatim}[commandchars=\\\{\}]
{\color{outcolor}Out[{\color{outcolor}3}]:} 20
\end{Verbatim}
            
    Avec cette implémentation il est très facile de créer une température
négative, qui n'a bien sûr pas de sens physique, ce n'est pas bon.

    \hypertarget{interface-gettersetter}{%
\subparagraph{\texorpdfstring{Interface
\emph{getter/setter}\\\\}{Interface getter/setter}}\label{interface-gettersetter}}

    Si vous avez été déjà exposés à des langages orientés objet comme C++,
Java ou autre, vous avez peut-être l'habitude d'accéder aux données
internes des instances par des \textbf{méthodes} de type
\emph{getter\textbf{ ou }setter}, de façon à contrôler les accès et,
dans une optique d'encapsulation, de préserver des invariants, comme ici
le fait que la température doit être positive.\\

C'est-à-dire que vous vous dites peut-être, ça ne devrait pas être fait
comme ça, on devrait plutôt proposer une interface pour accéder à
l'implémentation interne~; quelque chose comme~:

    \begin{Verbatim}[commandchars=\\\{\}]
{\color{incolor}In [{\color{incolor}4}]:} \PY{k}{class} \PY{n+nc}{Temperature2}\PY{p}{:}
            \PY{k}{def} \PY{n+nf}{\PYZus{}\PYZus{}init\PYZus{}\PYZus{}}\PY{p}{(}\PY{n+nb+bp}{self}\PY{p}{,} \PY{n}{kelvin}\PY{p}{)}\PY{p}{:}
                \PY{c+c1}{\PYZsh{} au lieu d\PYZsq{}écrire l\PYZsq{}attribut il est plus sûr}
                \PY{c+c1}{\PYZsh{} d\PYZsq{}utiliser le setter}
                \PY{n+nb+bp}{self}\PY{o}{.}\PY{n}{set\PYZus{}kelvin}\PY{p}{(}\PY{n}{kelvin}\PY{p}{)}
                
            \PY{k}{def} \PY{n+nf}{set\PYZus{}kelvin}\PY{p}{(}\PY{n+nb+bp}{self}\PY{p}{,} \PY{n}{kelvin}\PY{p}{)}\PY{p}{:}
                \PY{c+c1}{\PYZsh{} je m\PYZsq{}assure que \PYZus{}kelvin est toujours positif}
                \PY{c+c1}{\PYZsh{} et j\PYZsq{}utilise un nom d\PYZsq{}attribut avec un \PYZus{} pour signifier}
                \PY{c+c1}{\PYZsh{} que l\PYZsq{}attribut est privé et qu\PYZsq{}il ne faut pas y toucher directement}
                \PY{c+c1}{\PYZsh{} on pourrait aussi bien sûr lever une exception }
                \PY{c+c1}{\PYZsh{} mais ce n\PYZsq{}est pas mon sujet ici}
                \PY{n+nb+bp}{self}\PY{o}{.}\PY{n}{\PYZus{}kelvin} \PY{o}{=} \PY{n+nb}{max}\PY{p}{(}\PY{l+m+mi}{0}\PY{p}{,} \PY{n}{kelvin}\PY{p}{)}
                
            \PY{k}{def} \PY{n+nf}{get\PYZus{}kelvin}\PY{p}{(}\PY{n+nb+bp}{self}\PY{p}{)}\PY{p}{:}
                \PY{k}{return} \PY{n+nb+bp}{self}\PY{o}{.}\PY{n}{\PYZus{}kelvin}
                
            \PY{k}{def} \PY{n+nf}{\PYZus{}\PYZus{}repr\PYZus{}\PYZus{}}\PY{p}{(}\PY{n+nb+bp}{self}\PY{p}{)}\PY{p}{:}
                \PY{k}{return} \PY{n}{f}\PY{l+s+s2}{\PYZdq{}}\PY{l+s+si}{\PYZob{}self.\PYZus{}kelvin\PYZcb{}}\PY{l+s+s2}{K}\PY{l+s+s2}{\PYZdq{}}
\end{Verbatim}


    Bon c'est vrai que d'un coté, c'est mieux parce que je garantis un
invariant, la température est toujours positive~:

    \begin{Verbatim}[commandchars=\\\{\}]
{\color{incolor}In [{\color{incolor}5}]:} \PY{n}{t2} \PY{o}{=} \PY{n}{Temperature2}\PY{p}{(}\PY{o}{\PYZhy{}}\PY{l+m+mi}{30}\PY{p}{)}
        \PY{n}{t2}
\end{Verbatim}


\begin{Verbatim}[commandchars=\\\{\}]
{\color{outcolor}Out[{\color{outcolor}5}]:} 0K
\end{Verbatim}
            
    Mais par contre, d'un autre coté, c'est très lourd, parce que chaque
fois que je veux utiliser mon objet, je dois faire pour y accéder~:

    \begin{Verbatim}[commandchars=\\\{\}]
{\color{incolor}In [{\color{incolor}6}]:} \PY{n}{t2}\PY{o}{.}\PY{n}{get\PYZus{}kelvin}\PY{p}{(}\PY{p}{)}
\end{Verbatim}


\begin{Verbatim}[commandchars=\\\{\}]
{\color{outcolor}Out[{\color{outcolor}6}]:} 0
\end{Verbatim}
            
    et aussi, si j'avais déjà du code qui utilisait \texttt{t.kelvin} il va
falloir le modifier entièrement.

    \hypertarget{impluxe9mentation-pythonique}{%
\subparagraph{Implémentation
pythonique\\\\}\label{impluxe9mentation-pythonique}}

    La façon de s'en sortir ici consiste à définir une property. Comme on va
le voir ce mécanisme permet d'écrire du code qui fait référence à
l'attribut \texttt{kelvin} de l'instance, mais qui passe tout de même
par une couche de logique.\\

Ça ressemblerait à ceci~:

    \begin{Verbatim}[commandchars=\\\{\}]
{\color{incolor}In [{\color{incolor}7}]:} \PY{k}{class} \PY{n+nc}{Temperature3}\PY{p}{:}
            \PY{k}{def} \PY{n+nf}{\PYZus{}\PYZus{}init\PYZus{}\PYZus{}}\PY{p}{(}\PY{n+nb+bp}{self}\PY{p}{,} \PY{n}{kelvin}\PY{p}{)}\PY{p}{:}
                \PY{n+nb+bp}{self}\PY{o}{.}\PY{n}{kelvin} \PY{o}{=} \PY{n}{kelvin}
        
            \PY{c+c1}{\PYZsh{} je définis bel et bien mes accesseurs de type getter et setter}
            \PY{c+c1}{\PYZsh{} mais \PYZus{}get\PYZus{}kelvin commence avec un \PYZus{} }
            \PY{c+c1}{\PYZsh{} car il n\PYZsq{}est pas censé être appelé par l\PYZsq{}extérieur}
            \PY{k}{def} \PY{n+nf}{\PYZus{}get\PYZus{}kelvin}\PY{p}{(}\PY{n+nb+bp}{self}\PY{p}{)}\PY{p}{:}
                \PY{k}{return} \PY{n+nb+bp}{self}\PY{o}{.}\PY{n}{\PYZus{}kelvin}
        
            \PY{c+c1}{\PYZsh{} idem}
            \PY{k}{def} \PY{n+nf}{\PYZus{}set\PYZus{}kelvin}\PY{p}{(}\PY{n+nb+bp}{self}\PY{p}{,} \PY{n}{kelvin}\PY{p}{)}\PY{p}{:}
                \PY{n+nb+bp}{self}\PY{o}{.}\PY{n}{\PYZus{}kelvin} \PY{o}{=} \PY{n+nb}{max}\PY{p}{(}\PY{l+m+mi}{0}\PY{p}{,} \PY{n}{kelvin}\PY{p}{)}
                
            \PY{c+c1}{\PYZsh{} une fois que j\PYZsq{}ai ces deux éléments je peux créer une property}
            \PY{n}{kelvin} \PY{o}{=} \PY{n+nb}{property}\PY{p}{(}\PY{n}{\PYZus{}get\PYZus{}kelvin}\PY{p}{,} \PY{n}{\PYZus{}set\PYZus{}kelvin}\PY{p}{)}
            
            \PY{c+c1}{\PYZsh{} et toujours la façon d\PYZsq{}imprimer}
            \PY{k}{def} \PY{n+nf}{\PYZus{}\PYZus{}repr\PYZus{}\PYZus{}}\PY{p}{(}\PY{n+nb+bp}{self}\PY{p}{)}\PY{p}{:}
                \PY{k}{return} \PY{n}{f}\PY{l+s+s2}{\PYZdq{}}\PY{l+s+si}{\PYZob{}self.\PYZus{}kelvin\PYZcb{}}\PY{l+s+s2}{K}\PY{l+s+s2}{\PYZdq{}}    
\end{Verbatim}


    \begin{Verbatim}[commandchars=\\\{\}]
{\color{incolor}In [{\color{incolor}8}]:} \PY{n}{t3} \PY{o}{=} \PY{n}{Temperature3}\PY{p}{(}\PY{l+m+mi}{200}\PY{p}{)}
        \PY{n}{t3}
\end{Verbatim}


\begin{Verbatim}[commandchars=\\\{\}]
{\color{outcolor}Out[{\color{outcolor}8}]:} 200K
\end{Verbatim}
            
    \begin{Verbatim}[commandchars=\\\{\}]
{\color{incolor}In [{\color{incolor}9}]:} \PY{c+c1}{\PYZsh{} par contre ici on va le mettre à zéro}
        \PY{c+c1}{\PYZsh{} à nouveau, une exception serait préférable sans doute}
        \PY{n}{t3}\PY{o}{.}\PY{n}{kelvin} \PY{o}{=} \PY{o}{\PYZhy{}}\PY{l+m+mi}{30}
        \PY{n}{t3}
\end{Verbatim}


\begin{Verbatim}[commandchars=\\\{\}]
{\color{outcolor}Out[{\color{outcolor}9}]:} 0K
\end{Verbatim}
            
    Comme vous pouvez le voir, cette technique a plusieurs avantages~:

\begin{itemize}
\tightlist
\item
  on a ce qu'on cherchait, c'est-à-dire une façon d'ajouter une couche
  de logique lors des accès en lecture et en écriture à l'intérieur de
  l'objet,
\item
  mais \textbf{sans toutefois} demander à l'utilisateur de passer son
  temps à envoyer des méthodes \texttt{get\_} et \texttt{set()} sur
  l'objet, ce qui a tendance à alourdir considérablement le code.
\end{itemize}

C'est pour cette raison que vous ne rencontrerez presque jamais en
Python une bibliothèque qui offre une interface à base de méthodes
\texttt{get\_something} et \texttt{set\_something}, mais au contraire
les API vous exposeront directement des attributs que vous devez
utiliser directement.

    \hypertarget{compluxe9ment---niveau-avancuxe9}{%
\subsection{Complément - niveau
avancé}\label{compluxe9ment---niveau-avancuxe9}}

    À titre d'exemple d'utilisation, voici une dernière implémentation de
\texttt{Temperature} qui donne l'illusion d'avoir 3 attributs
(\texttt{kelvin}, \texttt{celsius} et \texttt{fahrenheit}), alors qu'en
réalité le seul attribut de donnée est \texttt{\_kelvin}.

    \begin{Verbatim}[commandchars=\\\{\}]
{\color{incolor}In [{\color{incolor}10}]:} \PY{k}{class} \PY{n+nc}{Temperature}\PY{p}{:}
         
             \PY{c+c1}{\PYZsh{}\PYZsh{} les constantes de conversion}
             \PY{c+c1}{\PYZsh{} kelvin / celsius}
             \PY{n}{K} \PY{o}{=} \PY{l+m+mf}{273.16}
             \PY{c+c1}{\PYZsh{} fahrenheit / celsius}
             \PY{n}{RF} \PY{o}{=} \PY{l+m+mi}{5} \PY{o}{/} \PY{l+m+mi}{9}
             \PY{n}{KF} \PY{o}{=} \PY{p}{(}\PY{n}{K} \PY{o}{/} \PY{n}{RF}\PY{p}{)} \PY{o}{\PYZhy{}} \PY{l+m+mi}{32}
         
             \PY{k}{def} \PY{n+nf}{\PYZus{}\PYZus{}init\PYZus{}\PYZus{}}\PY{p}{(}\PY{n+nb+bp}{self}\PY{p}{,} \PY{n}{kelvin}\PY{o}{=}\PY{k+kc}{None}\PY{p}{,} \PY{n}{celsius}\PY{o}{=}\PY{k+kc}{None}\PY{p}{,} \PY{n}{fahrenheit}\PY{o}{=}\PY{k+kc}{None}\PY{p}{)}\PY{p}{:}
                 \PY{l+s+sd}{\PYZdq{}\PYZdq{}\PYZdq{}}
         \PY{l+s+sd}{        Création à partir de n\PYZsq{}importe quelle unité}
         \PY{l+s+sd}{        Il faut préciser exactement une des trois unités}
         \PY{l+s+sd}{        \PYZdq{}\PYZdq{}\PYZdq{}}
                 \PY{c+c1}{\PYZsh{} on passe par les properties pour initialiser}
                 \PY{k}{if} \PY{n}{kelvin} \PY{o+ow}{is} \PY{o+ow}{not} \PY{k+kc}{None}\PY{p}{:}
                     \PY{n+nb+bp}{self}\PY{o}{.}\PY{n}{kelvin} \PY{o}{=} \PY{n}{kelvin}
                 \PY{k}{elif} \PY{n}{celsius} \PY{o+ow}{is} \PY{o+ow}{not} \PY{k+kc}{None}\PY{p}{:}
                     \PY{n+nb+bp}{self}\PY{o}{.}\PY{n}{celsius} \PY{o}{=} \PY{n}{celsius}
                 \PY{k}{elif} \PY{n}{fahrenheit} \PY{o+ow}{is} \PY{o+ow}{not} \PY{k+kc}{None}\PY{p}{:}
                     \PY{n+nb+bp}{self}\PY{o}{.}\PY{n}{fahrenheit} \PY{o}{=} \PY{n}{fahrenheit}
                 \PY{k}{else}\PY{p}{:}
                     \PY{n+nb+bp}{self}\PY{o}{.}\PY{n}{kelvin} \PY{o}{=} \PY{l+m+mi}{0}
                     \PY{k}{raise} \PY{n+ne}{ValueError}\PY{p}{(}\PY{l+s+s2}{\PYZdq{}}\PY{l+s+s2}{need to specify at least one unit}\PY{l+s+s2}{\PYZdq{}}\PY{p}{)}
         
             \PY{c+c1}{\PYZsh{} pour le confort}
             \PY{k}{def} \PY{n+nf}{\PYZus{}\PYZus{}repr\PYZus{}\PYZus{}}\PY{p}{(}\PY{n+nb+bp}{self}\PY{p}{)}\PY{p}{:}
                 \PY{k}{return} \PY{n}{f}\PY{l+s+s2}{\PYZdq{}}\PY{l+s+s2}{\PYZlt{}}\PY{l+s+si}{\PYZob{}self.kelvin:g\PYZcb{}}\PY{l+s+s2}{K == }\PY{l+s+si}{\PYZob{}self.celsius:g\PYZcb{}}\PY{l+s+s2}{\textdegree{C} }\PY{l+s+s2}{\PYZdq{}} \PYZbs{}
                        \PY{n}{f}\PY{l+s+s2}{\PYZdq{}}\PY{l+s+s2}{== }\PY{l+s+si}{\PYZob{}self.fahrenheit:g\PYZcb{}}\PY{l+s+s2}{\textdegree{F}\PYZgt{}}\PY{l+s+s2}{\PYZdq{}}
         
             \PY{k}{def} \PY{n+nf}{\PYZus{}\PYZus{}str\PYZus{}\PYZus{}}\PY{p}{(}\PY{n+nb+bp}{self}\PY{p}{)}\PY{p}{:}
                 \PY{k}{return} \PY{n}{f}\PY{l+s+s2}{\PYZdq{}}\PY{l+s+si}{\PYZob{}self.kelvin:g\PYZcb{}}\PY{l+s+s2}{K}\PY{l+s+s2}{\PYZdq{}}
         
         
             \PY{c+c1}{\PYZsh{} l\PYZsq{}attribut \PYZsq{}kelvin\PYZsq{} n\PYZsq{}a pas de conversion à faire,}
             \PY{c+c1}{\PYZsh{} mais il vérifie que la valeur est positive}
             \PY{k}{def} \PY{n+nf}{\PYZus{}get\PYZus{}kelvin}\PY{p}{(}\PY{n+nb+bp}{self}\PY{p}{)}\PY{p}{:}
                 \PY{k}{return} \PY{n+nb+bp}{self}\PY{o}{.}\PY{n}{\PYZus{}kelvin}
         
             \PY{k}{def} \PY{n+nf}{\PYZus{}set\PYZus{}kelvin}\PY{p}{(}\PY{n+nb+bp}{self}\PY{p}{,} \PY{n}{kelvin}\PY{p}{)}\PY{p}{:}
                 \PY{k}{if} \PY{n}{kelvin} \PY{o}{\PYZlt{}} \PY{l+m+mi}{0}\PY{p}{:}
                     \PY{k}{raise} \PY{n+ne}{ValueError}\PY{p}{(}\PY{n}{f}\PY{l+s+s2}{\PYZdq{}}\PY{l+s+s2}{Kelvin }\PY{l+s+si}{\PYZob{}kelvin\PYZcb{}}\PY{l+s+s2}{ must be positive}\PY{l+s+s2}{\PYZdq{}}\PY{p}{)}
                 \PY{n+nb+bp}{self}\PY{o}{.}\PY{n}{\PYZus{}kelvin} \PY{o}{=} \PY{n}{kelvin}
         
             \PY{c+c1}{\PYZsh{} la property qui définit l\PYZsq{}attribut `kelvin`}
             \PY{n}{kelvin} \PY{o}{=} \PY{n+nb}{property}\PY{p}{(}\PY{n}{\PYZus{}get\PYZus{}kelvin}\PY{p}{,} \PY{n}{\PYZus{}set\PYZus{}kelvin}\PY{p}{)}
         
         
             \PY{c+c1}{\PYZsh{} les deux autres properties font la conversion, puis }
             \PY{c+c1}{\PYZsh{} sous\PYZhy{}traitent à la property kelvin pour le contrôle de borne}
             \PY{k}{def} \PY{n+nf}{\PYZus{}set\PYZus{}celsius}\PY{p}{(}\PY{n+nb+bp}{self}\PY{p}{,} \PY{n}{celsius}\PY{p}{)}\PY{p}{:}
                 \PY{c+c1}{\PYZsh{} using .kelvin instead of .\PYZus{}kelvin to enforce}
                 \PY{n+nb+bp}{self}\PY{o}{.}\PY{n}{kelvin} \PY{o}{=} \PY{n}{celsius} \PY{o}{+} \PY{n+nb+bp}{self}\PY{o}{.}\PY{n}{K}
         
             \PY{k}{def} \PY{n+nf}{\PYZus{}get\PYZus{}celsius}\PY{p}{(}\PY{n+nb+bp}{self}\PY{p}{)}\PY{p}{:}
                 \PY{k}{return} \PY{n+nb+bp}{self}\PY{o}{.}\PY{n}{\PYZus{}kelvin} \PY{o}{\PYZhy{}} \PY{n+nb+bp}{self}\PY{o}{.}\PY{n}{K}
             
             \PY{n}{celsius} \PY{o}{=} \PY{n+nb}{property}\PY{p}{(}\PY{n}{\PYZus{}get\PYZus{}celsius}\PY{p}{,} \PY{n}{\PYZus{}set\PYZus{}celsius}\PY{p}{)}
         
             \PY{k}{def} \PY{n+nf}{\PYZus{}set\PYZus{}fahrenheit}\PY{p}{(}\PY{n+nb+bp}{self}\PY{p}{,} \PY{n}{fahrenheit}\PY{p}{)}\PY{p}{:}
                 \PY{c+c1}{\PYZsh{} using .kelvin instead of .\PYZus{}kelvin to enforce}
                 \PY{n+nb+bp}{self}\PY{o}{.}\PY{n}{kelvin} \PY{o}{=} \PY{p}{(}\PY{n}{fahrenheit} \PY{o}{+} \PY{n+nb+bp}{self}\PY{o}{.}\PY{n}{KF}\PY{p}{)} \PY{o}{*} \PY{n+nb+bp}{self}\PY{o}{.}\PY{n}{RF}
         
             \PY{k}{def} \PY{n+nf}{\PYZus{}get\PYZus{}fahrenheit}\PY{p}{(}\PY{n+nb+bp}{self}\PY{p}{)}\PY{p}{:}
                 \PY{k}{return} \PY{n+nb+bp}{self}\PY{o}{.}\PY{n}{\PYZus{}kelvin} \PY{o}{/} \PY{n+nb+bp}{self}\PY{o}{.}\PY{n}{RF} \PY{o}{\PYZhy{}} \PY{n+nb+bp}{self}\PY{o}{.}\PY{n}{KF}
             
             \PY{n}{fahrenheit} \PY{o}{=} \PY{n+nb}{property}\PY{p}{(}\PY{n}{\PYZus{}get\PYZus{}fahrenheit}\PY{p}{,} \PY{n}{\PYZus{}set\PYZus{}fahrenheit}\PY{p}{)}
\end{Verbatim}


    Et voici ce qu'on peut en faire~:

    \begin{Verbatim}[commandchars=\\\{\}]
{\color{incolor}In [{\color{incolor}11}]:} \PY{n}{t} \PY{o}{=} \PY{n}{Temperature}\PY{p}{(}\PY{n}{celsius}\PY{o}{=}\PY{l+m+mi}{0}\PY{p}{)}
         \PY{n}{t}
\end{Verbatim}


\begin{Verbatim}[commandchars=\\\{\}]
{\color{outcolor}Out[{\color{outcolor}11}]:} <273.16K == 0\textdegree{C} == 32\textdegree[F]>
\end{Verbatim}
            
    \begin{Verbatim}[commandchars=\\\{\}]
{\color{incolor}In [{\color{incolor}12}]:} \PY{n}{t}\PY{o}{.}\PY{n}{fahrenheit}
\end{Verbatim}


\begin{Verbatim}[commandchars=\\\{\}]
{\color{outcolor}Out[{\color{outcolor}12}]:} 32.0
\end{Verbatim}
            
    \begin{Verbatim}[commandchars=\\\{\}]
{\color{incolor}In [{\color{incolor}13}]:} \PY{n}{t}\PY{o}{.}\PY{n}{celsius} \PY{o}{+}\PY{o}{=} \PY{l+m+mi}{100}
         \PY{n+nb}{print}\PY{p}{(}\PY{n}{t}\PY{p}{)}
\end{Verbatim}


    \begin{Verbatim}[commandchars=\\\{\}]
373.16K

    \end{Verbatim}

    \begin{Verbatim}[commandchars=\\\{\}]
{\color{incolor}In [{\color{incolor}14}]:} \PY{k}{try}\PY{p}{:}
             \PY{n}{t} \PY{o}{=} \PY{n}{Temperature}\PY{p}{(}\PY{n}{fahrenheit} \PY{o}{=} \PY{o}{\PYZhy{}}\PY{l+m+mi}{1000}\PY{p}{)}
         \PY{k}{except} \PY{n+ne}{Exception} \PY{k}{as} \PY{n}{e}\PY{p}{:}
             \PY{n+nb}{print}\PY{p}{(}\PY{n}{f}\PY{l+s+s2}{\PYZdq{}}\PY{l+s+s2}{OOPS, }\PY{l+s+s2}{\PYZob{}}\PY{l+s+s2}{type(e)\PYZcb{}, }\PY{l+s+si}{\PYZob{}e\PYZcb{}}\PY{l+s+s2}{\PYZdq{}}\PY{p}{)}
\end{Verbatim}


    \begin{Verbatim}[commandchars=\\\{\}]
OOPS, <class 'ValueError'>, Kelvin -300.1733333333333 must be positive

    \end{Verbatim}

    \hypertarget{pour-en-savoir-plus}{%
\subparagraph{Pour en savoir plus\\\\}\label{pour-en-savoir-plus}}

    Voir aussi
\href{https://docs.python.org/3.6/library/functions.html\#property}{la
documentation officielle}.\\

    Vous pouvez notamment aussi, en option, ajouter un \emph{deleter} pour
intercepter les instructions du type~:

    \begin{Verbatim}[commandchars=\\\{\}]
{\color{incolor}In [{\color{incolor}15}]:} \PY{c+c1}{\PYZsh{} comme on n\PYZsq{}a pas défini de deleter, on ne peut pas faire ceci}
         \PY{k}{try}\PY{p}{:}
             \PY{k}{del} \PY{n}{t}\PY{o}{.}\PY{n}{kelvin}
         \PY{k}{except} \PY{n+ne}{Exception} \PY{k}{as} \PY{n}{e}\PY{p}{:}
             \PY{n+nb}{print}\PY{p}{(}\PY{n}{f}\PY{l+s+s2}{\PYZdq{}}\PY{l+s+s2}{OOPS }\PY{l+s+s2}{\PYZob{}}\PY{l+s+s2}{type(e)\PYZcb{} }\PY{l+s+si}{\PYZob{}e\PYZcb{}}\PY{l+s+s2}{\PYZdq{}}\PY{p}{)}
\end{Verbatim}


    \begin{Verbatim}[commandchars=\\\{\}]
OOPS <class 'AttributeError'> can't delete attribute

    \end{Verbatim}