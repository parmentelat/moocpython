    \hypertarget{espaces-de-nommage}{%
\section{Espaces de nommage}\label{espaces-de-nommage}}

    \hypertarget{compluxe9ment---niveau-basique}{%
\subsection{Complément - niveau
basique}\label{compluxe9ment---niveau-basique}}

    Nous venons de voir les règles pour l'affectation (ou l'assignation) et
le référencement des variables et des attributs~; en particulier, on
doit faire une distinction entre les attributs et les variables.

    \begin{itemize}
\tightlist
\item
  Les attributs sont résolus de manière \textbf{dynamique}, c'est-à-dire
  au moment de l'exécution (\emph{run-time})~;
\item
  alors que la liaison des variables est par contre \textbf{statique}
  (\emph{compile-time}) et \textbf{lexicale}, en ce sens qu'elle se base
  uniquement sur les imbrications de code.
\end{itemize}

    Vous voyez donc que la différence entre attributs et variables est
fondamentale. Dans ce complément, nous allons reprendre et résumer les
différentes règles qui régissent l'affectation et le référencement des
attributs et des variables.

    \hypertarget{attributs}{%
\subparagraph{Attributs\\\\}\label{attributs}}

    Un attribut est un symbole \texttt{x} utilisé dans la notation
\texttt{obj.x} où \texttt{obj} est l'objet qui définit l'espace de
nommage sur lequel \texttt{x} existe.\\

L'\textbf{affectation} (explicite ou implicite) d'un attribut \texttt{x}
sur un objet \texttt{obj} va créer (ou altérer) un symbole \texttt{x}
\textbf{directement} dans l'espace de nommage de \texttt{obj}, symbole
qui va référencer l'objet affecté, typiquement l'objet à droite du signe
\texttt{=}

    \begin{Verbatim}[commandchars=\\\{\}]
{\color{incolor}In [{\color{incolor}1}]:} \PY{k}{class} \PY{n+nc}{MaClasse}\PY{p}{:}
            \PY{k}{pass}
        
        \PY{c+c1}{\PYZsh{} affectation explicite}
        \PY{n}{MaClasse}\PY{o}{.}\PY{n}{x} \PY{o}{=} \PY{l+m+mi}{10} 
        
        \PY{c+c1}{\PYZsh{} le symbole x est défini dans l\PYZsq{}espace de nommage de MaClasse}
        \PY{l+s+s1}{\PYZsq{}}\PY{l+s+s1}{x}\PY{l+s+s1}{\PYZsq{}} \PY{o+ow}{in} \PY{n}{MaClasse}\PY{o}{.}\PY{n+nv+vm}{\PYZus{}\PYZus{}dict\PYZus{}\PYZus{}}
\end{Verbatim}


\begin{Verbatim}[commandchars=\\\{\}]
{\color{outcolor}Out[{\color{outcolor}1}]:} True
\end{Verbatim}
            
    Le \textbf{référencement} (la lecture) d'un attribut va chercher cet
attribut \textbf{le long de l'arbre d'héritage} en commençant par
l'instance, puis la classe qui a créé l'instance, puis les super-classes
et suivant la MRO (voir le complément sur l'héritage multiple).

    \hypertarget{variables}{%
\subparagraph{Variables\\\\}\label{variables}}

    Une variable est un symbole qui n'est pas précédé de la notation
\texttt{obj.} et l'affectation d'une variable rend cette variable locale
au bloc de code dans lequel elle est définie, un bloc de code pouvant
être~:

\begin{itemize}
\tightlist
\item
  une fonction, dans ce cas la variable est locale à la fonction~;
\item
  une classe, dans ce cas la variable est locale à la classe~;
\item
  un module, dans ce cas la variable est locale au module, on dit
  également que la variable est globale.
\end{itemize}

Une variable référencée est toujours cherchée suivant la règle LEGB~:

\begin{itemize}
\tightlist
\item
  localement au bloc de code dans lequel elle est référencée~;
\item
  puis dans les blocs de code des \textbf{fonctions ou méthodes}
  englobantes, s'il y en a, de la plus proche à la plus eloignée~;
\item
  puis dans le bloc de code du module.
\end{itemize}

Si la variable n'est toujours pas trouvée, elle est cherchée dans le
module \texttt{builtins} et si elle n'est toujours pas trouvée, une
exception est levée.\\

Par exemple~:

    \begin{Verbatim}[commandchars=\\\{\}]
{\color{incolor}In [{\color{incolor}2}]:} \PY{n}{var} \PY{o}{=} \PY{l+s+s1}{\PYZsq{}}\PY{l+s+s1}{dans le module}\PY{l+s+s1}{\PYZsq{}}
        
        \PY{k}{class} \PY{n+nc}{A}\PY{p}{:}
            \PY{n}{var} \PY{o}{=} \PY{l+s+s1}{\PYZsq{}}\PY{l+s+s1}{dans la classe A}\PY{l+s+s1}{\PYZsq{}}
            \PY{k}{def} \PY{n+nf}{f}\PY{p}{(}\PY{n+nb+bp}{self}\PY{p}{)}\PY{p}{:}
                \PY{n}{var} \PY{o}{=} \PY{l+s+s1}{\PYZsq{}}\PY{l+s+s1}{dans la fonction f}\PY{l+s+s1}{\PYZsq{}}
                \PY{k}{class} \PY{n+nc}{B}\PY{p}{:}
                    \PY{n+nb}{print}\PY{p}{(}\PY{n}{var}\PY{p}{)}
                \PY{n}{B}\PY{p}{(}\PY{p}{)}
        \PY{n}{A}\PY{p}{(}\PY{p}{)}\PY{o}{.}\PY{n}{f}\PY{p}{(}\PY{p}{)}
\end{Verbatim}


    \begin{Verbatim}[commandchars=\\\{\}]
dans la fonction f

    \end{Verbatim}

    \hypertarget{en-ruxe9sumuxe9}{%
\subparagraph{En résumé}\label{en-ruxe9sumuxe9}}

    Dans la vidéo et dans ce complément basique, on a couvert tous les cas
standards, et même si python est un langage plutôt mieux fait, avec
moins de cas particuliers que d'autres langages, il a également ses cas
étranges entre raisons historiques et bugs qui ne seront jamais corrigés
(parce que ça casserait plus de choses que ça n'en réparerait). Pour
éviter de tomber dans ces cas spéciaux, c'est simple, vous n'avez qu'à
suivre ces règles~:

\begin{itemize}
\tightlist
\item
  ne jamais affecter dans un bloc de code local une variable de même nom
  qu'une variable globale~;
\item
  éviter d'utiliser les directives \texttt{global} et \texttt{nonlocal},
  et les réserver pour du code avancé comme les décorateurs et les
  métaclasses~;
\item
  et lorsque vous devez vraiment les utiliser, toujours mettre les
  directives \texttt{global} et \texttt{nonlocal} comme premières
  instructions du bloc de code où elle s'appliquent.
\end{itemize}

Si vous ne suivez pas ces règles, vous risquez de tomber dans un cas
particulier que nous détaillons ci-dessous dans la partie avancée.

    \hypertarget{compluxe9ment---niveau-avancuxe9}{%
\subsection{Complément - niveau
avancé}\label{compluxe9ment---niveau-avancuxe9}}

    \hypertarget{la-documentation-officielle-est-fausse}{%
\subparagraph{La documentation officielle est
fausse\\\\}\label{la-documentation-officielle-est-fausse}}

    Oui, vous avez bien lu, la documentation officielle est fausse sur un
point subtil. Regardons le
\href{https://docs.python.org/3/reference/executionmodel.html}{modèle
d'exécution}, on trouve la phrase suivante ``If a name binding operation
occurs anywhere within a code block, all uses of the name within the
block are treated as references to the current block.'' qui est fausse,
il faut lire ``If a name binding operation occurs anywhere within a code
block \textbf{of a function}, all uses of the name within the block are
treated as references to the current block.''\\

En effet, les classes se comportent différemment des fonctions~:

    \begin{Verbatim}[commandchars=\\\{\}]
{\color{incolor}In [{\color{incolor}3}]:} \PY{n}{x} \PY{o}{=} \PY{l+s+s2}{\PYZdq{}}\PY{l+s+s2}{x du module}\PY{l+s+s2}{\PYZdq{}}
        \PY{k}{class} \PY{n+nc}{A}\PY{p}{(}\PY{p}{)}\PY{p}{:}
            \PY{n+nb}{print}\PY{p}{(}\PY{l+s+s2}{\PYZdq{}}\PY{l+s+s2}{dans classe A: }\PY{l+s+s2}{\PYZdq{}} \PY{o}{+} \PY{n}{x}\PY{p}{)}
            \PY{n}{x} \PY{o}{=} \PY{l+s+s2}{\PYZdq{}}\PY{l+s+s2}{x dans A}\PY{l+s+s2}{\PYZdq{}}
            \PY{n+nb}{print}\PY{p}{(}\PY{l+s+s2}{\PYZdq{}}\PY{l+s+s2}{dans classe A: }\PY{l+s+s2}{\PYZdq{}} \PY{o}{+} \PY{n}{x}\PY{p}{)}
            \PY{k}{del} \PY{n}{x}
            \PY{n+nb}{print}\PY{p}{(}\PY{l+s+s2}{\PYZdq{}}\PY{l+s+s2}{dans classe A: }\PY{l+s+s2}{\PYZdq{}} \PY{o}{+} \PY{n}{x}\PY{p}{)}
\end{Verbatim}


    \begin{Verbatim}[commandchars=\\\{\}]
dans classe A: x du module
dans classe A: x dans A
dans classe A: x du module

    \end{Verbatim}

    Alors pourquoi si c'est une mauvaise idée de mélanger variables globales
et locales de même nom dans une fonction, c'est possible dans une
classe~?\\

Cela vient de la manière dont sont implémentés les espaces de nommage.
Normalement, un objet a pour espace de nommage un dictionnaire qui
s'appelle \texttt{\_\_dict\_\_}. D'un côté un dictionnaire est un objet
python qui offre beaucoup de flexibilité, mais d'un autre côté, il
induit un petit surcoût pour chaque recherche d'éléments. Comme les
fonctions sont des objets qui par définition peuvent être appelés très
souvent, il a été décidé de mettre toutes les variables locales à la
fonction dans un objet écrit en C qui n'est pas dynamique (on ne peut
pas ajouter des éléments à l'exécution), mais qui est un peu plus rapide
qu'un dictionnaire lors de l'accès aux variables. Mais pour faire cela,
il faut déterminer la portée de la variable dans la phase de
précompilation. Donc si le précompilateur trouve une affectation
(explicite ou implicite) dans une fonction, il considère la variable
comme locale pour tout le bloc de code. Donc si on référence une
variable définie comme étant locale avant une affectation dans la
fonction, on ne va pas la chercher globalement, on a une erreur
\texttt{UnboundLocalError}.\\

Cette optimisation n'a pas été faite pour les classes, parce que dans
l'évaluation du compromis souplesse contre efficacité pour les classes,
c'est la souplesse, donc le dictionnaire qui a gagné.