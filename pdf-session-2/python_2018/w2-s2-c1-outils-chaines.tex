    \hypertarget{les-outils-de-base-sur-les-chauxeenes-de-caractuxe8res-str}{%
\section{\texorpdfstring{Les outils de base sur les chaînes de
caractères
(\texttt{str})}{Les outils de base sur les chaînes de caractères (str)}}\label{les-outils-de-base-sur-les-chauxeenes-de-caractuxe8res-str}}

    \hypertarget{compluxe9ment---niveau-intermuxe9diaire}{%
\subsection{Complément - niveau
intermédiaire}\label{compluxe9ment---niveau-intermuxe9diaire}}

    \hypertarget{lire-la-documentation}{%
\subsubsection{Lire la documentation}\label{lire-la-documentation}}

    Même après des années de pratique, il est difficile de se souvenir de
toutes les méthodes travaillant sur les chaînes de caractères. Aussi il
est toujours utile de recourir à la documentation embarquée

    \begin{Verbatim}[commandchars=\\\{\}]
{\color{incolor}In [{\color{incolor}1}]:} \PY{n}{help}\PY{p}{(}\PY{n+nb}{str}\PY{p}{)}
\end{Verbatim}



    Nous allons tenter ici de citer les méthodes les plus utilisées. Nous
n'avons le temps que de les utiliser de manière très simple, mais bien
souvent il est possible de passer en argument des options permettant de
ne travailler que sur une sous-chaîne, ou sur la première ou dernière
occurrence d'une sous-chaîne. Nous vous renvoyons à la documentation
pour obtenir toutes les précisions utiles.

    \hypertarget{duxe9coupage---assemblage-split-et-join}{%
\subsubsection{\texorpdfstring{Découpage - assemblage~: \texttt{split}
et
\texttt{join}}{Découpage - assemblage~: split et join}}\label{duxe9coupage---assemblage-split-et-join}}

    Les méthodes \texttt{split} et \texttt{join} permettent de découper une
chaîne selon un séparateur pour obtenir une liste, et à l'inverse de
reconstruire une chaîne à partir d'une liste.

    \texttt{split} permet donc de découper~:

    \begin{Verbatim}[commandchars=\\\{\}]
{\color{incolor}In [{\color{incolor}2}]:} \PY{l+s+s1}{\PYZsq{}}\PY{l+s+s1}{abc=:=def=:=ghi=:=jkl}\PY{l+s+s1}{\PYZsq{}}\PY{o}{.}\PY{n}{split}\PY{p}{(}\PY{l+s+s1}{\PYZsq{}}\PY{l+s+s1}{=:=}\PY{l+s+s1}{\PYZsq{}}\PY{p}{)}
\end{Verbatim}


\begin{Verbatim}[commandchars=\\\{\}]
{\color{outcolor}Out[{\color{outcolor}2}]:} ['abc', 'def', 'ghi', 'jkl']
\end{Verbatim}
            
    Et à l'inverse~:

    \begin{Verbatim}[commandchars=\\\{\}]
{\color{incolor}In [{\color{incolor}3}]:} \PY{l+s+s2}{\PYZdq{}}\PY{l+s+s2}{=:=}\PY{l+s+s2}{\PYZdq{}}\PY{o}{.}\PY{n}{join}\PY{p}{(}\PY{p}{[}\PY{l+s+s1}{\PYZsq{}}\PY{l+s+s1}{abc}\PY{l+s+s1}{\PYZsq{}}\PY{p}{,} \PY{l+s+s1}{\PYZsq{}}\PY{l+s+s1}{def}\PY{l+s+s1}{\PYZsq{}}\PY{p}{,} \PY{l+s+s1}{\PYZsq{}}\PY{l+s+s1}{ghi}\PY{l+s+s1}{\PYZsq{}}\PY{p}{,} \PY{l+s+s1}{\PYZsq{}}\PY{l+s+s1}{jkl}\PY{l+s+s1}{\PYZsq{}}\PY{p}{]}\PY{p}{)}
\end{Verbatim}


\begin{Verbatim}[commandchars=\\\{\}]
{\color{outcolor}Out[{\color{outcolor}3}]:} 'abc=:=def=:=ghi=:=jkl'
\end{Verbatim}
            
    Attention toutefois si le séparateur est un terminateur, la liste
résultat contient alors une dernière chaîne vide. En pratique, on
utilisera la méthode \texttt{strip}, que nous allons voir ci-dessous,
avant la méthode \texttt{split} pour éviter ce problème.

    \begin{Verbatim}[commandchars=\\\{\}]
{\color{incolor}In [{\color{incolor}4}]:} \PY{l+s+s1}{\PYZsq{}}\PY{l+s+s1}{abc;def;ghi;jkl;}\PY{l+s+s1}{\PYZsq{}}\PY{o}{.}\PY{n}{split}\PY{p}{(}\PY{l+s+s1}{\PYZsq{}}\PY{l+s+s1}{;}\PY{l+s+s1}{\PYZsq{}}\PY{p}{)}
\end{Verbatim}


\begin{Verbatim}[commandchars=\\\{\}]
{\color{outcolor}Out[{\color{outcolor}4}]:} ['abc', 'def', 'ghi', 'jkl', '']
\end{Verbatim}
            
    Qui s'inverse correctement cependant~:

    \begin{Verbatim}[commandchars=\\\{\}]
{\color{incolor}In [{\color{incolor}5}]:} \PY{l+s+s2}{\PYZdq{}}\PY{l+s+s2}{;}\PY{l+s+s2}{\PYZdq{}}\PY{o}{.}\PY{n}{join}\PY{p}{(}\PY{p}{[}\PY{l+s+s1}{\PYZsq{}}\PY{l+s+s1}{abc}\PY{l+s+s1}{\PYZsq{}}\PY{p}{,} \PY{l+s+s1}{\PYZsq{}}\PY{l+s+s1}{def}\PY{l+s+s1}{\PYZsq{}}\PY{p}{,} \PY{l+s+s1}{\PYZsq{}}\PY{l+s+s1}{ghi}\PY{l+s+s1}{\PYZsq{}}\PY{p}{,} \PY{l+s+s1}{\PYZsq{}}\PY{l+s+s1}{jkl}\PY{l+s+s1}{\PYZsq{}}\PY{p}{,} \PY{l+s+s1}{\PYZsq{}}\PY{l+s+s1}{\PYZsq{}}\PY{p}{]}\PY{p}{)}
\end{Verbatim}


\begin{Verbatim}[commandchars=\\\{\}]
{\color{outcolor}Out[{\color{outcolor}5}]:} 'abc;def;ghi;jkl;'
\end{Verbatim}
            
    \hypertarget{remplacement-replace}{%
\subsubsection{\texorpdfstring{Remplacement~:
\texttt{replace}}{Remplacement~: replace}}\label{remplacement-replace}}

    \texttt{replace} est très pratique pour remplacer une sous-chaîne par
une autre, avec une limite éventuelle sur le nombre de remplacements~:

    \begin{Verbatim}[commandchars=\\\{\}]
{\color{incolor}In [{\color{incolor}6}]:} \PY{l+s+s2}{\PYZdq{}}\PY{l+s+s2}{abcdefabcdefabcdef}\PY{l+s+s2}{\PYZdq{}}\PY{o}{.}\PY{n}{replace}\PY{p}{(}\PY{l+s+s2}{\PYZdq{}}\PY{l+s+s2}{abc}\PY{l+s+s2}{\PYZdq{}}\PY{p}{,} \PY{l+s+s2}{\PYZdq{}}\PY{l+s+s2}{zoo}\PY{l+s+s2}{\PYZdq{}}\PY{p}{)}
\end{Verbatim}


\begin{Verbatim}[commandchars=\\\{\}]
{\color{outcolor}Out[{\color{outcolor}6}]:} 'zoodefzoodefzoodef'
\end{Verbatim}
            
    \begin{Verbatim}[commandchars=\\\{\}]
{\color{incolor}In [{\color{incolor}7}]:} \PY{l+s+s2}{\PYZdq{}}\PY{l+s+s2}{abcdefabcdefabcdef}\PY{l+s+s2}{\PYZdq{}}\PY{o}{.}\PY{n}{replace}\PY{p}{(}\PY{l+s+s2}{\PYZdq{}}\PY{l+s+s2}{abc}\PY{l+s+s2}{\PYZdq{}}\PY{p}{,} \PY{l+s+s2}{\PYZdq{}}\PY{l+s+s2}{zoo}\PY{l+s+s2}{\PYZdq{}}\PY{p}{,} \PY{l+m+mi}{2}\PY{p}{)}
\end{Verbatim}


\begin{Verbatim}[commandchars=\\\{\}]
{\color{outcolor}Out[{\color{outcolor}7}]:} 'zoodefzoodefabcdef'
\end{Verbatim}
            
    Plusieurs appels à \texttt{replace} peuvent être chaînés comme ceci~:

    \begin{Verbatim}[commandchars=\\\{\}]
{\color{incolor}In [{\color{incolor}8}]:} \PY{l+s+s2}{\PYZdq{}}\PY{l+s+s2}{les [x] qui disent [y]}\PY{l+s+s2}{\PYZdq{}}\PY{o}{.}\PY{n}{replace}\PY{p}{(}\PY{l+s+s2}{\PYZdq{}}\PY{l+s+s2}{[x]}\PY{l+s+s2}{\PYZdq{}}\PY{p}{,} \PY{l+s+s2}{\PYZdq{}}\PY{l+s+s2}{chevaliers}\PY{l+s+s2}{\PYZdq{}}\PY{p}{)}\PY{o}{.}\PY{n}{replace}\PY{p}{(}\PY{l+s+s2}{\PYZdq{}}\PY{l+s+s2}{[y]}\PY{l+s+s2}{\PYZdq{}}\PY{p}{,} \PY{l+s+s2}{\PYZdq{}}\PY{l+s+s2}{Ni}\PY{l+s+s2}{\PYZdq{}}\PY{p}{)}
\end{Verbatim}


\begin{Verbatim}[commandchars=\\\{\}]
{\color{outcolor}Out[{\color{outcolor}8}]:} 'les chevaliers qui disent Ni'
\end{Verbatim}
            
    \hypertarget{nettoyage-strip}{%
\subsubsection{\texorpdfstring{Nettoyage~:
\texttt{strip}}{Nettoyage~: strip}}\label{nettoyage-strip}}

    On pourrait par exemple utiliser \texttt{replace} pour enlever les
espaces dans une chaîne, ce qui peut être utile pour ``nettoyer'' comme
ceci~:

    \begin{Verbatim}[commandchars=\\\{\}]
{\color{incolor}In [{\color{incolor}9}]:} \PY{l+s+s2}{\PYZdq{}}\PY{l+s+s2}{ abc:def:ghi }\PY{l+s+s2}{\PYZdq{}}\PY{o}{.}\PY{n}{replace}\PY{p}{(}\PY{l+s+s2}{\PYZdq{}}\PY{l+s+s2}{ }\PY{l+s+s2}{\PYZdq{}}\PY{p}{,} \PY{l+s+s2}{\PYZdq{}}\PY{l+s+s2}{\PYZdq{}}\PY{p}{)}
\end{Verbatim}


\begin{Verbatim}[commandchars=\\\{\}]
{\color{outcolor}Out[{\color{outcolor}9}]:} 'abc:def:ghi'
\end{Verbatim}
            
    Toutefois bien souvent on préfère utiliser \texttt{strip} qui ne
s'occupe que du début et de la fin de la chaîne, et gère aussi les
tabulations et autres retour à la ligne~:

    \begin{Verbatim}[commandchars=\\\{\}]
{\color{incolor}In [{\color{incolor}10}]:} \PY{l+s+s2}{\PYZdq{}}\PY{l+s+s2}{ }\PY{l+s+se}{\PYZbs{}t}\PY{l+s+s2}{une chaîne avec des trucs qui dépassent }\PY{l+s+se}{\PYZbs{}n}\PY{l+s+s2}{\PYZdq{}}\PY{o}{.}\PY{n}{strip}\PY{p}{(}\PY{p}{)}
\end{Verbatim}


\begin{Verbatim}[commandchars=\\\{\}]
{\color{outcolor}Out[{\color{outcolor}10}]:} 'une chaîne avec des trucs qui dépassent'
\end{Verbatim}
            
    On peut appliquer \texttt{strip} avant \texttt{split} pour éviter le
problème du dernier élément vide~:

    \begin{Verbatim}[commandchars=\\\{\}]
{\color{incolor}In [{\color{incolor}11}]:} \PY{l+s+s1}{\PYZsq{}}\PY{l+s+s1}{abc;def;ghi;jkl;}\PY{l+s+s1}{\PYZsq{}}\PY{o}{.}\PY{n}{strip}\PY{p}{(}\PY{l+s+s1}{\PYZsq{}}\PY{l+s+s1}{;}\PY{l+s+s1}{\PYZsq{}}\PY{p}{)}\PY{o}{.}\PY{n}{split}\PY{p}{(}\PY{l+s+s1}{\PYZsq{}}\PY{l+s+s1}{;}\PY{l+s+s1}{\PYZsq{}}\PY{p}{)}
\end{Verbatim}


\begin{Verbatim}[commandchars=\\\{\}]
{\color{outcolor}Out[{\color{outcolor}11}]:} ['abc', 'def', 'ghi', 'jkl']
\end{Verbatim}
            
    \hypertarget{rechercher-une-sous-chauxeene}{%
\subsubsection{Rechercher une
sous-chaîne}\label{rechercher-une-sous-chauxeene}}

    Plusieurs outils permettent de chercher une sous-chaîne. Il existe
\texttt{find} qui renvoie le plus petit index où on trouve la
sous-chaîne~:

    \begin{Verbatim}[commandchars=\\\{\}]
{\color{incolor}In [{\color{incolor}12}]:} \PY{c+c1}{\PYZsh{} l\PYZsq{}indice du début de la première occurrence}
         \PY{l+s+s2}{\PYZdq{}}\PY{l+s+s2}{abcdefcdefghefghijk}\PY{l+s+s2}{\PYZdq{}}\PY{o}{.}\PY{n}{find}\PY{p}{(}\PY{l+s+s2}{\PYZdq{}}\PY{l+s+s2}{def}\PY{l+s+s2}{\PYZdq{}}\PY{p}{)}
\end{Verbatim}


\begin{Verbatim}[commandchars=\\\{\}]
{\color{outcolor}Out[{\color{outcolor}12}]:} 3
\end{Verbatim}
            
    \begin{Verbatim}[commandchars=\\\{\}]
{\color{incolor}In [{\color{incolor}13}]:} \PY{c+c1}{\PYZsh{} ou \PYZhy{}1 si la chaîne n\PYZsq{}est pas présente}
         \PY{l+s+s2}{\PYZdq{}}\PY{l+s+s2}{abcdefcdefghefghijk}\PY{l+s+s2}{\PYZdq{}}\PY{o}{.}\PY{n}{find}\PY{p}{(}\PY{l+s+s2}{\PYZdq{}}\PY{l+s+s2}{zoo}\PY{l+s+s2}{\PYZdq{}}\PY{p}{)}
\end{Verbatim}


\begin{Verbatim}[commandchars=\\\{\}]
{\color{outcolor}Out[{\color{outcolor}13}]:} -1
\end{Verbatim}
            
    \texttt{rfind} fonctionne comme \texttt{find} mais en partant de la fin
de la chaîne~:

    \begin{Verbatim}[commandchars=\\\{\}]
{\color{incolor}In [{\color{incolor}14}]:} \PY{c+c1}{\PYZsh{} en partant de la fin}
         \PY{l+s+s2}{\PYZdq{}}\PY{l+s+s2}{abcdefcdefghefghijk}\PY{l+s+s2}{\PYZdq{}}\PY{o}{.}\PY{n}{rfind}\PY{p}{(}\PY{l+s+s2}{\PYZdq{}}\PY{l+s+s2}{fgh}\PY{l+s+s2}{\PYZdq{}}\PY{p}{)}
\end{Verbatim}


\begin{Verbatim}[commandchars=\\\{\}]
{\color{outcolor}Out[{\color{outcolor}14}]:} 13
\end{Verbatim}
            
    \begin{Verbatim}[commandchars=\\\{\}]
{\color{incolor}In [{\color{incolor}15}]:} \PY{c+c1}{\PYZsh{} notez que le résultat correspond}
         \PY{c+c1}{\PYZsh{} tout de même toujours au début de la chaîne}
         \PY{l+s+s2}{\PYZdq{}}\PY{l+s+s2}{abcdefcdefghefghijk}\PY{l+s+s2}{\PYZdq{}}\PY{p}{[}\PY{l+m+mi}{13}\PY{p}{]}
\end{Verbatim}


\begin{Verbatim}[commandchars=\\\{\}]
{\color{outcolor}Out[{\color{outcolor}15}]:} 'f'
\end{Verbatim}
            
    La méthode \texttt{index} se comporte comme \texttt{find}, mais en cas
d'absence elle lève une \textbf{exception} (nous verrons ce concept plus
tard) plutôt que de renvoyer \texttt{-1}~:

    \begin{Verbatim}[commandchars=\\\{\}]
{\color{incolor}In [{\color{incolor}16}]:} \PY{l+s+s2}{\PYZdq{}}\PY{l+s+s2}{abcdefcdefghefghijk}\PY{l+s+s2}{\PYZdq{}}\PY{o}{.}\PY{n}{index}\PY{p}{(}\PY{l+s+s2}{\PYZdq{}}\PY{l+s+s2}{def}\PY{l+s+s2}{\PYZdq{}}\PY{p}{)}
\end{Verbatim}


\begin{Verbatim}[commandchars=\\\{\}]
{\color{outcolor}Out[{\color{outcolor}16}]:} 3
\end{Verbatim}
            
    \begin{Verbatim}[commandchars=\\\{\}]
{\color{incolor}In [{\color{incolor}17}]:} \PY{k}{try}\PY{p}{:}
             \PY{l+s+s2}{\PYZdq{}}\PY{l+s+s2}{abcdefcdefghefghijk}\PY{l+s+s2}{\PYZdq{}}\PY{o}{.}\PY{n}{index}\PY{p}{(}\PY{l+s+s2}{\PYZdq{}}\PY{l+s+s2}{zoo}\PY{l+s+s2}{\PYZdq{}}\PY{p}{)}
         \PY{k}{except} \PY{n+ne}{Exception} \PY{k}{as} \PY{n}{e}\PY{p}{:}
             \PY{n+nb}{print}\PY{p}{(}\PY{l+s+s2}{\PYZdq{}}\PY{l+s+s2}{OOPS}\PY{l+s+s2}{\PYZdq{}}\PY{p}{,} \PY{n+nb}{type}\PY{p}{(}\PY{n}{e}\PY{p}{)}\PY{p}{,} \PY{n}{e}\PY{p}{)}
\end{Verbatim}


    \begin{Verbatim}[commandchars=\\\{\}]
OOPS <class 'ValueError'> substring not found

    \end{Verbatim}

    Mais le plus simple pour chercher si une sous-chaîne est dans une autre
chaîne est d'utiliser l'instruction \texttt{in} sur laquelle nous
reviendrons lorsque nous parlerons des séquences~:

    \begin{Verbatim}[commandchars=\\\{\}]
{\color{incolor}In [{\color{incolor}18}]:} \PY{l+s+s2}{\PYZdq{}}\PY{l+s+s2}{def}\PY{l+s+s2}{\PYZdq{}} \PY{o+ow}{in} \PY{l+s+s2}{\PYZdq{}}\PY{l+s+s2}{abcdefcdefghefghijk}\PY{l+s+s2}{\PYZdq{}}
\end{Verbatim}


\begin{Verbatim}[commandchars=\\\{\}]
{\color{outcolor}Out[{\color{outcolor}18}]:} True
\end{Verbatim}
            
    La méthode \texttt{count} compte le nombre d'occurrences d'une
sous-chaîne~:

    \begin{Verbatim}[commandchars=\\\{\}]
{\color{incolor}In [{\color{incolor}19}]:} \PY{l+s+s2}{\PYZdq{}}\PY{l+s+s2}{abcdefcdefghefghijk}\PY{l+s+s2}{\PYZdq{}}\PY{o}{.}\PY{n}{count}\PY{p}{(}\PY{l+s+s2}{\PYZdq{}}\PY{l+s+s2}{ef}\PY{l+s+s2}{\PYZdq{}}\PY{p}{)}
\end{Verbatim}


\begin{Verbatim}[commandchars=\\\{\}]
{\color{outcolor}Out[{\color{outcolor}19}]:} 3
\end{Verbatim}
            
    Signalons enfin les méthodes de commodité suivantes~:

    \begin{Verbatim}[commandchars=\\\{\}]
{\color{incolor}In [{\color{incolor}20}]:} \PY{l+s+s2}{\PYZdq{}}\PY{l+s+s2}{abcdefcdefghefghijk}\PY{l+s+s2}{\PYZdq{}}\PY{o}{.}\PY{n}{startswith}\PY{p}{(}\PY{l+s+s2}{\PYZdq{}}\PY{l+s+s2}{abcd}\PY{l+s+s2}{\PYZdq{}}\PY{p}{)}
\end{Verbatim}


\begin{Verbatim}[commandchars=\\\{\}]
{\color{outcolor}Out[{\color{outcolor}20}]:} True
\end{Verbatim}
            
    \begin{Verbatim}[commandchars=\\\{\}]
{\color{incolor}In [{\color{incolor}21}]:} \PY{l+s+s2}{\PYZdq{}}\PY{l+s+s2}{abcdefcdefghefghijk}\PY{l+s+s2}{\PYZdq{}}\PY{o}{.}\PY{n}{endswith}\PY{p}{(}\PY{l+s+s2}{\PYZdq{}}\PY{l+s+s2}{ghijk}\PY{l+s+s2}{\PYZdq{}}\PY{p}{)}
\end{Verbatim}


\begin{Verbatim}[commandchars=\\\{\}]
{\color{outcolor}Out[{\color{outcolor}21}]:} True
\end{Verbatim}
            
    S'agissant des deux dernières, remarquons que~:\\

    \texttt{chaine.startswith(sous\_chaine)} \(\Longleftrightarrow\)
\texttt{chaine.find(sous\_chaine)\ ==\ 0}\\

\texttt{chaine.endswith(sous\_chaine)} \(\Longleftrightarrow\)
\texttt{chaine.rfind(sous\_chaine)\ ==\ (len(chaine)\ -\ len(sous\_chaine))}\\

    On remarque ici la supériorité, du point de vue de l'expressivité, des
méthodes pythoniques \texttt{startswith} et \texttt{endswith}.

    \hypertarget{changement-de-casse}{%
\subsubsection{Changement de casse}\label{changement-de-casse}}

    Voici pour conclure quelques méthodes utiles qui parlent d'elles-mêmes~:

    \begin{Verbatim}[commandchars=\\\{\}]
{\color{incolor}In [{\color{incolor}22}]:} \PY{l+s+s2}{\PYZdq{}}\PY{l+s+s2}{monty PYTHON}\PY{l+s+s2}{\PYZdq{}}\PY{o}{.}\PY{n}{upper}\PY{p}{(}\PY{p}{)}
\end{Verbatim}


\begin{Verbatim}[commandchars=\\\{\}]
{\color{outcolor}Out[{\color{outcolor}22}]:} 'MONTY PYTHON'
\end{Verbatim}
            
    \begin{Verbatim}[commandchars=\\\{\}]
{\color{incolor}In [{\color{incolor}23}]:} \PY{l+s+s2}{\PYZdq{}}\PY{l+s+s2}{monty PYTHON}\PY{l+s+s2}{\PYZdq{}}\PY{o}{.}\PY{n}{lower}\PY{p}{(}\PY{p}{)}
\end{Verbatim}


\begin{Verbatim}[commandchars=\\\{\}]
{\color{outcolor}Out[{\color{outcolor}23}]:} 'monty python'
\end{Verbatim}
            
    \begin{Verbatim}[commandchars=\\\{\}]
{\color{incolor}In [{\color{incolor}24}]:} \PY{l+s+s2}{\PYZdq{}}\PY{l+s+s2}{monty PYTHON}\PY{l+s+s2}{\PYZdq{}}\PY{o}{.}\PY{n}{swapcase}\PY{p}{(}\PY{p}{)}
\end{Verbatim}


\begin{Verbatim}[commandchars=\\\{\}]
{\color{outcolor}Out[{\color{outcolor}24}]:} 'MONTY python'
\end{Verbatim}
            
    \begin{Verbatim}[commandchars=\\\{\}]
{\color{incolor}In [{\color{incolor}25}]:} \PY{l+s+s2}{\PYZdq{}}\PY{l+s+s2}{monty PYTHON}\PY{l+s+s2}{\PYZdq{}}\PY{o}{.}\PY{n}{capitalize}\PY{p}{(}\PY{p}{)}
\end{Verbatim}


\begin{Verbatim}[commandchars=\\\{\}]
{\color{outcolor}Out[{\color{outcolor}25}]:} 'Monty python'
\end{Verbatim}
            
    \begin{Verbatim}[commandchars=\\\{\}]
{\color{incolor}In [{\color{incolor}26}]:} \PY{l+s+s2}{\PYZdq{}}\PY{l+s+s2}{monty PYTHON}\PY{l+s+s2}{\PYZdq{}}\PY{o}{.}\PY{n}{title}\PY{p}{(}\PY{p}{)}
\end{Verbatim}


\begin{Verbatim}[commandchars=\\\{\}]
{\color{outcolor}Out[{\color{outcolor}26}]:} 'Monty Python'
\end{Verbatim}
            
    \hypertarget{pour-en-savoir-plus}{%
\subsubsection{Pour en savoir plus}\label{pour-en-savoir-plus}}

    Tous ces outils sont
\href{https://docs.python.org/3/library/stdtypes.html\#string-methods}{documentés
en détail ici (en anglais)}.