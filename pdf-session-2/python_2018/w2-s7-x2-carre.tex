    \hypertarget{compruxe9hensions-2}{%
\section{Compréhensions (2)}\label{compruxe9hensions-2}}

    \hypertarget{exercice---niveau-intermuxe9diaire}{%
\subsection{Exercice - niveau
intermédiaire}\label{exercice---niveau-intermuxe9diaire}}

    \hypertarget{mise-au-carruxe9}{%
\subsubsection{Mise au carré}\label{mise-au-carruxe9}}

    \begin{Verbatim}[commandchars=\\\{\}]
{\color{incolor}In [{\color{incolor} }]:} \PY{c+c1}{\PYZsh{} chargement de l\PYZsq{}exercice}
        \PY{k+kn}{from} \PY{n+nn}{corrections}\PY{n+nn}{.}\PY{n+nn}{exo\PYZus{}carre} \PY{k}{import} \PY{n}{exo\PYZus{}carre}
\end{Verbatim}


    On vous demande à présent d'écrire une fonction dans le même esprit que
ci-dessus. Cette fois, chaque ligne contient, séparés par des
points-virgules, une liste d'entiers, et on veut obtenir une nouvelle
chaîne avec les carrés de ces entiers, séparés par des deux-points.\\

À nouveau les lignes peuvent être remplies de manière approximative,
avec des espaces, des tabulations, ou même des points-virgules en trop,
que ce soit au début, à la fin, ou au milieu d'une ligne.

    \begin{Verbatim}[commandchars=\\\{\}]
{\color{incolor}In [{\color{incolor} }]:} \PY{c+c1}{\PYZsh{} exemples}
        \PY{n}{exo\PYZus{}carre}\PY{o}{.}\PY{n}{example}\PY{p}{(}\PY{p}{)}
\end{Verbatim}


    \begin{Verbatim}[commandchars=\\\{\}]
{\color{incolor}In [{\color{incolor} }]:} \PY{c+c1}{\PYZsh{} écrivez votre code ici}
        \PY{k}{def} \PY{n+nf}{carre}\PY{p}{(}\PY{n}{ligne}\PY{p}{)}\PY{p}{:}
            \PY{l+s+s2}{\PYZdq{}}\PY{l+s+s2}{\PYZlt{}votre\PYZus{}code\PYZgt{}}\PY{l+s+s2}{\PYZdq{}}
\end{Verbatim}


    \begin{Verbatim}[commandchars=\\\{\}]
{\color{incolor}In [{\color{incolor} }]:} \PY{c+c1}{\PYZsh{} pour corriger}
        \PY{n}{exo\PYZus{}carre}\PY{o}{.}\PY{n}{correction}\PY{p}{(}\PY{n}{carre}\PY{p}{)}
\end{Verbatim}