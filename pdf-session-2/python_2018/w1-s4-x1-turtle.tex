    \hypertarget{dessiner-un-carruxe9}{%
\section{Dessiner un carré}\label{dessiner-un-carruxe9}}

    \hypertarget{exercice---niveau-intermuxe9diaire}{%
\subsection{Exercice - niveau
intermédiaire}\label{exercice---niveau-intermuxe9diaire}}

    Voici un tout petit programme qui dessine un carré.\\

    Il utilise le module \texttt{turtle}, conçu précisément à des fins
pédagogiques. Pour des raisons techniques, le module \texttt{turtle}
n'est \textbf{pas disponible} au travers de la plateforme FUN.\\

    \textbf{Il est donc inutile d'essayer d'exécuter ce programme depuis le
notebook}. L'objectif de cet exercice est plutôt de vous entraîner à
télécharger ce programme en utilisant le menu
\emph{``File~-\textgreater{}~Download~as~-\textgreater{}~Python''}, puis
à le charger dans votre IDLE pour l'exécuter sur votre machine.\\

    \textbf{Attention} également à sauver le programme téléchargé
\textbf{sous un autre nom} que \texttt{turtle.py}, car sinon vous allez
empêcher python de trouver le module standard \texttt{turtle}~;
appelez-le par exemple \texttt{turtle\_basic.py}.

    \begin{Verbatim}[commandchars=\\\{\}]
{\color{incolor}In [{\color{incolor} }]:} \PY{c+c1}{\PYZsh{} on a besoin du module turtle}
        \PY{k+kn}{import} \PY{n+nn}{turtle}
\end{Verbatim}


    On commence par définir une fonction qui dessine un carré de coté
\texttt{length}~:

    \begin{Verbatim}[commandchars=\\\{\}]
{\color{incolor}In [{\color{incolor} }]:} \PY{k}{def} \PY{n+nf}{square}\PY{p}{(}\PY{n}{length}\PY{p}{)}\PY{p}{:}
            \PY{l+s+s2}{\PYZdq{}}\PY{l+s+s2}{have the turtle draw a square of side \PYZlt{}length\PYZgt{}}\PY{l+s+s2}{\PYZdq{}}
            \PY{k}{for} \PY{n}{side} \PY{o+ow}{in} \PY{n+nb}{range}\PY{p}{(}\PY{l+m+mi}{4}\PY{p}{)}\PY{p}{:}
                \PY{n}{turtle}\PY{o}{.}\PY{n}{forward}\PY{p}{(}\PY{n}{length}\PY{p}{)}
                \PY{n}{turtle}\PY{o}{.}\PY{n}{left}\PY{p}{(}\PY{l+m+mi}{90}\PY{p}{)}
\end{Verbatim}


    Maintenant on commence par initialiser la tortue~:

    \begin{Verbatim}[commandchars=\\\{\}]
{\color{incolor}In [{\color{incolor} }]:} \PY{n}{turtle}\PY{o}{.}\PY{n}{reset}\PY{p}{(}\PY{p}{)}
\end{Verbatim}


    On peut alors dessiner notre carré~:

    \begin{Verbatim}[commandchars=\\\{\}]
{\color{incolor}In [{\color{incolor} }]:} \PY{n}{square}\PY{p}{(}\PY{l+m+mi}{200}\PY{p}{)}
\end{Verbatim}


    Et pour finir on attend que l'utilisateur clique dans la fenêtre de la
tortue, et alors on termine~:

    \begin{Verbatim}[commandchars=\\\{\}]
{\color{incolor}In [{\color{incolor} }]:} \PY{n}{turtle}\PY{o}{.}\PY{n}{exitonclick}\PY{p}{(}\PY{p}{)}
\end{Verbatim}


    \hypertarget{exercice---niveau-avancuxe9}{%
\subsection{Exercice - niveau
avancé}\label{exercice---niveau-avancuxe9}}

    Naturellement vous pouvez vous amuser à modifier ce code pour dessiner
des choses un peu plus amusantes.\\

Dans ce cas, commencez par chercher ``\emph{module python turtle}'' dans
votre moteur de recherche favori, pour localiser la documentation du
module
\href{https://docs.python.org/3/library/turtle.html}{\texttt{turtle}}.\\

Vous trouverez quelques exemples pour commencer ici~:
\begin{itemize}
	\item 
	\href{media/turtle_multi_squares.py}{turtle\_multi\_squares.py} pour
	dessiner des carrés à l'emplacement de la souris en utilisant plusieurs
	tortues~;
	\item
	\href{media/turtle_fractal.py}{turtle\_fractal.py} pour
	dessiner une fractale simple~;
	\item \href{media/turtle_fractal_reglable.py}{turtle\_fractal\_reglable.py}
	une variation sur la fractale, plus paramétrable.
\end{itemize}