    
    
    
    

    

    \hypertarget{listes-infinies-ruxe9fuxe9rences-circulaires}{%
\section{Listes infinies \& références
circulaires}\label{listes-infinies-ruxe9fuxe9rences-circulaires}}

    \hypertarget{compluxe9ment---niveau-intermuxe9diaire}{%
\subsection{Complément - niveau
intermédiaire}\label{compluxe9ment---niveau-intermuxe9diaire}}

    \begin{Verbatim}[commandchars=\\\{\},frame=single,framerule=0.3mm,rulecolor=\color{cellframecolor}]
{\color{incolor}In [{\color{incolor}1}]:} \PY{o}{\PYZpc{}}\PY{k}{load\PYZus{}ext} ipythontutor
\end{Verbatim}


    Nous allons maintenant construire un objet un peu abscons. Cet exemple
précis n'a aucune utilité pratique, mais permet de bien comprendre la
logique du langage.

    Construisons une liste à un seul élément, peu importe quoi~:

    \begin{Verbatim}[commandchars=\\\{\},frame=single,framerule=0.3mm,rulecolor=\color{cellframecolor}]
{\color{incolor}In [{\color{incolor}2}]:} \PY{n}{infini\PYZus{}1} \PY{o}{=} \PY{p}{[}\PY{k+kc}{None}\PY{p}{]}
\end{Verbatim}


    À présent nous allons remplacer le premier et seul élément de la liste
par\ldots{} la liste elle-même~:

    \begin{Verbatim}[commandchars=\\\{\},frame=single,framerule=0.3mm,rulecolor=\color{cellframecolor}]
{\color{incolor}In [{\color{incolor}3}]:} \PY{n}{infini\PYZus{}1}\PY{p}{[}\PY{l+m+mi}{0}\PY{p}{]} \PY{o}{=} \PY{n}{infini\PYZus{}1}
        \PY{n+nb}{print}\PY{p}{(}\PY{n}{infini\PYZus{}1}\PY{p}{)}
\end{Verbatim}


    \begin{Verbatim}[commandchars=\\\{\},frame=single,framerule=0.3mm,rulecolor=\color{cellframecolor}]
[[{\ldots}]]
\end{Verbatim}

    Pour essayer de décrire l'objet liste ainsi obtenu, on pourrait dire
qu'il s'agit d'une liste de taille 1 et de profondeur infinie, une sorte
de fil infini en quelque sorte.

Naturellement, l'objet obtenu est difficile à imprimer de manière
convaincante. Pour faire en sorte que cet objet soit tout de même
imprimable, et éviter une boucle infinie, python utilise l'ellipse
\texttt{...} pour indiquer ce qu'on appelle une référence circulaire. Si
on n'y prenait pas garde en effet, il faudrait écrire
\texttt{{[}{[}{[}{[}\ etc.\ {]}{]}{]}{]}} avec une infinité de crochets.

    Voici la même séquence exécutée sous \url{http://pythontutor.com}~; il
s'agit d'un site très utile pour comprendre comment python implémente
les objets, les références et les partages.

Cliquez sur le bouton \texttt{Forward} pour avancer dans l'exécution de
la séquence. À la fin de la séquence vous verrez - ce n'est pas
forcément clair - la seule cellule de la liste à se référencer
elle-même~:

    \begin{Verbatim}[commandchars=\\\{\},frame=single,framerule=0.3mm,rulecolor=\color{cellframecolor}]
{\color{incolor}In [{\color{incolor} }]:} \PY{c+c1}{\PYZsh{} NOTE}
        \PY{c+c1}{\PYZsh{} auto\PYZhy{}exec\PYZhy{}for\PYZhy{}latex has skipped execution of this cell}
        
        \PY{o}{\PYZpc{}\PYZpc{}}\PY{k}{ipythontutor} height=230
        infini\PYZus{}1 = [None]
        infini\PYZus{}1[0] = infini\PYZus{}1
\end{Verbatim}


    Toutes les fonctions de python ne sont pas aussi intelligentes que
\texttt{print}. Bien qu'on puisse comparer cette liste avec elle-même~:

    \begin{Verbatim}[commandchars=\\\{\},frame=single,framerule=0.3mm,rulecolor=\color{cellframecolor}]
{\color{incolor}In [{\color{incolor}4}]:} \PY{n}{infini\PYZus{}1} \PY{o}{==} \PY{n}{infini\PYZus{}1}
\end{Verbatim}


\begin{Verbatim}[commandchars=\\\{\},frame=single,framerule=0.3mm,rulecolor=\color{cellframecolor}]
{\color{outcolor}Out[{\color{outcolor}4}]:} True
\end{Verbatim}
            
    il n'en est pas de même si on la compare avec un objet analogue mais pas
identique~:

    \begin{Verbatim}[commandchars=\\\{\},frame=single,framerule=0.3mm,rulecolor=\color{cellframecolor}]
{\color{incolor}In [{\color{incolor}5}]:} \PY{n}{infini\PYZus{}2} \PY{o}{=} \PY{p}{[}\PY{l+m+mi}{0}\PY{p}{]}
        \PY{n}{infini\PYZus{}2}\PY{p}{[}\PY{l+m+mi}{0}\PY{p}{]} \PY{o}{=} \PY{n}{infini\PYZus{}2}
        \PY{n+nb}{print}\PY{p}{(}\PY{n}{infini\PYZus{}2}\PY{p}{)}
\end{Verbatim}


    \begin{Verbatim}[commandchars=\\\{\},frame=single,framerule=0.3mm,rulecolor=\color{cellframecolor}]
[[{\ldots}]]
\end{Verbatim}

    \begin{Verbatim}[commandchars=\\\{\},frame=single,framerule=0.3mm,rulecolor=\color{cellframecolor}]
{\color{incolor}In [{\color{incolor} }]:} \PY{c+c1}{\PYZsh{} NOTE}
        \PY{c+c1}{\PYZsh{} auto\PYZhy{}exec\PYZhy{}for\PYZhy{}latex has skipped execution of this cell}
        
        \PY{c+c1}{\PYZsh{} attention, ceci provoque une erreur à l\PYZsq{}exécution}
        \PY{c+c1}{\PYZsh{} RecursionError: maximum recursion depth exceeded in comparison}
        \PY{n}{infini\PYZus{}1} \PY{o}{==} \PY{n}{infini\PYZus{}2}
\end{Verbatim}


    \hypertarget{guxe9nuxe9ralisation-aux-ruxe9fuxe9rences-circulaires}{%
\subsubsection{Généralisation aux références
circulaires}\label{guxe9nuxe9ralisation-aux-ruxe9fuxe9rences-circulaires}}

    On obtient un phénomène équivalent dès lors qu'un élément contenu dans
un objet fait référence à l'objet lui-même. Voici par exemple comment on
peut construire un dictionnaire qui contient une référence circulaire~:

    \begin{Verbatim}[commandchars=\\\{\},frame=single,framerule=0.3mm,rulecolor=\color{cellframecolor}]
{\color{incolor}In [{\color{incolor}6}]:} \PY{n}{collection\PYZus{}de\PYZus{}points} \PY{o}{=} \PY{p}{[}
            \PY{p}{\PYZob{}}\PY{l+s+s1}{\PYZsq{}}\PY{l+s+s1}{x}\PY{l+s+s1}{\PYZsq{}}\PY{p}{:} \PY{l+m+mi}{10}\PY{p}{,}\PY{l+s+s1}{\PYZsq{}}\PY{l+s+s1}{y}\PY{l+s+s1}{\PYZsq{}}\PY{p}{:} \PY{l+m+mi}{20}\PY{p}{\PYZcb{}}\PY{p}{,}
            \PY{p}{\PYZob{}}\PY{l+s+s1}{\PYZsq{}}\PY{l+s+s1}{x}\PY{l+s+s1}{\PYZsq{}}\PY{p}{:} \PY{l+m+mi}{30}\PY{p}{,}\PY{l+s+s1}{\PYZsq{}}\PY{l+s+s1}{y}\PY{l+s+s1}{\PYZsq{}}\PY{p}{:} \PY{l+m+mi}{50}\PY{p}{\PYZcb{}}\PY{p}{,}
            \PY{c+c1}{\PYZsh{} imaginez plein de points}
        \PY{p}{]}
        
        \PY{c+c1}{\PYZsh{} on rajoute dans chaque dictionnaire une clé \PYZsq{}points\PYZsq{}}
        \PY{c+c1}{\PYZsh{} qui référence la collection complète}
        \PY{k}{for} \PY{n}{point} \PY{o+ow}{in} \PY{n}{collection\PYZus{}de\PYZus{}points}\PY{p}{:}
            \PY{n}{point}\PY{p}{[}\PY{l+s+s1}{\PYZsq{}}\PY{l+s+s1}{points}\PY{l+s+s1}{\PYZsq{}}\PY{p}{]} \PY{o}{=} \PY{n}{collection\PYZus{}de\PYZus{}points}
        
        \PY{c+c1}{\PYZsh{} la structure possède maintenant des références circulaires}
        \PY{n+nb}{print}\PY{p}{(}\PY{n}{collection\PYZus{}de\PYZus{}points}\PY{p}{)}
\end{Verbatim}


    \begin{Verbatim}[commandchars=\\\{\},frame=single,framerule=0.3mm,rulecolor=\color{cellframecolor}]
[\{'x': 10, 'y': 20, 'points': [{\ldots}]\}, \{'x': 30, 'y': 50, 'points': [{\ldots}]\}]
\end{Verbatim}

    On voit à nouveau réapparaître les ellipses, qui indiquent que pour
chaque point, le nouveau champ \texttt{points} est un objet qui a déjà
été imprimé.

Cette technique est cette fois très utile et très utilisée dans la
pratique, dès lors qu'on a besoin de naviguer de manière arbitraire dans
une structure de données compliquée. Dans cet exemple, pas très réaliste
naturellement, on pourrait à présent accéder depuis un point à tous les
autres points de la collection dont il fait partie.

    À nouveau il peut être intéressant de voir le comportement de cet
exemple avec \url{http://pythontutor.com} pour bien comprendre ce qui se
passe, si cela ne vous semble pas clair à première vue~:

    \begin{Verbatim}[commandchars=\\\{\},frame=single,framerule=0.3mm,rulecolor=\color{cellframecolor}]
{\color{incolor}In [{\color{incolor} }]:} \PY{c+c1}{\PYZsh{} NOTE}
        \PY{c+c1}{\PYZsh{} auto\PYZhy{}exec\PYZhy{}for\PYZhy{}latex has skipped execution of this cell}
        
        \PY{o}{\PYZpc{}\PYZpc{}}\PY{k}{ipythontutor} curInstr=7
        points = [
            \PYZob{}\PYZsq{}x\PYZsq{}: 10,\PYZsq{}y\PYZsq{}: 20\PYZcb{},
            \PYZob{}\PYZsq{}x\PYZsq{}: 30,\PYZsq{}y\PYZsq{}: 50\PYZcb{},
        ]
        
        for point in points:
            point[\PYZsq{}points\PYZsq{}] = points
\end{Verbatim}



    % Add a bibliography block to the postdoc
    
    
    
