    
    
    
    

    

    \hypertarget{le-broadcasting}{%
\section{\texorpdfstring{Le
\emph{broadcasting}}{Le broadcasting}}\label{le-broadcasting}}

    \begin{Verbatim}[commandchars=\\\{\},frame=single,framerule=0.3mm,rulecolor=\color{cellframecolor}]
{\color{incolor}In [{\color{incolor}1}]:} \PY{k+kn}{import} \PY{n+nn}{numpy} \PY{k}{as} \PY{n+nn}{np}
\end{Verbatim}


    \hypertarget{compluxe9ment---niveau-intermuxe9diaire}{%
\subsection{Complément - niveau
intermédiaire}\label{compluxe9ment---niveau-intermuxe9diaire}}

    Lorsque l'on a parlé de programmation vectorielle, on a vu que l'on
pouvait écrire quelque chose comme ceci~:

    \begin{Verbatim}[commandchars=\\\{\},frame=single,framerule=0.3mm,rulecolor=\color{cellframecolor}]
{\color{incolor}In [{\color{incolor}2}]:} \PY{n}{X} \PY{o}{=} \PY{n}{np}\PY{o}{.}\PY{n}{linspace}\PY{p}{(}\PY{l+m+mi}{0}\PY{p}{,} \PY{l+m+mi}{2} \PY{o}{*} \PY{n}{np}\PY{o}{.}\PY{n}{pi}\PY{p}{)}
        \PY{n}{Y} \PY{o}{=} \PY{n}{np}\PY{o}{.}\PY{n}{cos}\PY{p}{(}\PY{n}{X}\PY{p}{)} \PY{o}{+} \PY{n}{np}\PY{o}{.}\PY{n}{sin}\PY{p}{(}\PY{n}{X}\PY{p}{)} \PY{o}{+} \PY{l+m+mi}{2}
\end{Verbatim}


    Je vous fais remarquer que dans cette dernière ligne on combine~:

\begin{itemize}
\tightlist
\item
  deux tableaux de mêmes tailles - quand on ajoute \texttt{np.cos(X)}
  avec \texttt{np.sin(X)}~;
\item
  un tableau avec un scalaire - quand on ajoute \texttt{2} au résultat.
\end{itemize}

    En fait, le \emph{broadcasting} est ce qui permet~:

\begin{itemize}
\tightlist
\item
  d'unifier le sens de ces deux opérations~;
\item
  de donner du sens à des cas plus généraux, où on fait des opérations
  entre des tableaux qui ont des \emph{tailles différentes}, mais assez
  semblables pour que l'on puisse tout de même les combiner.
\end{itemize}

    \hypertarget{exemples-en-2d}{%
\subsection{Exemples en 2D}\label{exemples-en-2d}}

    Nous allons commencer par quelques exemples simples, avant de
généraliser le mécanisme. Pour commencer, nous nous donnons un tableau
de base~:

    \begin{Verbatim}[commandchars=\\\{\},frame=single,framerule=0.3mm,rulecolor=\color{cellframecolor}]
{\color{incolor}In [{\color{incolor}3}]:} \PY{n}{a} \PY{o}{=} \PY{l+m+mi}{100} \PY{o}{*} \PY{n}{np}\PY{o}{.}\PY{n}{ones}\PY{p}{(}\PY{p}{(}\PY{l+m+mi}{3}\PY{p}{,} \PY{l+m+mi}{5}\PY{p}{)}\PY{p}{,} \PY{n}{dtype}\PY{o}{=}\PY{n}{np}\PY{o}{.}\PY{n}{int32}\PY{p}{)}
        \PY{n+nb}{print}\PY{p}{(}\PY{n}{a}\PY{p}{)}
\end{Verbatim}


    \begin{Verbatim}[commandchars=\\\{\},frame=single,framerule=0.3mm,rulecolor=\color{cellframecolor}]
[[100 100 100 100 100]
 [100 100 100 100 100]
 [100 100 100 100 100]]
\end{Verbatim}

    Je vais illustrer le broadcasting avec l'opération \texttt{+}, mais bien
entendu ce mécanisme est à l'œuvre dès que vous faites des opérations
entre deux tableaux qui n'ont pas les mêmes dimensions.

Pour commencer, je vais donc ajouter à mon tableau de base un scalaire~:

    \hypertarget{broadcasting-entre-les-dimensions-3-5-et-1}{%
\subsubsection{\texorpdfstring{Broadcasting entre les dimensions
\texttt{(3,\ 5)} et
\texttt{(1,)}}{Broadcasting entre les dimensions (3, 5) et (1,)}}\label{broadcasting-entre-les-dimensions-3-5-et-1}}

    \begin{Verbatim}[commandchars=\\\{\},frame=single,framerule=0.3mm,rulecolor=\color{cellframecolor}]
{\color{incolor}In [{\color{incolor}4}]:} \PY{n+nb}{print}\PY{p}{(}\PY{n}{a}\PY{p}{)}
\end{Verbatim}


    \begin{Verbatim}[commandchars=\\\{\},frame=single,framerule=0.3mm,rulecolor=\color{cellframecolor}]
[[100 100 100 100 100]
 [100 100 100 100 100]
 [100 100 100 100 100]]
\end{Verbatim}

    \begin{Verbatim}[commandchars=\\\{\},frame=single,framerule=0.3mm,rulecolor=\color{cellframecolor}]
{\color{incolor}In [{\color{incolor}5}]:} \PY{n}{b} \PY{o}{=} \PY{l+m+mi}{3}
        \PY{n+nb}{print}\PY{p}{(}\PY{n}{b}\PY{p}{)}
\end{Verbatim}


    \begin{Verbatim}[commandchars=\\\{\},frame=single,framerule=0.3mm,rulecolor=\color{cellframecolor}]
3
\end{Verbatim}

    \begin{center}\rule{0.5\linewidth}{\linethickness}\end{center}

    Lorsque j'ajoute ces deux tableaux, c'est comme si j'avais ajouté à
\texttt{a} la différence~:

    \begin{Verbatim}[commandchars=\\\{\},frame=single,framerule=0.3mm,rulecolor=\color{cellframecolor}]
{\color{incolor}In [{\color{incolor}6}]:} \PY{c+c1}{\PYZsh{} pour élaborer c}
        \PY{n}{c} \PY{o}{=} \PY{n}{a} \PY{o}{+} \PY{n}{b}
        \PY{n+nb}{print}\PY{p}{(}\PY{n}{c}\PY{p}{)}
\end{Verbatim}


    \begin{Verbatim}[commandchars=\\\{\},frame=single,framerule=0.3mm,rulecolor=\color{cellframecolor}]
[[103 103 103 103 103]
 [103 103 103 103 103]
 [103 103 103 103 103]]
\end{Verbatim}

    \begin{Verbatim}[commandchars=\\\{\},frame=single,framerule=0.3mm,rulecolor=\color{cellframecolor}]
{\color{incolor}In [{\color{incolor}7}]:} \PY{c+c1}{\PYZsh{} c\PYZsq{}est comme si j\PYZsq{}avais}
        \PY{c+c1}{\PYZsh{} ajouté à a ce terme\PYZhy{}ci}
        \PY{n+nb}{print}\PY{p}{(}\PY{n}{c} \PY{o}{\PYZhy{}} \PY{n}{a}\PY{p}{)}
\end{Verbatim}


    \begin{Verbatim}[commandchars=\\\{\},frame=single,framerule=0.3mm,rulecolor=\color{cellframecolor}]
[[3 3 3 3 3]
 [3 3 3 3 3]
 [3 3 3 3 3]]
\end{Verbatim}

    C'est un premier cas particulier de \emph{broadcasting} dans sa version
extrême.

Le scalaire \texttt{b}, qui est en l'occurrence considéré comme un
tableau dont le \texttt{shape} vaut \texttt{(1,)}, est dupliqué dans les
deux directions jusqu'à obtenir ce tableau uniforme de taille
\texttt{(5,\ 3)} et qui contient un \texttt{3} partout.

Et c'est ce tableau, qui est maintenant de la même taille que
\texttt{a}, qui est ajouté à \texttt{a}.

    Je précise que cette explication est du domaine du modèle pédagogique~;
je ne dis pas que l'implémentation va réellement allouer un second
tableau, bien évidemment on peut optimiser pour éviter cette
construction inutile.

    \hypertarget{broadcasting-3-5-et-5}{%
\subsubsection{\texorpdfstring{Broadcasting \texttt{(3,\ 5)} et
\texttt{(5,)}}{Broadcasting (3, 5) et (5,)}}\label{broadcasting-3-5-et-5}}

    Voyons maintenant un cas un peu moins évident. Je peux ajouter à mon
tableau de base une ligne, c'est-à-dire un tableau de taille
\texttt{(5,\ )}. Voyons cela~:

    \begin{Verbatim}[commandchars=\\\{\},frame=single,framerule=0.3mm,rulecolor=\color{cellframecolor}]
{\color{incolor}In [{\color{incolor}8}]:} \PY{n+nb}{print}\PY{p}{(}\PY{n}{a}\PY{p}{)}
\end{Verbatim}


    \begin{Verbatim}[commandchars=\\\{\},frame=single,framerule=0.3mm,rulecolor=\color{cellframecolor}]
[[100 100 100 100 100]
 [100 100 100 100 100]
 [100 100 100 100 100]]
\end{Verbatim}

    \begin{Verbatim}[commandchars=\\\{\},frame=single,framerule=0.3mm,rulecolor=\color{cellframecolor}]
{\color{incolor}In [{\color{incolor}9}]:} \PY{n}{b} \PY{o}{=} \PY{n}{np}\PY{o}{.}\PY{n}{arange}\PY{p}{(}\PY{l+m+mi}{1}\PY{p}{,} \PY{l+m+mi}{6}\PY{p}{)}
        \PY{n+nb}{print}\PY{p}{(}\PY{n}{b}\PY{p}{)}
\end{Verbatim}


    \begin{Verbatim}[commandchars=\\\{\},frame=single,framerule=0.3mm,rulecolor=\color{cellframecolor}]
[1 2 3 4 5]
\end{Verbatim}

    \begin{Verbatim}[commandchars=\\\{\},frame=single,framerule=0.3mm,rulecolor=\color{cellframecolor}]
{\color{incolor}In [{\color{incolor}10}]:} \PY{n}{b}\PY{o}{.}\PY{n}{shape}
\end{Verbatim}


\begin{Verbatim}[commandchars=\\\{\},frame=single,framerule=0.3mm,rulecolor=\color{cellframecolor}]
{\color{outcolor}Out[{\color{outcolor}10}]:} (5,)
\end{Verbatim}
            
    \begin{center}\rule{0.5\linewidth}{\linethickness}\end{center}

    Ici encore, je peux ajouter les deux termes~:

    \begin{Verbatim}[commandchars=\\\{\},frame=single,framerule=0.3mm,rulecolor=\color{cellframecolor}]
{\color{incolor}In [{\color{incolor}11}]:} \PY{c+c1}{\PYZsh{} je peux ici encore}
         \PY{c+c1}{\PYZsh{} ajouter les tableaux}
         \PY{n}{c} \PY{o}{=} \PY{n}{a} \PY{o}{+} \PY{n}{b}
         \PY{n+nb}{print}\PY{p}{(}\PY{n}{c}\PY{p}{)}
\end{Verbatim}


    \begin{Verbatim}[commandchars=\\\{\},frame=single,framerule=0.3mm,rulecolor=\color{cellframecolor}]
[[101 102 103 104 105]
 [101 102 103 104 105]
 [101 102 103 104 105]]
\end{Verbatim}

    \begin{Verbatim}[commandchars=\\\{\},frame=single,framerule=0.3mm,rulecolor=\color{cellframecolor}]
{\color{incolor}In [{\color{incolor}12}]:} \PY{c+c1}{\PYZsh{} et c\PYZsq{}est comme si j\PYZsq{}avais}
         \PY{c+c1}{\PYZsh{} ajouté à a ce terme\PYZhy{}ci}
         \PY{n+nb}{print}\PY{p}{(}\PY{n}{c} \PY{o}{\PYZhy{}} \PY{n}{a}\PY{p}{)}
\end{Verbatim}


    \begin{Verbatim}[commandchars=\\\{\},frame=single,framerule=0.3mm,rulecolor=\color{cellframecolor}]
[[1 2 3 4 5]
 [1 2 3 4 5]
 [1 2 3 4 5]]
\end{Verbatim}

    Avec le même point de vue que tout à l'heure, on peut se dire qu'on a
d'abord transformé (broadcasté) le tableau \texttt{b}~:

    depuis la dimension \texttt{(5,)}

    vers la dimension \texttt{(3,\ 5)}

    \begin{Verbatim}[commandchars=\\\{\},frame=single,framerule=0.3mm,rulecolor=\color{cellframecolor}]
{\color{incolor}In [{\color{incolor}13}]:} \PY{c+c1}{\PYZsh{} départ}
         \PY{n+nb}{print}\PY{p}{(}\PY{n}{b}\PY{p}{)}
\end{Verbatim}


    \begin{Verbatim}[commandchars=\\\{\},frame=single,framerule=0.3mm,rulecolor=\color{cellframecolor}]
[1 2 3 4 5]
\end{Verbatim}

    \begin{Verbatim}[commandchars=\\\{\},frame=single,framerule=0.3mm,rulecolor=\color{cellframecolor}]
{\color{incolor}In [{\color{incolor}14}]:} \PY{c+c1}{\PYZsh{} arrivée}
         \PY{n+nb}{print}\PY{p}{(}\PY{n}{c} \PY{o}{\PYZhy{}} \PY{n}{a}\PY{p}{)}
\end{Verbatim}


    \begin{Verbatim}[commandchars=\\\{\},frame=single,framerule=0.3mm,rulecolor=\color{cellframecolor}]
[[1 2 3 4 5]
 [1 2 3 4 5]
 [1 2 3 4 5]]
\end{Verbatim}

    Vous commencez à mieux voir comment ça fonctionne~; s'il existe une
direction dans laquelle on peut ``tirer'' les données pour faire
coincider les formes, on peut faire du broadcasting. Et ça marche dans
toutes les directions, comme on va le voir tout de suite.

    \hypertarget{broadcasting-3-5-et-3-1}{%
\subsubsection{\texorpdfstring{Broadcasting \texttt{(3,\ 5)} et
\texttt{(3,\ 1)}}{Broadcasting (3, 5) et (3, 1)}}\label{broadcasting-3-5-et-3-1}}

    Au lieu d'ajouter à \texttt{a} une ligne, on peut lui ajouter une
colonne, pourvu qu'elle ait la même taille que les colonnes de
\texttt{a}~:

    \begin{Verbatim}[commandchars=\\\{\},frame=single,framerule=0.3mm,rulecolor=\color{cellframecolor}]
{\color{incolor}In [{\color{incolor}15}]:} \PY{n+nb}{print}\PY{p}{(}\PY{n}{a}\PY{p}{)}
\end{Verbatim}


    \begin{Verbatim}[commandchars=\\\{\},frame=single,framerule=0.3mm,rulecolor=\color{cellframecolor}]
[[100 100 100 100 100]
 [100 100 100 100 100]
 [100 100 100 100 100]]
\end{Verbatim}

    \begin{Verbatim}[commandchars=\\\{\},frame=single,framerule=0.3mm,rulecolor=\color{cellframecolor}]
{\color{incolor}In [{\color{incolor}16}]:} \PY{n}{b} \PY{o}{=} \PY{n}{np}\PY{o}{.}\PY{n}{arange}\PY{p}{(}\PY{l+m+mi}{1}\PY{p}{,} \PY{l+m+mi}{4}\PY{p}{)}\PY{o}{.}\PY{n}{reshape}\PY{p}{(}\PY{l+m+mi}{3}\PY{p}{,} \PY{l+m+mi}{1}\PY{p}{)}
         \PY{n+nb}{print}\PY{p}{(}\PY{n}{b}\PY{p}{)}
\end{Verbatim}


    \begin{Verbatim}[commandchars=\\\{\},frame=single,framerule=0.3mm,rulecolor=\color{cellframecolor}]
[[1]
 [2]
 [3]]
\end{Verbatim}

    \begin{center}\rule{0.5\linewidth}{\linethickness}\end{center}

    Voyons comment se passe le broadcasting dans ce cas-là~:

    \begin{Verbatim}[commandchars=\\\{\},frame=single,framerule=0.3mm,rulecolor=\color{cellframecolor}]
{\color{incolor}In [{\color{incolor}17}]:} \PY{n}{c} \PY{o}{=} \PY{n}{a} \PY{o}{+} \PY{n}{b}
         \PY{n+nb}{print}\PY{p}{(}\PY{n}{c}\PY{p}{)}
\end{Verbatim}


    \begin{Verbatim}[commandchars=\\\{\},frame=single,framerule=0.3mm,rulecolor=\color{cellframecolor}]
[[101 101 101 101 101]
 [102 102 102 102 102]
 [103 103 103 103 103]]
\end{Verbatim}

    \begin{Verbatim}[commandchars=\\\{\},frame=single,framerule=0.3mm,rulecolor=\color{cellframecolor}]
{\color{incolor}In [{\color{incolor}18}]:} \PY{n+nb}{print}\PY{p}{(}\PY{n}{c} \PY{o}{\PYZhy{}} \PY{n}{a}\PY{p}{)}
\end{Verbatim}


    \begin{Verbatim}[commandchars=\\\{\},frame=single,framerule=0.3mm,rulecolor=\color{cellframecolor}]
[[1 1 1 1 1]
 [2 2 2 2 2]
 [3 3 3 3 3]]
\end{Verbatim}

    Vous voyez que tout se passe exactement de la même façon que lorsqu'on
avait ajouté une simple ligne, on a cette fois ``tiré'' la colonne dans
la direction des lignes, pour passer~:

    depuis la dimension \texttt{(3,\ 1)}

    vers la dimension \texttt{(3,\ 5)}

    \begin{Verbatim}[commandchars=\\\{\},frame=single,framerule=0.3mm,rulecolor=\color{cellframecolor}]
{\color{incolor}In [{\color{incolor}19}]:} \PY{c+c1}{\PYZsh{} départ}
         \PY{n+nb}{print}\PY{p}{(}\PY{n}{b}\PY{p}{)}
\end{Verbatim}


    \begin{Verbatim}[commandchars=\\\{\},frame=single,framerule=0.3mm,rulecolor=\color{cellframecolor}]
[[1]
 [2]
 [3]]
\end{Verbatim}

    \begin{Verbatim}[commandchars=\\\{\},frame=single,framerule=0.3mm,rulecolor=\color{cellframecolor}]
{\color{incolor}In [{\color{incolor}20}]:} \PY{c+c1}{\PYZsh{} arrivée}
         \PY{n+nb}{print}\PY{p}{(}\PY{n}{c} \PY{o}{\PYZhy{}} \PY{n}{a}\PY{p}{)}
\end{Verbatim}


    \begin{Verbatim}[commandchars=\\\{\},frame=single,framerule=0.3mm,rulecolor=\color{cellframecolor}]
[[1 1 1 1 1]
 [2 2 2 2 2]
 [3 3 3 3 3]]
\end{Verbatim}

    \hypertarget{broadcasting-3-1-et-1-5}{%
\subsubsection{\texorpdfstring{Broadcasting \texttt{(3,\ 1)} et
\texttt{(1,\ 5)}}{Broadcasting (3, 1) et (1, 5)}}\label{broadcasting-3-1-et-1-5}}

    Nous avons maintenant tous les éléments en main pour comprendre un
exemple plus intéressant, où les deux tableaux ont des formes pas
vraiment compatibles à première vue~:

    \begin{Verbatim}[commandchars=\\\{\},frame=single,framerule=0.3mm,rulecolor=\color{cellframecolor}]
{\color{incolor}In [{\color{incolor}21}]:} \PY{n}{col} \PY{o}{=} \PY{n}{np}\PY{o}{.}\PY{n}{arange}\PY{p}{(}\PY{l+m+mi}{1}\PY{p}{,} \PY{l+m+mi}{4}\PY{p}{)}\PY{o}{.}\PY{n}{reshape}\PY{p}{(}\PY{p}{(}\PY{l+m+mi}{3}\PY{p}{,} \PY{l+m+mi}{1}\PY{p}{)}\PY{p}{)}
         \PY{n+nb}{print}\PY{p}{(}\PY{n}{col}\PY{p}{)}
\end{Verbatim}


    \begin{Verbatim}[commandchars=\\\{\},frame=single,framerule=0.3mm,rulecolor=\color{cellframecolor}]
[[1]
 [2]
 [3]]
\end{Verbatim}

    \begin{Verbatim}[commandchars=\\\{\},frame=single,framerule=0.3mm,rulecolor=\color{cellframecolor}]
{\color{incolor}In [{\color{incolor}22}]:} \PY{n}{line} \PY{o}{=} \PY{l+m+mi}{100} \PY{o}{*} \PY{n}{np}\PY{o}{.}\PY{n}{arange}\PY{p}{(}\PY{l+m+mi}{1}\PY{p}{,} \PY{l+m+mi}{6}\PY{p}{)}
         \PY{n+nb}{print}\PY{p}{(}\PY{n}{line}\PY{p}{)}
\end{Verbatim}


    \begin{Verbatim}[commandchars=\\\{\},frame=single,framerule=0.3mm,rulecolor=\color{cellframecolor}]
[100 200 300 400 500]
\end{Verbatim}

    \begin{center}\rule{0.5\linewidth}{\linethickness}\end{center}

    Grâce au broadcasting, on peut additionner ces deux tableaux pour
obtenir ceci~:

    \begin{Verbatim}[commandchars=\\\{\},frame=single,framerule=0.3mm,rulecolor=\color{cellframecolor}]
{\color{incolor}In [{\color{incolor}23}]:} \PY{n}{m} \PY{o}{=} \PY{n}{col} \PY{o}{+} \PY{n}{line}
         \PY{n+nb}{print}\PY{p}{(}\PY{n}{m}\PY{p}{)}
\end{Verbatim}


    \begin{Verbatim}[commandchars=\\\{\},frame=single,framerule=0.3mm,rulecolor=\color{cellframecolor}]
[[101 201 301 401 501]
 [102 202 302 402 502]
 [103 203 303 403 503]]
\end{Verbatim}

    Remarquez qu'ici les \textbf{deux} entrées ont été étirées pour
atteindre une dimension commune.

    Et donc pour illustrer le broadcasting dans ce cas, tout se passe comme
si on avait~:

    transformé la colonne \texttt{(3,\ 1)}

    en tableau \texttt{(3,\ 5)}

    \begin{Verbatim}[commandchars=\\\{\},frame=single,framerule=0.3mm,rulecolor=\color{cellframecolor}]
{\color{incolor}In [{\color{incolor}24}]:} \PY{n+nb}{print}\PY{p}{(}\PY{n}{col}\PY{p}{)}
\end{Verbatim}


    \begin{Verbatim}[commandchars=\\\{\},frame=single,framerule=0.3mm,rulecolor=\color{cellframecolor}]
[[1]
 [2]
 [3]]
\end{Verbatim}

    \begin{Verbatim}[commandchars=\\\{\},frame=single,framerule=0.3mm,rulecolor=\color{cellframecolor}]
{\color{incolor}In [{\color{incolor}25}]:} \PY{n+nb}{print}\PY{p}{(}\PY{n}{col} \PY{o}{+} \PY{n}{np}\PY{o}{.}\PY{n}{zeros}\PY{p}{(}\PY{l+m+mi}{5}\PY{p}{,} \PY{n}{dtype}\PY{o}{=}\PY{n}{np}\PY{o}{.}\PY{n}{int}\PY{p}{)}\PY{p}{)}
\end{Verbatim}


    \begin{Verbatim}[commandchars=\\\{\},frame=single,framerule=0.3mm,rulecolor=\color{cellframecolor}]
[[1 1 1 1 1]
 [2 2 2 2 2]
 [3 3 3 3 3]]
\end{Verbatim}

    et transformé la ligne \texttt{(1,\ 5)}

    en tableau \texttt{(3,\ 5)}

    \begin{Verbatim}[commandchars=\\\{\},frame=single,framerule=0.3mm,rulecolor=\color{cellframecolor}]
{\color{incolor}In [{\color{incolor}26}]:} \PY{n+nb}{print}\PY{p}{(}\PY{n}{line}\PY{p}{)}
\end{Verbatim}


    \begin{Verbatim}[commandchars=\\\{\},frame=single,framerule=0.3mm,rulecolor=\color{cellframecolor}]
[100 200 300 400 500]
\end{Verbatim}

    \begin{Verbatim}[commandchars=\\\{\},frame=single,framerule=0.3mm,rulecolor=\color{cellframecolor}]
{\color{incolor}In [{\color{incolor}27}]:} \PY{n+nb}{print}\PY{p}{(}\PY{n}{line} \PY{o}{+} \PY{n}{np}\PY{o}{.}\PY{n}{zeros}\PY{p}{(}\PY{l+m+mi}{3}\PY{p}{,} \PY{n}{dtype}\PY{o}{=}\PY{n}{np}\PY{o}{.}\PY{n}{int}\PY{p}{)}\PY{o}{.}\PY{n}{reshape}\PY{p}{(}\PY{p}{(}\PY{l+m+mi}{3}\PY{p}{,} \PY{l+m+mi}{1}\PY{p}{)}\PY{p}{)}\PY{p}{)}
\end{Verbatim}


    \begin{Verbatim}[commandchars=\\\{\},frame=single,framerule=0.3mm,rulecolor=\color{cellframecolor}]
[[100 200 300 400 500]
 [100 200 300 400 500]
 [100 200 300 400 500]]
\end{Verbatim}

    avant d'additionner terme à terme ces deux tableaux 3 x 5.

    \hypertarget{en-dimensions-supuxe9rieures}{%
\subsection{En dimensions
supérieures}\label{en-dimensions-supuxe9rieures}}

    Pour savoir si deux tableaux peuvent être compatibles via
\emph{broadcasting}, il faut comparer leurs formes. Je commence par vous
donner des exemples. Ici encore quand on mentionne l'addition, cela vaut
pour n'importe quel opérateur binaire.

    \hypertarget{exemples-de-dimensions-compatibles}{%
\subsubsection{Exemples de dimensions
compatibles}\label{exemples-de-dimensions-compatibles}}

    \begin{Shaded}
\begin{Highlighting}[frame=lines,framerule=0.6mm,rulecolor=\color{asisframecolor}]
\NormalTok{A   }\DecValTok{15}\NormalTok{ x }\DecValTok{3}\NormalTok{ x }\DecValTok{5}
\NormalTok{B   }\DecValTok{15}\NormalTok{ x }\DecValTok{1}\NormalTok{ x }\DecValTok{5}
\NormalTok{A}\OperatorTok{+}\NormalTok{B }\DecValTok{15}\NormalTok{ x }\DecValTok{3}\NormalTok{ x }\DecValTok{5}
\end{Highlighting}
\end{Shaded}

    Cas de l'ajout d'un scalaire~:

\begin{Shaded}
\begin{Highlighting}[frame=lines,framerule=0.6mm,rulecolor=\color{asisframecolor}]
\NormalTok{A   }\DecValTok{15}\NormalTok{ x }\DecValTok{3}\NormalTok{ x }\DecValTok{5}
\NormalTok{B            }\DecValTok{1}
\NormalTok{A}\OperatorTok{+}\NormalTok{B }\DecValTok{15}\NormalTok{ x }\DecValTok{3}\NormalTok{ x }\DecValTok{5}
\end{Highlighting}
\end{Shaded}

    \begin{Shaded}
\begin{Highlighting}[frame=lines,framerule=0.6mm,rulecolor=\color{asisframecolor}]
\NormalTok{A   }\DecValTok{15}\NormalTok{ x }\DecValTok{3}\NormalTok{ x }\DecValTok{5}
\NormalTok{B        }\DecValTok{3}\NormalTok{ x }\DecValTok{5}
\NormalTok{A}\OperatorTok{+}\NormalTok{B }\DecValTok{15}\NormalTok{ x }\DecValTok{3}\NormalTok{ x }\DecValTok{5}
\end{Highlighting}
\end{Shaded}

    \begin{Shaded}
\begin{Highlighting}[frame=lines,framerule=0.6mm,rulecolor=\color{asisframecolor}]
\NormalTok{A   }\DecValTok{15}\NormalTok{ x }\DecValTok{3}\NormalTok{ x }\DecValTok{5}
\NormalTok{B        }\DecValTok{3}\NormalTok{ x }\DecValTok{1}
\NormalTok{A}\OperatorTok{+}\NormalTok{B }\DecValTok{15}\NormalTok{ x }\DecValTok{3}\NormalTok{ x }\DecValTok{5}
\end{Highlighting}
\end{Shaded}

    \hypertarget{exemples-de-dimensions-non-compatibles}{%
\subsubsection{\texorpdfstring{Exemples de dimensions \textbf{non
compatibles}}{Exemples de dimensions non compatibles}}\label{exemples-de-dimensions-non-compatibles}}

    Deux lignes de longueurs différentes~:

\begin{Shaded}
\begin{Highlighting}[frame=lines,framerule=0.6mm,rulecolor=\color{asisframecolor}]
\NormalTok{A  }\DecValTok{3}
\NormalTok{B  }\DecValTok{4}
\end{Highlighting}
\end{Shaded}

    Un cas plus douteux~:

\begin{Shaded}
\begin{Highlighting}[frame=lines,framerule=0.6mm,rulecolor=\color{asisframecolor}]
\NormalTok{A      }\DecValTok{2}\NormalTok{ x }\DecValTok{1}
\NormalTok{B  }\DecValTok{8}\NormalTok{ x }\DecValTok{4}\NormalTok{ x }\DecValTok{3}
\end{Highlighting}
\end{Shaded}

    Comme vous le voyez sur tous ces exemples~:

\begin{itemize}
\item
  on peut ajouter A et B lorsqu'il existe une dimension C qui ``étire''
  à la fois celle de A et celle de B~;
\item
  on le voit sur le dernier exemple, mais on ne peut broadcaster que de
  \textbf{1} vers \(n\)~; lorsque \(p>1\) divise \(n\), on ne
  \textbf{peut pas} broadcaster de \(p\) vers \(n\), comme on pourrait
  peut-être l'imaginer.
\end{itemize}

    Comme c'est un cours de Python, plutôt que de formaliser ça sous une
forme mathématique - je vous le laisse en exercice - je vais vous
proposer plutôt une fonction Python qui détermine si deux tuples sont
des \texttt{shape} compatibles de ce point de vue.

    \begin{Verbatim}[commandchars=\\\{\},frame=single,framerule=0.3mm,rulecolor=\color{cellframecolor}]
{\color{incolor}In [{\color{incolor}28}]:} \PY{c+c1}{\PYZsh{} le module broadcasting n\PYZsq{}est pas standard}
         \PY{c+c1}{\PYZsh{} c\PYZsq{}est moi qui l\PYZsq{}ai écrit pour illustrer le cours}
         \PY{k+kn}{from} \PY{n+nn}{broadcasting} \PY{k}{import} \PY{n}{compatible}\PY{p}{,} \PY{n}{compatible2}
\end{Verbatim}


    \begin{Verbatim}[commandchars=\\\{\},frame=single,framerule=0.3mm,rulecolor=\color{cellframecolor}]
{\color{incolor}In [{\color{incolor}29}]:} \PY{c+c1}{\PYZsh{} on peut dupliquer selon un axe}
         \PY{n}{compatible}\PY{p}{(}\PY{p}{(}\PY{l+m+mi}{15}\PY{p}{,} \PY{l+m+mi}{3}\PY{p}{,} \PY{l+m+mi}{5}\PY{p}{)}\PY{p}{,} \PY{p}{(}\PY{l+m+mi}{15}\PY{p}{,} \PY{l+m+mi}{1}\PY{p}{,} \PY{l+m+mi}{5}\PY{p}{)}\PY{p}{)}
\end{Verbatim}


\begin{Verbatim}[commandchars=\\\{\},frame=single,framerule=0.3mm,rulecolor=\color{cellframecolor}]
{\color{outcolor}Out[{\color{outcolor}29}]:} (15, 3, 5)
\end{Verbatim}
            
    \begin{Verbatim}[commandchars=\\\{\},frame=single,framerule=0.3mm,rulecolor=\color{cellframecolor}]
{\color{incolor}In [{\color{incolor}30}]:} \PY{c+c1}{\PYZsh{} ou selon deux axes}
         \PY{n}{compatible}\PY{p}{(}\PY{p}{(}\PY{l+m+mi}{15}\PY{p}{,} \PY{l+m+mi}{3}\PY{p}{,} \PY{l+m+mi}{5}\PY{p}{)}\PY{p}{,} \PY{p}{(}\PY{l+m+mi}{5}\PY{p}{,}\PY{p}{)}\PY{p}{)}
\end{Verbatim}


\begin{Verbatim}[commandchars=\\\{\},frame=single,framerule=0.3mm,rulecolor=\color{cellframecolor}]
{\color{outcolor}Out[{\color{outcolor}30}]:} (15, 3, 5)
\end{Verbatim}
            
    \begin{Verbatim}[commandchars=\\\{\},frame=single,framerule=0.3mm,rulecolor=\color{cellframecolor}]
{\color{incolor}In [{\color{incolor}31}]:} \PY{c+c1}{\PYZsh{} c\PYZsq{}est bien clair que non}
         \PY{n}{compatible}\PY{p}{(}\PY{p}{(}\PY{l+m+mi}{2}\PY{p}{,}\PY{p}{)}\PY{p}{,} \PY{p}{(}\PY{l+m+mi}{3}\PY{p}{,}\PY{p}{)}\PY{p}{)}
\end{Verbatim}


\begin{Verbatim}[commandchars=\\\{\},frame=single,framerule=0.3mm,rulecolor=\color{cellframecolor}]
{\color{outcolor}Out[{\color{outcolor}31}]:} False
\end{Verbatim}
            
    \begin{Verbatim}[commandchars=\\\{\},frame=single,framerule=0.3mm,rulecolor=\color{cellframecolor}]
{\color{incolor}In [{\color{incolor}32}]:} \PY{c+c1}{\PYZsh{} on ne peut pas passer de 2 à 4}
         \PY{n}{compatible}\PY{p}{(}\PY{p}{(}\PY{l+m+mi}{1}\PY{p}{,} \PY{l+m+mi}{2}\PY{p}{)}\PY{p}{,} \PY{p}{(}\PY{l+m+mi}{2}\PY{p}{,} \PY{l+m+mi}{4}\PY{p}{)}\PY{p}{)}
\end{Verbatim}


\begin{Verbatim}[commandchars=\\\{\},frame=single,framerule=0.3mm,rulecolor=\color{cellframecolor}]
{\color{outcolor}Out[{\color{outcolor}32}]:} False
\end{Verbatim}
            

    % Add a bibliography block to the postdoc
    
    
    
