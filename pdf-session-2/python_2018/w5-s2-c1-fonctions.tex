    
    
    
    

    

    \hypertarget{programmation-fonctionnelle}{%
\section{Programmation
fonctionnelle}\label{programmation-fonctionnelle}}

    \hypertarget{compluxe9ment---niveau-basique}{%
\subsection{Complément - niveau
basique}\label{compluxe9ment---niveau-basique}}

    \hypertarget{pour-ruxe9sumer}{%
\subsubsection{Pour résumer}\label{pour-ruxe9sumer}}

    La notion de programmation fonctionnelle consiste essentiellement à
pouvoir manipuler les fonctions comme des objets à part entière, et à
les passer en argument à d'autres fonctions, comme cela est illustré
dans la vidéo.

    On peut créer une fonction par l'intermédiaire de~:

\begin{itemize}
\tightlist
\item
  l'\emph{expression} \texttt{lambda:}, on obtient alors une fonction
  \emph{anonyme}~;
\item
  l'\emph{instruction} \texttt{def} et dans ce cas on peut accéder à
  l'objet fonction par son nom.
\end{itemize}

Pour des raisons de syntaxe surtout, on a davantage de puissance avec
\texttt{def}.

    On peut calculer la liste des résultats d'une fonction sur une liste
(plus généralement un itérable) d'entrées par~:

\begin{itemize}
\tightlist
\item
  \texttt{map}, éventuellement combiné à \texttt{filter}~;
\item
  une compréhension de liste, éventuellement assortie d'un \texttt{if}.
\end{itemize}

Nous allons revoir les compréhensions dans la prochaine vidéo.

    \hypertarget{compluxe9ment---niveau-intermuxe9diaire}{%
\subsection{Complément - niveau
intermédiaire}\label{compluxe9ment---niveau-intermuxe9diaire}}

    Pour les curieux qui ont entendu le terme de \emph{map - reduce} , voici
la logique derrière l'opération \texttt{reduce}, qui est également
disponible en Python au travers du module \texttt{functools}.

    \hypertarget{reduce}{%
\subsubsection{\texorpdfstring{\texttt{reduce}}{reduce}}\label{reduce}}

    La fonction \texttt{reduce} permet d'appliquer une opération associative
à une liste d'entrées. Pour faire simple, étant donné un opérateur
binaire \(\otimes\) on veut pouvoir calculer

\(x_1 \otimes x_2 ... \otimes x_n\)

De manière un peu moins abstraite, on suppose qu'on dispose d'une
\textbf{fonction binaire} \texttt{f} qui implémente l'opérateur
\(\otimes\), et alors

\texttt{reduce}
\(( f, [x_1, .. x_n] ) = f ( ... f(f(x_1,x_2), x_3), .. , x_n)\)

    En fait \texttt{reduce} accepte un troisième argument - qu'il faut
comprendre comme l'élément neutre de l'opérateur/fonction en question -
et qui est retourné lorsque la liste en entrée est vide.

    Par exemple voici - encore - une autre implémentation possible de la
fonction \texttt{factoriel}.

On utilise ici \href{https://docs.python.org/3/library/operator.html}{le
module \texttt{operator}}, qui fournit sous forme de fonctions la
plupart des opérateurs du langage, et notamment, dans notre cas,
\texttt{operator.mul}~; cette fonction retourne tout simplement le
produit de ses deux arguments.

    \begin{Verbatim}[commandchars=\\\{\}]
{\color{incolor}In [{\color{incolor}1}]:} \PY{c+c1}{\PYZsh{} la fonction reduce dans Python 3 n\PYZsq{}est plus une built\PYZhy{}in comme en Python 2}
        \PY{c+c1}{\PYZsh{} elle fait partie du module functools}
        \PY{k+kn}{from} \PY{n+nn}{functools} \PY{k}{import} \PY{n}{reduce}
        
        \PY{c+c1}{\PYZsh{} la multiplication, mais sous forme de fonction et non d\PYZsq{}opérateur}
        \PY{k+kn}{from} \PY{n+nn}{operator} \PY{k}{import} \PY{n}{mul}
        
        \PY{k}{def} \PY{n+nf}{factoriel}\PY{p}{(}\PY{n}{n}\PY{p}{)}\PY{p}{:}
            \PY{k}{return} \PY{n}{reduce}\PY{p}{(}\PY{n}{mul}\PY{p}{,} \PY{n+nb}{range}\PY{p}{(}\PY{l+m+mi}{1}\PY{p}{,} \PY{n}{n}\PY{o}{+}\PY{l+m+mi}{1}\PY{p}{)}\PY{p}{,} \PY{l+m+mi}{1}\PY{p}{)}
        
        \PY{c+c1}{\PYZsh{} ceci fonctionne aussi pour factoriel (0)}
        \PY{k}{for} \PY{n}{i} \PY{o+ow}{in} \PY{n+nb}{range}\PY{p}{(}\PY{l+m+mi}{5}\PY{p}{)}\PY{p}{:}
            \PY{n+nb}{print}\PY{p}{(}\PY{n}{f}\PY{l+s+s2}{\PYZdq{}}\PY{l+s+si}{\PYZob{}i\PYZcb{}}\PY{l+s+s2}{ \PYZhy{}\PYZgt{} }\PY{l+s+s2}{\PYZob{}}\PY{l+s+s2}{factoriel(i)\PYZcb{}}\PY{l+s+s2}{\PYZdq{}}\PY{p}{)}
\end{Verbatim}


    \begin{Verbatim}[commandchars=\\\{\}]
0 -> 1
1 -> 1
2 -> 2
3 -> 6
4 -> 24

    \end{Verbatim}

    \hypertarget{cas-fruxe9quents-de-reduce}{%
\subparagraph{\texorpdfstring{Cas fréquents de
\texttt{reduce}}{Cas fréquents de reduce}}\label{cas-fruxe9quents-de-reduce}}

    Par commodité, Python fournit des fonctions built-in qui correspondent
en fait à des \texttt{reduce} fréquents, comme la somme, et les
opérations \texttt{min} et \texttt{max}~:

    \begin{Verbatim}[commandchars=\\\{\}]
{\color{incolor}In [{\color{incolor}2}]:} \PY{n}{entrees} \PY{o}{=} \PY{p}{[}\PY{l+m+mi}{8}\PY{p}{,} \PY{l+m+mi}{5}\PY{p}{,} \PY{l+m+mi}{12}\PY{p}{,} \PY{l+m+mi}{4}\PY{p}{,} \PY{l+m+mi}{45}\PY{p}{,} \PY{l+m+mi}{7}\PY{p}{]}
        
        \PY{n+nb}{print}\PY{p}{(}\PY{l+s+s1}{\PYZsq{}}\PY{l+s+s1}{sum}\PY{l+s+s1}{\PYZsq{}}\PY{p}{,} \PY{n+nb}{sum}\PY{p}{(}\PY{n}{entrees}\PY{p}{)}\PY{p}{)}
        \PY{n+nb}{print}\PY{p}{(}\PY{l+s+s1}{\PYZsq{}}\PY{l+s+s1}{min}\PY{l+s+s1}{\PYZsq{}}\PY{p}{,} \PY{n+nb}{min}\PY{p}{(}\PY{n}{entrees}\PY{p}{)}\PY{p}{)}
        \PY{n+nb}{print}\PY{p}{(}\PY{l+s+s1}{\PYZsq{}}\PY{l+s+s1}{max}\PY{l+s+s1}{\PYZsq{}}\PY{p}{,} \PY{n+nb}{max}\PY{p}{(}\PY{n}{entrees}\PY{p}{)}\PY{p}{)}
\end{Verbatim}


    \begin{Verbatim}[commandchars=\\\{\}]
sum 81
min 4
max 45

    \end{Verbatim}


    % Add a bibliography block to the postdoc
    
    
    
