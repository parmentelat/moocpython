    
    
    
    

    

    \hypertarget{compluxe9ment---niveau-avancuxe9}{%
\subsection{Complément - niveau
avancé}\label{compluxe9ment---niveau-avancuxe9}}

    \hypertarget{impluxe9menter-un-ituxe9rateur-de-permutations}{%
\subsubsection{Implémenter un itérateur de
permutations}\label{impluxe9menter-un-ituxe9rateur-de-permutations}}

    Dans ce complément nous allons nous amuser à implémenter une
fonctionnalité qui est déjà disponible dans le module
\texttt{itertools}.

    \hypertarget{cest-quoi-duxe9juxe0-les-permutations}{%
\subparagraph{C'est quoi déjà les
permutations~?}\label{cest-quoi-duxe9juxe0-les-permutations}}

    En guise de rappel, l'ensemble des permutations d'un ensemble fini
correspond à toutes les façons d'ordonner ses éléments~; si l'ensemble
est de cardinal \(n\), il possède \(n!\) permutations~: on a \(n\)
façons de choisir le premier élément, \(n-1\) façons de choisir le
second, etc.

    Un itérateur sur les permutations est disponible au travers du module
standard \texttt{itertools}. Cependant il nous a semblé intéressant de
vous montrer comment nous pourrions écrire nous-mêmes cette
fonctionnalité, de manière relativement simple.

    Pour illustrer le concept, voici à quoi ressemblent les 6 permutations
d'un ensemble à trois éléments~:

    \begin{Verbatim}[commandchars=\\\{\},frame=single,framerule=0.3mm,rulecolor=\color{cellframecolor}]
{\color{incolor}In [{\color{incolor}1}]:} \PY{k+kn}{from} \PY{n+nn}{itertools} \PY{k}{import} \PY{n}{permutations}
\end{Verbatim}


    \begin{Verbatim}[commandchars=\\\{\},frame=single,framerule=0.3mm,rulecolor=\color{cellframecolor}]
{\color{incolor}In [{\color{incolor}2}]:} \PY{n+nb}{set} \PY{o}{=} \PY{p}{\PYZob{}}\PY{l+m+mi}{1}\PY{p}{,} \PY{l+m+mi}{2}\PY{p}{,} \PY{l+m+mi}{3}\PY{p}{\PYZcb{}}
        
        \PY{k}{for} \PY{n}{p} \PY{o+ow}{in} \PY{n}{permutations}\PY{p}{(}\PY{n+nb}{set}\PY{p}{)}\PY{p}{:}
            \PY{n+nb}{print}\PY{p}{(}\PY{n}{p}\PY{p}{)}
\end{Verbatim}


    \begin{Verbatim}[commandchars=\\\{\},frame=single,framerule=0.3mm,rulecolor=\color{cellframecolor}]
(1, 2, 3)
(1, 3, 2)
(2, 1, 3)
(2, 3, 1)
(3, 1, 2)
(3, 2, 1)
\end{Verbatim}

    \hypertarget{une-impluxe9mentation}{%
\subparagraph{Une implémentation}\label{une-impluxe9mentation}}

    Voici une implémentation possible pour un itérateur de permutations~:

    \begin{Verbatim}[commandchars=\\\{\},frame=single,framerule=0.3mm,rulecolor=\color{cellframecolor}]
{\color{incolor}In [{\color{incolor}3}]:} \PY{k}{class} \PY{n+nc}{Permutations}\PY{p}{:}
            \PY{l+s+sd}{\PYZdq{}\PYZdq{}\PYZdq{}}
        \PY{l+s+sd}{    Un itérateur qui énumère les permutations de n}
        \PY{l+s+sd}{    sous la forme d\PYZsq{}une liste d\PYZsq{}indices commençant à 0}
        \PY{l+s+sd}{    \PYZdq{}\PYZdq{}\PYZdq{}}
            \PY{k}{def} \PY{n+nf}{\PYZus{}\PYZus{}init\PYZus{}\PYZus{}}\PY{p}{(}\PY{n+nb+bp}{self}\PY{p}{,} \PY{n}{n}\PY{p}{)}\PY{p}{:}
                \PY{c+c1}{\PYZsh{} le constructeur bien sûr ne fait (presque) rien}
                \PY{n+nb+bp}{self}\PY{o}{.}\PY{n}{n} \PY{o}{=} \PY{n}{n}
                \PY{c+c1}{\PYZsh{} au fur et à mesure des itérations}
                \PY{c+c1}{\PYZsh{} le compteur va aller de 0 à n\PYZhy{}1}
                \PY{c+c1}{\PYZsh{} puis retour à 0 et comme ça en boucle sans fin}
                \PY{n+nb+bp}{self}\PY{o}{.}\PY{n}{counter} \PY{o}{=} \PY{l+m+mi}{0}
                \PY{c+c1}{\PYZsh{} on se contente d\PYZsq{}allouer un iterateur de rang n\PYZhy{}1}
                \PY{c+c1}{\PYZsh{} si bien qu\PYZsq{}une fois qu\PYZsq{}on a fini de construire}
                \PY{c+c1}{\PYZsh{} l\PYZsq{}objet d\PYZsq{}ordre n on a n objets Permutations en tout}
                \PY{k}{if} \PY{n}{n} \PY{o}{\PYZgt{}}\PY{o}{=} \PY{l+m+mi}{2}\PY{p}{:}
                    \PY{n+nb+bp}{self}\PY{o}{.}\PY{n}{subiterator} \PY{o}{=} \PY{n}{Permutations}\PY{p}{(}\PY{n}{n}\PY{o}{\PYZhy{}}\PY{l+m+mi}{1}\PY{p}{)}
        
            \PY{c+c1}{\PYZsh{} pour satisfaire le protocole d\PYZsq{}itération}
            \PY{k}{def} \PY{n+nf}{\PYZus{}\PYZus{}iter\PYZus{}\PYZus{}}\PY{p}{(}\PY{n+nb+bp}{self}\PY{p}{)}\PY{p}{:}
                \PY{k}{return} \PY{n+nb+bp}{self}
        
            \PY{c+c1}{\PYZsh{} c\PYZsq{}est ici bien sûr que se fait tout le travail}
            \PY{k}{def} \PY{n+nf}{\PYZus{}\PYZus{}next\PYZus{}\PYZus{}}\PY{p}{(}\PY{n+nb+bp}{self}\PY{p}{)}\PY{p}{:}
                \PY{c+c1}{\PYZsh{} pour n == 1}
                \PY{c+c1}{\PYZsh{} le travail est très simple}
                \PY{k}{if} \PY{n+nb+bp}{self}\PY{o}{.}\PY{n}{n} \PY{o}{==} \PY{l+m+mi}{1}\PY{p}{:}
                    \PY{c+c1}{\PYZsh{} on doit renvoyer une fois la liste [0]}
                    \PY{c+c1}{\PYZsh{} car les indices commencent à 0}
                    \PY{k}{if} \PY{n+nb+bp}{self}\PY{o}{.}\PY{n}{counter} \PY{o}{==} \PY{l+m+mi}{0}\PY{p}{:} 
                        \PY{n+nb+bp}{self}\PY{o}{.}\PY{n}{counter} \PY{o}{+}\PY{o}{=} \PY{l+m+mi}{1}
                        \PY{k}{return} \PY{p}{[}\PY{l+m+mi}{0}\PY{p}{]}
                    \PY{c+c1}{\PYZsh{} et ensuite c\PYZsq{}est terminé}
                    \PY{k}{else}\PY{p}{:}
                        \PY{k}{raise} \PY{n+ne}{StopIteration}
        
                \PY{c+c1}{\PYZsh{} pour n \PYZgt{}= 2}
                \PY{c+c1}{\PYZsh{} lorsque counter est nul,}
                \PY{c+c1}{\PYZsh{} on traite la permutation d\PYZsq{}ordre n\PYZhy{}1 suivante}
                \PY{c+c1}{\PYZsh{} si next() lève StopIteration on n\PYZsq{}a qu\PYZsq{}à laisser passer}
                \PY{c+c1}{\PYZsh{} car en effet c\PYZsq{}est qu\PYZsq{}on a terminé}
                \PY{k}{if} \PY{n+nb+bp}{self}\PY{o}{.}\PY{n}{counter} \PY{o}{==} \PY{l+m+mi}{0}\PY{p}{:}
                    \PY{n+nb+bp}{self}\PY{o}{.}\PY{n}{subsequence} \PY{o}{=} \PY{n+nb}{next}\PY{p}{(}\PY{n+nb+bp}{self}\PY{o}{.}\PY{n}{subiterator}\PY{p}{)}
                \PY{c+c1}{\PYZsh{}}
                \PY{c+c1}{\PYZsh{} on insère alors n\PYZhy{}1 (car les indices commencent à 0)}
                \PY{c+c1}{\PYZsh{} successivement dans la sous\PYZhy{}sequence}
                \PY{c+c1}{\PYZsh{}}
                \PY{c+c1}{\PYZsh{} naivement on écrirait}
                \PY{c+c1}{\PYZsh{} result = self.subsequence[0:self.counter] \PYZbs{}}
                \PY{c+c1}{\PYZsh{}    + [self.n \PYZhy{} 1] \PYZbs{}}
                \PY{c+c1}{\PYZsh{}    + self.subsequence[self.counter:self.n\PYZhy{}1]}
                \PY{c+c1}{\PYZsh{} mais c\PYZsq{}est mettre le nombre le plus élevé en premier}
                \PY{c+c1}{\PYZsh{} et donc à itérer les permutations dans le mauvais ordre,}
                \PY{c+c1}{\PYZsh{} en commençant par la fin}
                \PY{c+c1}{\PYZsh{}}
                \PY{c+c1}{\PYZsh{} donc on fait plutôt une symétrie}
                \PY{c+c1}{\PYZsh{} pour insérer en commençant par la fin}
                \PY{n}{cutter} \PY{o}{=} \PY{n+nb+bp}{self}\PY{o}{.}\PY{n}{n}\PY{o}{\PYZhy{}}\PY{l+m+mi}{1} \PY{o}{\PYZhy{}} \PY{n+nb+bp}{self}\PY{o}{.}\PY{n}{counter}
                \PY{n}{result} \PY{o}{=} \PY{n+nb+bp}{self}\PY{o}{.}\PY{n}{subsequence}\PY{p}{[}\PY{l+m+mi}{0}\PY{p}{:}\PY{n}{cutter}\PY{p}{]} \PY{o}{+} \PY{p}{[}\PY{n+nb+bp}{self}\PY{o}{.}\PY{n}{n} \PY{o}{\PYZhy{}} \PY{l+m+mi}{1}\PY{p}{]} \PYZbs{}
                         \PY{o}{+} \PY{n+nb+bp}{self}\PY{o}{.}\PY{n}{subsequence}\PY{p}{[}\PY{n}{cutter}\PY{p}{:}\PY{n+nb+bp}{self}\PY{o}{.}\PY{n}{n}\PY{o}{\PYZhy{}}\PY{l+m+mi}{1}\PY{p}{]}
                \PY{c+c1}{\PYZsh{} }
                \PY{c+c1}{\PYZsh{} on n\PYZsq{}oublie pas de maintenir le compteur et de}
                \PY{c+c1}{\PYZsh{} le remettre à zéro tous les n tours}
                \PY{n+nb+bp}{self}\PY{o}{.}\PY{n}{counter} \PY{o}{=} \PY{p}{(}\PY{n+nb+bp}{self}\PY{o}{.}\PY{n}{counter}\PY{o}{+}\PY{l+m+mi}{1}\PY{p}{)} \PY{o}{\PYZpc{}} \PY{n+nb+bp}{self}\PY{o}{.}\PY{n}{n}
                \PY{k}{return} \PY{n}{result}
        
            \PY{c+c1}{\PYZsh{} la longeur de cet itérateur est connue}
            \PY{k}{def} \PY{n+nf}{\PYZus{}\PYZus{}len\PYZus{}\PYZus{}}\PY{p}{(}\PY{n+nb+bp}{self}\PY{p}{)}\PY{p}{:}
                \PY{k+kn}{import} \PY{n+nn}{math}
                \PY{k}{return} \PY{n}{math}\PY{o}{.}\PY{n}{factorial}\PY{p}{(}\PY{n+nb+bp}{self}\PY{o}{.}\PY{n}{n}\PY{p}{)}
\end{Verbatim}


    Ce qu'on a essayé d'expliquer dans les commentaires, c'est qu'on procède
en fin de compte par récurrence. Un objet \texttt{Permutations} de rang
\texttt{n} possède un sous-itérateur de rang \texttt{n-1} qu'on crée
dans le constructeur. Ensuite l'objet de rang \texttt{n} va faire
successivement (c'est-à-dire à chaque appel de \texttt{next()})~:

\begin{itemize}
\item
  appel \emph{0}~:

  \begin{itemize}
  \tightlist
  \item
    demander à son sous-itérateur une permutation de rang \texttt{n-1}
    (en lui envoyant \texttt{next}),
  \item
    la stocker dans l'objet de rang \texttt{n}, ce sera utilisé par les
    \emph{n} premier appels,
  \item
    et construire une liste de taille \texttt{n} en insérant
    \texttt{n-1} à la fin de la séquence de taille \texttt{n-1},
  \end{itemize}
\item
  appel \emph{1}~:

  \begin{itemize}
  \tightlist
  \item
    insérer \texttt{n-1} dans la même séquence de rang \texttt{n-1} mais
    cette fois 1 cran avant la fin,
  \end{itemize}
\item
  \ldots{}
\item
  appel \emph{n-1}~:

  \begin{itemize}
  \tightlist
  \item
    insérer \texttt{n-1} au début de la séquence de rang \texttt{n-1},
  \end{itemize}
\item
  appel \emph{n}~:

  \begin{itemize}
  \tightlist
  \item
    refaire \texttt{next()} sur le sous-itérateur pour traiter une
    nouvelle sous-séquence,
  \item
    la stocker dans l'objet de rang \texttt{n}, comme à l'appel
    \emph{0}, pour ce bloc de n appels,
  \item
    et construire la permutation en insérant \emph{n-1} à la fin, comme
    à l'appel 0,
  \end{itemize}
\item
  \ldots{}
\end{itemize}

    On voit donc le caractère cyclique d'ordre \emph{n} qui est matérialisé
par \texttt{counter}, que l'on incrémente à chaque boucle mais modulo
\emph{n} - notez d'ailleurs que pour ce genre de comportement on dispose
aussi de \texttt{itertools.cycle} comme on le verra dans une deuxième
version, mais pour l'instant j'ai préféré ne pas l'utiliser pour ne pas
tout embrouiller ;)

La terminaison se gère très simplement, car une fois que l'on a traité
toutes les séquences d'ordre \emph{n-1} eh bien on a fini, on n'a même
pas besoin de lever StopIteration explicitement, sauf bien sûr dans le
cas \emph{n=1}.

Le seul point un peu délicat, si on veut avoir les permutations dans le
``bon'' ordre, consiste à commencer à insérer \texttt{n-1} par la droite
(la fin de la sous-séquence).

    \hypertarget{discussion}{%
\subparagraph{Discussion}\label{discussion}}

    Il existe certainement des tas d'autres façons de faire bien entendu. Le
point important ici, et qui donne toute sa puissance à la notion
d'itérateur, c'est \textbf{qu'à aucun moment on ne construit} une liste
ou une séquence quelconque de \textbf{n! termes}.

C'est une erreur fréquente chez les débutants que de calculer une telle
liste dans le constructeur, mais procéder de cette façon c'est aller
exactement à l'opposé de ce pourquoi les itérateurs ont été conçus~; au
contraire, on veut éviter à tout prix le coût d'une telle construction.

On peut le voir sur un code qui n'utiliserait que les 20 premières
valeurs de l'itérateur, vous constatez que ce code est immédiat~:

    \begin{Verbatim}[commandchars=\\\{\},frame=single,framerule=0.3mm,rulecolor=\color{cellframecolor}]
{\color{incolor}In [{\color{incolor}4}]:} \PY{k}{def} \PY{n+nf}{show\PYZus{}first\PYZus{}items}\PY{p}{(}\PY{n}{iterable}\PY{p}{,} \PY{n}{nb\PYZus{}items}\PY{p}{)}\PY{p}{:}
            \PY{l+s+sd}{\PYZdq{}\PYZdq{}\PYZdq{}}
        \PY{l+s+sd}{    montre les \PYZlt{}nb\PYZus{}items\PYZgt{} premiers items de iterable}
        \PY{l+s+sd}{    \PYZdq{}\PYZdq{}\PYZdq{}}
            \PY{n+nb}{print}\PY{p}{(}\PY{n}{f}\PY{l+s+s2}{\PYZdq{}}\PY{l+s+s2}{Il y a }\PY{l+s+s2}{\PYZob{}}\PY{l+s+s2}{len(iterable)\PYZcb{} items dans l}\PY{l+s+s2}{\PYZsq{}}\PY{l+s+s2}{itérable}\PY{l+s+s2}{\PYZdq{}}\PY{p}{)}
            \PY{k}{for} \PY{n}{i}\PY{p}{,} \PY{n}{item} \PY{o+ow}{in} \PY{n+nb}{enumerate}\PY{p}{(}\PY{n}{iterable}\PY{p}{)}\PY{p}{:}
                \PY{n+nb}{print}\PY{p}{(}\PY{n}{item}\PY{p}{)}
                \PY{k}{if} \PY{n}{i} \PY{o}{\PYZgt{}}\PY{o}{=} \PY{n}{nb\PYZus{}items}\PY{p}{:}
                    \PY{n+nb}{print}\PY{p}{(}\PY{l+s+s1}{\PYZsq{}}\PY{l+s+s1}{....}\PY{l+s+s1}{\PYZsq{}}\PY{p}{)}
                    \PY{k}{break}
\end{Verbatim}


    \begin{Verbatim}[commandchars=\\\{\},frame=single,framerule=0.3mm,rulecolor=\color{cellframecolor}]
{\color{incolor}In [{\color{incolor}5}]:} \PY{n}{show\PYZus{}first\PYZus{}items}\PY{p}{(}\PY{n}{Permutations}\PY{p}{(}\PY{l+m+mi}{12}\PY{p}{)}\PY{p}{,} \PY{l+m+mi}{20}\PY{p}{)}
\end{Verbatim}


    \begin{Verbatim}[commandchars=\\\{\},frame=single,framerule=0.3mm,rulecolor=\color{cellframecolor}]
Il y a 479001600 items dans l'itérable
[0, 1, 2, 3, 4, 5, 6, 7, 8, 9, 10, 11]
[0, 1, 2, 3, 4, 5, 6, 7, 8, 9, 11, 10]
[0, 1, 2, 3, 4, 5, 6, 7, 8, 11, 9, 10]
[0, 1, 2, 3, 4, 5, 6, 7, 11, 8, 9, 10]
[0, 1, 2, 3, 4, 5, 6, 11, 7, 8, 9, 10]
[0, 1, 2, 3, 4, 5, 11, 6, 7, 8, 9, 10]
[0, 1, 2, 3, 4, 11, 5, 6, 7, 8, 9, 10]
[0, 1, 2, 3, 11, 4, 5, 6, 7, 8, 9, 10]
[0, 1, 2, 11, 3, 4, 5, 6, 7, 8, 9, 10]
[0, 1, 11, 2, 3, 4, 5, 6, 7, 8, 9, 10]
[0, 11, 1, 2, 3, 4, 5, 6, 7, 8, 9, 10]
[11, 0, 1, 2, 3, 4, 5, 6, 7, 8, 9, 10]
[0, 1, 2, 3, 4, 5, 6, 7, 8, 10, 9, 11]
[0, 1, 2, 3, 4, 5, 6, 7, 8, 10, 11, 9]
[0, 1, 2, 3, 4, 5, 6, 7, 8, 11, 10, 9]
[0, 1, 2, 3, 4, 5, 6, 7, 11, 8, 10, 9]
[0, 1, 2, 3, 4, 5, 6, 11, 7, 8, 10, 9]
[0, 1, 2, 3, 4, 5, 11, 6, 7, 8, 10, 9]
[0, 1, 2, 3, 4, 11, 5, 6, 7, 8, 10, 9]
[0, 1, 2, 3, 11, 4, 5, 6, 7, 8, 10, 9]
[0, 1, 2, 11, 3, 4, 5, 6, 7, 8, 10, 9]
{\ldots}
\end{Verbatim}

    Ce tableau vous montre par ailleurs sous un autre angle comment
fonctionne l'algorithme, si vous observez le \texttt{11} qui balaie en
diagonale les 12 premières lignes, puis les 12 suivantes, etc..

    \hypertarget{ultimes-amuxe9liorations}{%
\subparagraph{Ultimes améliorations}\label{ultimes-amuxe9liorations}}

    Dernières remarques, sur des améliorations possibles - mais tout à fait
optionnelles~:

\begin{itemize}
\tightlist
\item
  le lecteur attentif aura remarqué qu'au lieu d'un entier
  \texttt{counter} on aurait pu profitablement utiliser une instance de
  \texttt{itertools.cycle}, ce qui aurait eu l'avantage d'être plus
  clair sur le propos de ce compteur~;
\item
  aussi dans le même mouvement, au lieu de se livrer à la gymnastique
  qui calcule \texttt{cutter} à partir de \texttt{counter}, on pourrait
  dès le départ créer dans le cycle les bonnes valeurs en commençant à
  \texttt{n-1}.
\end{itemize}

    C'est ce qu'on a fait dans cette deuxième version~; après avoir enlevé
la logorrhée de commentaires ça redevient presque lisible ;)

    \begin{Verbatim}[commandchars=\\\{\},frame=single,framerule=0.3mm,rulecolor=\color{cellframecolor}]
{\color{incolor}In [{\color{incolor}6}]:} \PY{k+kn}{import} \PY{n+nn}{itertools}
        
        \PY{k}{class} \PY{n+nc}{Permutations2}\PY{p}{:}
            \PY{l+s+sd}{\PYZdq{}\PYZdq{}\PYZdq{}}
        \PY{l+s+sd}{    Un itérateur qui énumère les permutations de n}
        \PY{l+s+sd}{    sous la forme d\PYZsq{}une liste d\PYZsq{}indices commençant à 0}
        \PY{l+s+sd}{    \PYZdq{}\PYZdq{}\PYZdq{}}
            \PY{k}{def} \PY{n+nf}{\PYZus{}\PYZus{}init\PYZus{}\PYZus{}}\PY{p}{(}\PY{n+nb+bp}{self}\PY{p}{,} \PY{n}{n}\PY{p}{)}\PY{p}{:}
                \PY{n+nb+bp}{self}\PY{o}{.}\PY{n}{n} \PY{o}{=} \PY{n}{n}
                \PY{c+c1}{\PYZsh{} on commence à insérer à la fin }
                \PY{n+nb+bp}{self}\PY{o}{.}\PY{n}{cycle} \PY{o}{=} \PY{n}{itertools}\PY{o}{.}\PY{n}{cycle}\PY{p}{(}\PY{n+nb}{list}\PY{p}{(}\PY{n+nb}{range}\PY{p}{(}\PY{n}{n}\PY{p}{)}\PY{p}{)}\PY{p}{[}\PY{p}{:}\PY{p}{:}\PY{o}{\PYZhy{}}\PY{l+m+mi}{1}\PY{p}{]}\PY{p}{)}
                \PY{k}{if} \PY{n}{n} \PY{o}{\PYZgt{}}\PY{o}{=} \PY{l+m+mi}{2}\PY{p}{:}
                    \PY{n+nb+bp}{self}\PY{o}{.}\PY{n}{subiterator} \PY{o}{=} \PY{n}{Permutations2}\PY{p}{(}\PY{n}{n}\PY{o}{\PYZhy{}}\PY{l+m+mi}{1}\PY{p}{)}
                \PY{c+c1}{\PYZsh{} pour savoir quand terminer le cas n==1}
                \PY{k}{if} \PY{n}{n} \PY{o}{==} \PY{l+m+mi}{1}\PY{p}{:}
                    \PY{n+nb+bp}{self}\PY{o}{.}\PY{n}{done} \PY{o}{=} \PY{k+kc}{False}
        
            \PY{k}{def} \PY{n+nf}{\PYZus{}\PYZus{}iter\PYZus{}\PYZus{}}\PY{p}{(}\PY{n+nb+bp}{self}\PY{p}{)}\PY{p}{:}
                \PY{k}{return} \PY{n+nb+bp}{self}
        
            \PY{k}{def} \PY{n+nf}{\PYZus{}\PYZus{}next\PYZus{}\PYZus{}}\PY{p}{(}\PY{n+nb+bp}{self}\PY{p}{)}\PY{p}{:}
                \PY{n}{cutter} \PY{o}{=} \PY{n+nb}{next}\PY{p}{(}\PY{n+nb+bp}{self}\PY{o}{.}\PY{n}{cycle}\PY{p}{)}
        
                \PY{c+c1}{\PYZsh{} quand n==1 on a toujours la même valeur 0}
                \PY{k}{if} \PY{n+nb+bp}{self}\PY{o}{.}\PY{n}{n} \PY{o}{==} \PY{l+m+mi}{1}\PY{p}{:}
                    \PY{k}{if} \PY{o+ow}{not} \PY{n+nb+bp}{self}\PY{o}{.}\PY{n}{done}\PY{p}{:}
                        \PY{n+nb+bp}{self}\PY{o}{.}\PY{n}{done} \PY{o}{=} \PY{k+kc}{True}
                        \PY{k}{return} \PY{p}{[}\PY{l+m+mi}{0}\PY{p}{]}
                    \PY{k}{else}\PY{p}{:}
                        \PY{k}{raise} \PY{n+ne}{StopIteration}
        
                \PY{c+c1}{\PYZsh{} au début de chaque séquence de n appels}
                \PY{c+c1}{\PYZsh{} il faut appeler une nouvelle sous\PYZhy{}séquence}
                \PY{k}{if} \PY{n}{cutter} \PY{o}{==} \PY{n+nb+bp}{self}\PY{o}{.}\PY{n}{n}\PY{o}{\PYZhy{}}\PY{l+m+mi}{1}\PY{p}{:}
                    \PY{n+nb+bp}{self}\PY{o}{.}\PY{n}{subsequence} \PY{o}{=} \PY{n+nb}{next}\PY{p}{(}\PY{n+nb+bp}{self}\PY{o}{.}\PY{n}{subiterator}\PY{p}{)}
                \PY{c+c1}{\PYZsh{} dans laquelle on insére n\PYZhy{}1}
                \PY{k}{return} \PY{n+nb+bp}{self}\PY{o}{.}\PY{n}{subsequence}\PY{p}{[}\PY{l+m+mi}{0}\PY{p}{:}\PY{n}{cutter}\PY{p}{]} \PY{o}{+} \PY{p}{[}\PY{n+nb+bp}{self}\PY{o}{.}\PY{n}{n}\PY{o}{\PYZhy{}}\PY{l+m+mi}{1}\PY{p}{]} \PYZbs{}
                         \PY{o}{+} \PY{n+nb+bp}{self}\PY{o}{.}\PY{n}{subsequence}\PY{p}{[}\PY{n}{cutter}\PY{p}{:}\PY{n+nb+bp}{self}\PY{o}{.}\PY{n}{n}\PY{o}{\PYZhy{}}\PY{l+m+mi}{1}\PY{p}{]}
        
            \PY{c+c1}{\PYZsh{} la longeur de cet itérateur est connue}
            \PY{k}{def} \PY{n+nf}{\PYZus{}\PYZus{}len\PYZus{}\PYZus{}}\PY{p}{(}\PY{n+nb+bp}{self}\PY{p}{)}\PY{p}{:}
                \PY{k+kn}{import} \PY{n+nn}{math}
                \PY{k}{return} \PY{n}{math}\PY{o}{.}\PY{n}{factorial}\PY{p}{(}\PY{n+nb+bp}{self}\PY{o}{.}\PY{n}{n}\PY{p}{)}
\end{Verbatim}


    \begin{Verbatim}[commandchars=\\\{\},frame=single,framerule=0.3mm,rulecolor=\color{cellframecolor}]
{\color{incolor}In [{\color{incolor}7}]:} \PY{n}{show\PYZus{}first\PYZus{}items}\PY{p}{(}\PY{n}{Permutations2}\PY{p}{(}\PY{l+m+mi}{5}\PY{p}{)}\PY{p}{,} \PY{l+m+mi}{20}\PY{p}{)}
\end{Verbatim}


    \begin{Verbatim}[commandchars=\\\{\},frame=single,framerule=0.3mm,rulecolor=\color{cellframecolor}]
Il y a 120 items dans l'itérable
[0, 1, 2, 3, 4]
[0, 1, 2, 4, 3]
[0, 1, 4, 2, 3]
[0, 4, 1, 2, 3]
[4, 0, 1, 2, 3]
[0, 1, 3, 2, 4]
[0, 1, 3, 4, 2]
[0, 1, 4, 3, 2]
[0, 4, 1, 3, 2]
[4, 0, 1, 3, 2]
[0, 3, 1, 2, 4]
[0, 3, 1, 4, 2]
[0, 3, 4, 1, 2]
[0, 4, 3, 1, 2]
[4, 0, 3, 1, 2]
[3, 0, 1, 2, 4]
[3, 0, 1, 4, 2]
[3, 0, 4, 1, 2]
[3, 4, 0, 1, 2]
[4, 3, 0, 1, 2]
[0, 2, 1, 3, 4]
{\ldots}
\end{Verbatim}

    \begin{center}\rule{0.5\linewidth}{\linethickness}\end{center}

\begin{center}\rule{0.5\linewidth}{\linethickness}\end{center}

    Il me semble intéressant de montrer une autre façon, plus simple,
d'écrire un itérateur de permutations, à base cette fois de générateurs;
c'est un tout petit peu une digression par rapport au cours qui est sur
la conception d'itérateurs et d'itérables. Ça va nous permettre surtout
de réviser la notion de \texttt{yield\ from}.

    On commence par une version très rustique qui fait des impressions~:

    \begin{Verbatim}[commandchars=\\\{\},frame=single,framerule=0.3mm,rulecolor=\color{cellframecolor}]
{\color{incolor}In [{\color{incolor}8}]:} \PY{c+c1}{\PYZsh{} pour simplifier ici on suppose que l\PYZsq{}entrée est une vraie liste}
        \PY{c+c1}{\PYZsh{} que l\PYZsq{}on va ainsi pouvoir modifier par effets de bord}
        \PY{k}{def} \PY{n+nf}{gen\PYZus{}perm1}\PY{p}{(}\PY{n}{subject}\PY{p}{,} \PY{n}{k}\PY{o}{=}\PY{l+m+mi}{0}\PY{p}{)}\PY{p}{:}
            \PY{k}{if} \PY{n}{k} \PY{o}{==} \PY{n+nb}{len}\PY{p}{(}\PY{n}{subject}\PY{p}{)}\PY{p}{:}
                \PY{c+c1}{\PYZsh{} cette version hyper rustique se contente de faire une impression}
                \PY{n+nb}{print}\PY{p}{(}\PY{n}{subject}\PY{p}{)}
            \PY{k}{else}\PY{p}{:}
                \PY{k}{for} \PY{n}{i} \PY{o+ow}{in} \PY{n+nb}{range}\PY{p}{(}\PY{n}{k}\PY{p}{,} \PY{n+nb}{len}\PY{p}{(}\PY{n}{subject}\PY{p}{)}\PY{p}{)}\PY{p}{:}
                    \PY{c+c1}{\PYZsh{} on échange }
                    \PY{n}{subject}\PY{p}{[}\PY{n}{k}\PY{p}{]}\PY{p}{,} \PY{n}{subject}\PY{p}{[}\PY{n}{i}\PY{p}{]} \PY{o}{=} \PY{n}{subject}\PY{p}{[}\PY{n}{i}\PY{p}{]}\PY{p}{,} \PY{n}{subject}\PY{p}{[}\PY{n}{k}\PY{p}{]}
                    \PY{n}{gen\PYZus{}perm1}\PY{p}{(}\PY{n}{subject}\PY{p}{,} \PY{n}{k}\PY{o}{+}\PY{l+m+mi}{1}\PY{p}{)}
                    \PY{c+c1}{\PYZsh{} on remet comme c\PYZsq{}était pour le prochain échange}
                    \PY{n}{subject}\PY{p}{[}\PY{n}{k}\PY{p}{]}\PY{p}{,} \PY{n}{subject}\PY{p}{[}\PY{n}{i}\PY{p}{]} \PY{o}{=} \PY{n}{subject}\PY{p}{[}\PY{n}{i}\PY{p}{]}\PY{p}{,} \PY{n}{subject}\PY{p}{[}\PY{n}{k}\PY{p}{]}
\end{Verbatim}


    \begin{Verbatim}[commandchars=\\\{\},frame=single,framerule=0.3mm,rulecolor=\color{cellframecolor}]
{\color{incolor}In [{\color{incolor}9}]:} \PY{n}{gen\PYZus{}perm1}\PY{p}{(}\PY{p}{[}\PY{l+s+s1}{\PYZsq{}}\PY{l+s+s1}{a}\PY{l+s+s1}{\PYZsq{}}\PY{p}{,} \PY{l+s+s1}{\PYZsq{}}\PY{l+s+s1}{b}\PY{l+s+s1}{\PYZsq{}}\PY{p}{,} \PY{l+s+s1}{\PYZsq{}}\PY{l+s+s1}{c}\PY{l+s+s1}{\PYZsq{}}\PY{p}{,} \PY{l+s+s1}{\PYZsq{}}\PY{l+s+s1}{d}\PY{l+s+s1}{\PYZsq{}}\PY{p}{]}\PY{p}{)}
\end{Verbatim}


    \begin{Verbatim}[commandchars=\\\{\},frame=single,framerule=0.3mm,rulecolor=\color{cellframecolor}]
['a', 'b', 'c', 'd']
['a', 'b', 'd', 'c']
['a', 'c', 'b', 'd']
['a', 'c', 'd', 'b']
['a', 'd', 'c', 'b']
['a', 'd', 'b', 'c']
['b', 'a', 'c', 'd']
['b', 'a', 'd', 'c']
['b', 'c', 'a', 'd']
['b', 'c', 'd', 'a']
['b', 'd', 'c', 'a']
['b', 'd', 'a', 'c']
['c', 'b', 'a', 'd']
['c', 'b', 'd', 'a']
['c', 'a', 'b', 'd']
['c', 'a', 'd', 'b']
['c', 'd', 'a', 'b']
['c', 'd', 'b', 'a']
['d', 'b', 'c', 'a']
['d', 'b', 'a', 'c']
['d', 'c', 'b', 'a']
['d', 'c', 'a', 'b']
['d', 'a', 'c', 'b']
['d', 'a', 'b', 'c']
\end{Verbatim}

    Très bien, mais on ne veut pas imprimer, on veut itérer. On pourrait se
dire, il me suffit de remplacer \texttt{print} par \texttt{yield}.
Essayons cela~:

    \begin{Verbatim}[commandchars=\\\{\},frame=single,framerule=0.3mm,rulecolor=\color{cellframecolor}]
{\color{incolor}In [{\color{incolor}10}]:} \PY{c+c1}{\PYZsh{} pour simplifier ici on suppose que l\PYZsq{}entrée est une vraie liste}
         \PY{c+c1}{\PYZsh{} que l\PYZsq{}on va ainsi pouvoir modifier par effets de bord}
         \PY{k}{def} \PY{n+nf}{gen\PYZus{}perm2}\PY{p}{(}\PY{n}{subject}\PY{p}{,} \PY{n}{k}\PY{o}{=}\PY{l+m+mi}{0}\PY{p}{)}\PY{p}{:}
             \PY{k}{if} \PY{n}{k} \PY{o}{==} \PY{n+nb}{len}\PY{p}{(}\PY{n}{subject}\PY{p}{)}\PY{p}{:}
                 \PY{c+c1}{\PYZsh{} cette version hyper rustique se contente de faire une impression}
                 \PY{k}{yield} \PY{n}{subject}
             \PY{k}{else}\PY{p}{:}
                 \PY{k}{for} \PY{n}{i} \PY{o+ow}{in} \PY{n+nb}{range}\PY{p}{(}\PY{n}{k}\PY{p}{,} \PY{n+nb}{len}\PY{p}{(}\PY{n}{subject}\PY{p}{)}\PY{p}{)}\PY{p}{:}
                     \PY{c+c1}{\PYZsh{} on échange }
                     \PY{n}{subject}\PY{p}{[}\PY{n}{k}\PY{p}{]}\PY{p}{,} \PY{n}{subject}\PY{p}{[}\PY{n}{i}\PY{p}{]} \PY{o}{=} \PY{n}{subject}\PY{p}{[}\PY{n}{i}\PY{p}{]}\PY{p}{,} \PY{n}{subject}\PY{p}{[}\PY{n}{k}\PY{p}{]}
                     \PY{n}{gen\PYZus{}perm2}\PY{p}{(}\PY{n}{subject}\PY{p}{,} \PY{n}{k}\PY{o}{+}\PY{l+m+mi}{1}\PY{p}{)}
                     \PY{c+c1}{\PYZsh{} on remet comme c\PYZsq{}était pour le prochain échange}
                     \PY{n}{subject}\PY{p}{[}\PY{n}{k}\PY{p}{]}\PY{p}{,} \PY{n}{subject}\PY{p}{[}\PY{n}{i}\PY{p}{]} \PY{o}{=} \PY{n}{subject}\PY{p}{[}\PY{n}{i}\PY{p}{]}\PY{p}{,} \PY{n}{subject}\PY{p}{[}\PY{n}{k}\PY{p}{]}
\end{Verbatim}


    \begin{Verbatim}[commandchars=\\\{\},frame=single,framerule=0.3mm,rulecolor=\color{cellframecolor}]
{\color{incolor}In [{\color{incolor}11}]:} \PY{k}{for} \PY{n}{perm} \PY{o+ow}{in} \PY{n}{gen\PYZus{}perm2}\PY{p}{(}\PY{p}{[}\PY{l+s+s1}{\PYZsq{}}\PY{l+s+s1}{a}\PY{l+s+s1}{\PYZsq{}}\PY{p}{,} \PY{l+s+s1}{\PYZsq{}}\PY{l+s+s1}{b}\PY{l+s+s1}{\PYZsq{}}\PY{p}{,} \PY{l+s+s1}{\PYZsq{}}\PY{l+s+s1}{c}\PY{l+s+s1}{\PYZsq{}}\PY{p}{,} \PY{l+s+s1}{\PYZsq{}}\PY{l+s+s1}{d}\PY{l+s+s1}{\PYZsq{}}\PY{p}{]}\PY{p}{)}\PY{p}{:}
             \PY{n+nb}{print}\PY{p}{(}\PY{n}{perm}\PY{p}{)}
\end{Verbatim}


    On est exactement dans le cas où il nous faut utiliser
\texttt{yield\ from}. En effet lorsqu'on appelle
\texttt{gen\_perm(subject,\ k+1)} ici, ce qu'on obtient en retour c'est
maintenant un objet générateur. Pour faire ce qu'on cherche à faire il
nous faut bien utiliser cet objet générateur, et pour cela on utilise
\texttt{yield\ from}.

    \begin{Verbatim}[commandchars=\\\{\},frame=single,framerule=0.3mm,rulecolor=\color{cellframecolor}]
{\color{incolor}In [{\color{incolor}12}]:} \PY{c+c1}{\PYZsh{} pour simplifier ici on suppose que l\PYZsq{}entrée est une vraie liste}
         \PY{c+c1}{\PYZsh{} que l\PYZsq{}on va ainsi pouvoir modifier par effets de bord}
         \PY{k}{def} \PY{n+nf}{gen\PYZus{}perm3}\PY{p}{(}\PY{n}{subject}\PY{p}{,} \PY{n}{k}\PY{o}{=}\PY{l+m+mi}{0}\PY{p}{)}\PY{p}{:}
             \PY{k}{if} \PY{n}{k} \PY{o}{==} \PY{n+nb}{len}\PY{p}{(}\PY{n}{subject}\PY{p}{)}\PY{p}{:}
                 \PY{c+c1}{\PYZsh{} cette version hyper rustique se contente de faire une impression}
                 \PY{k}{yield} \PY{n}{subject}
             \PY{k}{else}\PY{p}{:}
                 \PY{k}{for} \PY{n}{i} \PY{o+ow}{in} \PY{n+nb}{range}\PY{p}{(}\PY{n}{k}\PY{p}{,} \PY{n+nb}{len}\PY{p}{(}\PY{n}{subject}\PY{p}{)}\PY{p}{)}\PY{p}{:}
                     \PY{c+c1}{\PYZsh{} on échange }
                     \PY{n}{subject}\PY{p}{[}\PY{n}{k}\PY{p}{]}\PY{p}{,} \PY{n}{subject}\PY{p}{[}\PY{n}{i}\PY{p}{]} \PY{o}{=} \PY{n}{subject}\PY{p}{[}\PY{n}{i}\PY{p}{]}\PY{p}{,} \PY{n}{subject}\PY{p}{[}\PY{n}{k}\PY{p}{]}
                     \PY{k}{yield from} \PY{n}{gen\PYZus{}perm3}\PY{p}{(}\PY{n}{subject}\PY{p}{,} \PY{n}{k}\PY{o}{+}\PY{l+m+mi}{1}\PY{p}{)}
                     \PY{c+c1}{\PYZsh{} on remet comme c\PYZsq{}était pour le prochain échange}
                     \PY{n}{subject}\PY{p}{[}\PY{n}{k}\PY{p}{]}\PY{p}{,} \PY{n}{subject}\PY{p}{[}\PY{n}{i}\PY{p}{]} \PY{o}{=} \PY{n}{subject}\PY{p}{[}\PY{n}{i}\PY{p}{]}\PY{p}{,} \PY{n}{subject}\PY{p}{[}\PY{n}{k}\PY{p}{]}
\end{Verbatim}


    \begin{Verbatim}[commandchars=\\\{\},frame=single,framerule=0.3mm,rulecolor=\color{cellframecolor}]
{\color{incolor}In [{\color{incolor}13}]:} \PY{k}{for} \PY{n}{perm} \PY{o+ow}{in} \PY{n}{gen\PYZus{}perm3}\PY{p}{(}\PY{p}{[}\PY{l+s+s1}{\PYZsq{}}\PY{l+s+s1}{a}\PY{l+s+s1}{\PYZsq{}}\PY{p}{,} \PY{l+s+s1}{\PYZsq{}}\PY{l+s+s1}{b}\PY{l+s+s1}{\PYZsq{}}\PY{p}{,} \PY{l+s+s1}{\PYZsq{}}\PY{l+s+s1}{c}\PY{l+s+s1}{\PYZsq{}}\PY{p}{,} \PY{l+s+s1}{\PYZsq{}}\PY{l+s+s1}{d}\PY{l+s+s1}{\PYZsq{}}\PY{p}{]}\PY{p}{)}\PY{p}{:}
             \PY{n+nb}{print}\PY{p}{(}\PY{n}{perm}\PY{p}{)}
\end{Verbatim}


    \begin{Verbatim}[commandchars=\\\{\},frame=single,framerule=0.3mm,rulecolor=\color{cellframecolor}]
['a', 'b', 'c', 'd']
['a', 'b', 'd', 'c']
['a', 'c', 'b', 'd']
['a', 'c', 'd', 'b']
['a', 'd', 'c', 'b']
['a', 'd', 'b', 'c']
['b', 'a', 'c', 'd']
['b', 'a', 'd', 'c']
['b', 'c', 'a', 'd']
['b', 'c', 'd', 'a']
['b', 'd', 'c', 'a']
['b', 'd', 'a', 'c']
['c', 'b', 'a', 'd']
['c', 'b', 'd', 'a']
['c', 'a', 'b', 'd']
['c', 'a', 'd', 'b']
['c', 'd', 'a', 'b']
['c', 'd', 'b', 'a']
['d', 'b', 'c', 'a']
['d', 'b', 'a', 'c']
['d', 'c', 'b', 'a']
['d', 'c', 'a', 'b']
['d', 'a', 'c', 'b']
['d', 'a', 'b', 'c']
\end{Verbatim}


    % Add a bibliography block to the postdoc
    
    
    
