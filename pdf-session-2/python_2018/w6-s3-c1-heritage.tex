    
    
    
    

    

    \hypertarget{huxe9ritage}{%
\section{Héritage}\label{huxe9ritage}}

    \hypertarget{compluxe9ment---niveau-basique}{%
\subsection{Complément - niveau
basique}\label{compluxe9ment---niveau-basique}}

    La notion d'héritage, qui fait partie intégrante de la Programmation
Orientée Objet, permet principalement de maximiser la
\textbf{réutilisabilité}.

Nous avons vu dans la vidéo les mécanismes d'héritage \emph{in
abstracto}. Pour résumer très brièvement, on recherche un attribut (pour
notre propos, disons une méthode) à partir d'une instance en cherchant~:

\begin{itemize}
\tightlist
\item
  d'abord dans l'instance elle-même~;
\item
  puis dans la classe de l'instance~;
\item
  puis dans les super-classes de la classe.
\end{itemize}

    L'objet de ce complément est de vous donner, d'un point de vue plus
appliqué, des idées de ce que l'on peut faire ou non avec ce mécanisme.
Le sujet étant assez rabâché par ailleurs, nous nous concentrerons sur
deux points~:

\begin{itemize}
\tightlist
\item
  les pratiques de base utilisées pour la conception de classes, et
  notamment pour bien distinguer \textbf{héritage} et
  \textbf{composition}~;
\item
  nous verrons ensuite, sur des \textbf{exemples extraits de code réel},
  comment ces mécanismes permettent en effet de contribuer à la
  réutilisabilité du code.
\end{itemize}

    \hypertarget{plusieurs-formes-dhuxe9ritage}{%
\subsubsection{Plusieurs formes
d'héritage}\label{plusieurs-formes-dhuxe9ritage}}

    Une méthode héritée peut être rangée dans une de ces trois catégories~:

\begin{itemize}
\tightlist
\item
  \emph{implicite}~: si la classe fille ne définit pas du tout la
  méthode~;
\item
  \emph{redéfinie}~: si on récrit la méthode entièrement~;
\item
  \emph{modifiée}~: si on récrit la méthode dans la classe fille, mais
  en utilisant le code de la classe mère.
\end{itemize}

    Commençons par illustrer tout ceci sur un petit exemple~:

    \begin{Verbatim}[commandchars=\\\{\},frame=single,framerule=0.3mm,rulecolor=\color{cellframecolor}]
{\color{incolor}In [{\color{incolor}1}]:} \PY{c+c1}{\PYZsh{} Une classe mère}
        \PY{k}{class} \PY{n+nc}{Fleur}\PY{p}{:}
            \PY{k}{def} \PY{n+nf}{implicite}\PY{p}{(}\PY{n+nb+bp}{self}\PY{p}{)}\PY{p}{:}
                \PY{n+nb}{print}\PY{p}{(}\PY{l+s+s1}{\PYZsq{}}\PY{l+s+s1}{Fleur.implicite}\PY{l+s+s1}{\PYZsq{}}\PY{p}{)}
            \PY{k}{def} \PY{n+nf}{redefinie}\PY{p}{(}\PY{n+nb+bp}{self}\PY{p}{)}\PY{p}{:}
                \PY{n+nb}{print}\PY{p}{(}\PY{l+s+s1}{\PYZsq{}}\PY{l+s+s1}{Fleur.redéfinie}\PY{l+s+s1}{\PYZsq{}}\PY{p}{)}
            \PY{k}{def} \PY{n+nf}{modifiee}\PY{p}{(}\PY{n+nb+bp}{self}\PY{p}{)}\PY{p}{:}
                \PY{n+nb}{print}\PY{p}{(}\PY{l+s+s1}{\PYZsq{}}\PY{l+s+s1}{Fleur.modifiée}\PY{l+s+s1}{\PYZsq{}}\PY{p}{)}
        
        \PY{c+c1}{\PYZsh{} Une classe fille}
        \PY{k}{class} \PY{n+nc}{Rose}\PY{p}{(}\PY{n}{Fleur}\PY{p}{)}\PY{p}{:}
            \PY{c+c1}{\PYZsh{} on ne définit pas implicite}
            \PY{c+c1}{\PYZsh{} on rédéfinit complètement redefinie}
            \PY{k}{def} \PY{n+nf}{redefinie}\PY{p}{(}\PY{n+nb+bp}{self}\PY{p}{)}\PY{p}{:}
                \PY{n+nb}{print}\PY{p}{(}\PY{l+s+s1}{\PYZsq{}}\PY{l+s+s1}{Rose.redefinie}\PY{l+s+s1}{\PYZsq{}}\PY{p}{)}
            \PY{c+c1}{\PYZsh{} on change un peu le comportement de modifiee}
            \PY{k}{def} \PY{n+nf}{modifiee}\PY{p}{(}\PY{n+nb+bp}{self}\PY{p}{)}\PY{p}{:}
                \PY{n}{Fleur}\PY{o}{.}\PY{n}{modifiee}\PY{p}{(}\PY{n+nb+bp}{self}\PY{p}{)}
                \PY{n+nb}{print}\PY{p}{(}\PY{l+s+s1}{\PYZsq{}}\PY{l+s+s1}{Rose.modifiee apres Fleur}\PY{l+s+s1}{\PYZsq{}}\PY{p}{)}
\end{Verbatim}


    On peut à présent créer une instance de Rose et appeler sur cette
instance les trois méthodes.

    \begin{Verbatim}[commandchars=\\\{\},frame=single,framerule=0.3mm,rulecolor=\color{cellframecolor}]
{\color{incolor}In [{\color{incolor}2}]:} \PY{c+c1}{\PYZsh{} fille est une instance de Rose}
        \PY{n}{fille} \PY{o}{=} \PY{n}{Rose}\PY{p}{(}\PY{p}{)}
        
        \PY{n}{fille}\PY{o}{.}\PY{n}{implicite}\PY{p}{(}\PY{p}{)}
\end{Verbatim}


    \begin{Verbatim}[commandchars=\\\{\},frame=single,framerule=0.3mm,rulecolor=\color{cellframecolor}]
Fleur.implicite
\end{Verbatim}

    \begin{Verbatim}[commandchars=\\\{\},frame=single,framerule=0.3mm,rulecolor=\color{cellframecolor}]
{\color{incolor}In [{\color{incolor}3}]:} \PY{n}{fille}\PY{o}{.}\PY{n}{redefinie}\PY{p}{(}\PY{p}{)}
\end{Verbatim}


    \begin{Verbatim}[commandchars=\\\{\},frame=single,framerule=0.3mm,rulecolor=\color{cellframecolor}]
Rose.redefinie
\end{Verbatim}

    S'agissant des deux premières méthodes, le comportement qu'on observe
est simplement la conséquence de l'algorithme de recherche d'attributs~:
\texttt{implicite} est trouvée dans la classe Fleur et
\texttt{redefinie} est trouvée dans la classe Rose.

    \begin{Verbatim}[commandchars=\\\{\},frame=single,framerule=0.3mm,rulecolor=\color{cellframecolor}]
{\color{incolor}In [{\color{incolor}4}]:} \PY{n}{fille}\PY{o}{.}\PY{n}{modifiee}\PY{p}{(}\PY{p}{)}
\end{Verbatim}


    \begin{Verbatim}[commandchars=\\\{\},frame=single,framerule=0.3mm,rulecolor=\color{cellframecolor}]
Fleur.modifiée
Rose.modifiee apres Fleur
\end{Verbatim}

    Pour la troisième méthode, attardons-nous un peu car on voit ici comment
\emph{combiner} facilement le code de la classe mère avec du code
spécifique à la classe fille, en appelant explicitement la méthode de la
classe mère lorsqu'on fait~:

\begin{Shaded}
\begin{Highlighting}[frame=lines,framerule=0.6mm,rulecolor=\color{asisframecolor}]
\NormalTok{Fleur.modifiee(}\VariableTok{self}\NormalTok{)}
\end{Highlighting}
\end{Shaded}

    \hypertarget{la-fonction-built-in-super}{%
\subparagraph{\texorpdfstring{La fonction \emph{built-in}
\texttt{super()}}{La fonction built-in super()}}\label{la-fonction-built-in-super}}

    Signalons à ce sujet, pour être exhaustif, l'existence de la
\href{https://docs.python.org/3/library/functions.html\#super}{fonction
\emph{built-in} \texttt{super()}} qui permet de réaliser la même chose
sans nommer explicitement la classe mère, comme on le fait ici~:

    \begin{Verbatim}[commandchars=\\\{\},frame=single,framerule=0.3mm,rulecolor=\color{cellframecolor}]
{\color{incolor}In [{\color{incolor}5}]:} \PY{c+c1}{\PYZsh{} Une version allégée de la classe fille, qui utilise super()}
        \PY{k}{class} \PY{n+nc}{Rose}\PY{p}{(}\PY{n}{Fleur}\PY{p}{)}\PY{p}{:}
            \PY{k}{def} \PY{n+nf}{modifiee}\PY{p}{(}\PY{n+nb+bp}{self}\PY{p}{)}\PY{p}{:}
                \PY{n+nb}{super}\PY{p}{(}\PY{p}{)}\PY{o}{.}\PY{n}{modifiee}\PY{p}{(}\PY{p}{)}
                \PY{n+nb}{print}\PY{p}{(}\PY{l+s+s1}{\PYZsq{}}\PY{l+s+s1}{Rose.modifiee apres Fleur}\PY{l+s+s1}{\PYZsq{}}\PY{p}{)}
\end{Verbatim}


    \begin{Verbatim}[commandchars=\\\{\},frame=single,framerule=0.3mm,rulecolor=\color{cellframecolor}]
{\color{incolor}In [{\color{incolor}6}]:} \PY{n}{fille} \PY{o}{=} \PY{n}{Rose}\PY{p}{(}\PY{p}{)}
        
        \PY{n}{fille}\PY{o}{.}\PY{n}{modifiee}\PY{p}{(}\PY{p}{)}
\end{Verbatim}


    \begin{Verbatim}[commandchars=\\\{\},frame=single,framerule=0.3mm,rulecolor=\color{cellframecolor}]
Fleur.modifiée
Rose.modifiee apres Fleur
\end{Verbatim}

    On peut envisager d'utiliser \texttt{super()} dans du code très abstrait
où on ne sait pas déterminer statiquement le nom de la classe dont il
est question. Mais, c'est une question de goût évidemment, je recommande
personnellement la première forme, où on qualifie la méthode avec le nom
de la classe mère qu'on souhaite utiliser. En effet, assez souvent~:

\begin{itemize}
\tightlist
\item
  on sait déterminer le nom de la classe dont il est question~;
\item
  ou alors on veut mélanger plusieurs méthodes héritées (via l'héritage
  multiple, dont on va parler dans un prochain complément) et dans ce
  cas \texttt{super()} ne peut rien pour nous.
\end{itemize}

    \hypertarget{huxe9ritage-vs-composition}{%
\subsubsection{\texorpdfstring{Héritage \emph{vs}
Composition}{Héritage vs Composition}}\label{huxe9ritage-vs-composition}}

    Dans le domaine de la conception orientée objet, on fait la différence
entre deux constructions, l'héritage et la composition, qui à une
analyse superficielle peuvent paraître procurer des résultats
similaires, mais qu'il est important de bien distinguer.

Voyons d'abord en quoi consiste la composition et pourquoi le résultat
est voisin~:

    \begin{Verbatim}[commandchars=\\\{\},frame=single,framerule=0.3mm,rulecolor=\color{cellframecolor}]
{\color{incolor}In [{\color{incolor}7}]:} \PY{c+c1}{\PYZsh{} Une classe avec qui on n\PYZsq{}aura pas de relation d\PYZsq{}héritage}
        \PY{k}{class} \PY{n+nc}{Tige}\PY{p}{:}
            \PY{k}{def} \PY{n+nf}{implicite}\PY{p}{(}\PY{n+nb+bp}{self}\PY{p}{)}\PY{p}{:}
                \PY{n+nb}{print}\PY{p}{(}\PY{l+s+s1}{\PYZsq{}}\PY{l+s+s1}{Tige.implicite}\PY{l+s+s1}{\PYZsq{}}\PY{p}{)}
            \PY{k}{def} \PY{n+nf}{redefinie}\PY{p}{(}\PY{n+nb+bp}{self}\PY{p}{)}\PY{p}{:}
                \PY{n+nb}{print}\PY{p}{(}\PY{l+s+s1}{\PYZsq{}}\PY{l+s+s1}{Tige.redefinie}\PY{l+s+s1}{\PYZsq{}}\PY{p}{)}
            \PY{k}{def} \PY{n+nf}{modifiee}\PY{p}{(}\PY{n+nb+bp}{self}\PY{p}{)}\PY{p}{:}
                \PY{n+nb}{print}\PY{p}{(}\PY{l+s+s1}{\PYZsq{}}\PY{l+s+s1}{Tige.modifiee}\PY{l+s+s1}{\PYZsq{}}\PY{p}{)}
        
        \PY{c+c1}{\PYZsh{} on n\PYZsq{}hérite pas}
        \PY{c+c1}{\PYZsh{} mais on fait ce qu\PYZsq{}on appelle une composition}
        \PY{c+c1}{\PYZsh{} avec la classe Tige}
        \PY{k}{class} \PY{n+nc}{Rose}\PY{p}{:}
            \PY{c+c1}{\PYZsh{} mais pour chaque objet de la classe Rose}
            \PY{c+c1}{\PYZsh{} on va créer un objet de la classe Tige}
            \PY{c+c1}{\PYZsh{} et le conserver dans un champ}
            \PY{k}{def} \PY{n+nf}{\PYZus{}\PYZus{}init\PYZus{}\PYZus{}}\PY{p}{(}\PY{n+nb+bp}{self}\PY{p}{)}\PY{p}{:}
                \PY{n+nb+bp}{self}\PY{o}{.}\PY{n}{externe} \PY{o}{=} \PY{n}{Tige}\PY{p}{(}\PY{p}{)}
        
            \PY{c+c1}{\PYZsh{} le reste est presque comme tout à l\PYZsq{}heure}
            \PY{c+c1}{\PYZsh{} sauf qu\PYZsq{}il faut definir implicite}
            \PY{k}{def} \PY{n+nf}{implicite}\PY{p}{(}\PY{n+nb+bp}{self}\PY{p}{)}\PY{p}{:}
                \PY{n+nb+bp}{self}\PY{o}{.}\PY{n}{externe}\PY{o}{.}\PY{n}{implicite}\PY{p}{(}\PY{p}{)}
                
            \PY{c+c1}{\PYZsh{} on redéfinit complètement redefinie}
            \PY{k}{def} \PY{n+nf}{redefinie}\PY{p}{(}\PY{n+nb+bp}{self}\PY{p}{)}\PY{p}{:}
                \PY{n+nb}{print}\PY{p}{(}\PY{l+s+s1}{\PYZsq{}}\PY{l+s+s1}{Rose.redefinie}\PY{l+s+s1}{\PYZsq{}}\PY{p}{)}
                
            \PY{k}{def} \PY{n+nf}{modifiee}\PY{p}{(}\PY{n+nb+bp}{self}\PY{p}{)}\PY{p}{:}
                \PY{n+nb+bp}{self}\PY{o}{.}\PY{n}{externe}\PY{o}{.}\PY{n}{modifiee}\PY{p}{(}\PY{p}{)}
                \PY{n+nb}{print}\PY{p}{(}\PY{l+s+s1}{\PYZsq{}}\PY{l+s+s1}{Rose.modifiee apres Tige}\PY{l+s+s1}{\PYZsq{}}\PY{p}{)}
\end{Verbatim}


    \begin{Verbatim}[commandchars=\\\{\},frame=single,framerule=0.3mm,rulecolor=\color{cellframecolor}]
{\color{incolor}In [{\color{incolor}8}]:} \PY{c+c1}{\PYZsh{} on obtient ici exactement le même comportement }
        \PY{c+c1}{\PYZsh{} pour les trois sortes de méthodes}
        \PY{n}{fille} \PY{o}{=} \PY{n}{Rose}\PY{p}{(}\PY{p}{)}
        
        \PY{n}{fille}\PY{o}{.}\PY{n}{implicite}\PY{p}{(}\PY{p}{)}
        \PY{n}{fille}\PY{o}{.}\PY{n}{redefinie}\PY{p}{(}\PY{p}{)}
        \PY{n}{fille}\PY{o}{.}\PY{n}{modifiee}\PY{p}{(}\PY{p}{)}
\end{Verbatim}


    \begin{Verbatim}[commandchars=\\\{\},frame=single,framerule=0.3mm,rulecolor=\color{cellframecolor}]
Tige.implicite
Rose.redefinie
Tige.modifiee
Rose.modifiee apres Tige
\end{Verbatim}

    \hypertarget{comment-choisir}{%
\subparagraph{Comment choisir~?}\label{comment-choisir}}

    Alors, quand faut-il utiliser l'héritage et quand faut-il utiliser la
composition~?\\
On arrive ici à la limite de notre cours, il s'agit plus de conception
que de codage à proprement parler, mais pour donner une réponse très
courte à cette question~:

\begin{itemize}
\tightlist
\item
  on utilise l'héritage lorsqu'un objet de la sous-classe \textbf{est
  aussi un} objet de la super-classe (une rose est aussi une fleur)~;
\item
  on utilise la composition lorsqu'un objet de la sous-classe \textbf{a
  une relation avec} un objet de la super-classe (une rose possède une
  tige, mais c'est un autre objet).
\end{itemize}

    \hypertarget{compluxe9ment---niveau-intermuxe9diaire}{%
\subsection{Complément - niveau
intermédiaire}\label{compluxe9ment---niveau-intermuxe9diaire}}

    \hypertarget{des-exemples-de-code}{%
\subsubsection{Des exemples de code}\label{des-exemples-de-code}}

    Sans transition, dans cette section un peu plus prospective, nous vous
avons signalé quelques morceaux de code de la bibliothèque standard qui
utilisent l'héritage. Sans aller nécessairement jusqu'à la lecture de
ces codes, il nous a semblé intéressant de commenter spécifiquement
l'usage qui est fait de l'héritage dans ces bibliothèques.

    \hypertarget{le-module-calendar}{%
\subparagraph{\texorpdfstring{Le module
\texttt{calendar}}{Le module calendar}}\label{le-module-calendar}}

    On trouve dans la bibliothèque standard
\href{https://docs.python.org/3/library/calendar.html}{le module
\texttt{calendar}}. Ce module expose deux classes \texttt{TextCalendar}
et \texttt{HTMLCalendar}. Sans entrer du tout dans le détail, ces deux
classes permettent d'imprimer dans des formats différents le même type
d'informations.

Le point ici est que les concepteurs ont choisi un graphe d'héritage
comme ceci~:

\begin{Shaded}
\begin{Highlighting}[frame=lines,framerule=0.6mm,rulecolor=\color{asisframecolor}]
\NormalTok{    Calendar}
    \OperatorTok{|--}\NormalTok{ TextCalendar}
    \OperatorTok{|--}\NormalTok{ HTMLCalendar}
\end{Highlighting}
\end{Shaded}

qui permet de grouper le code concernant la logique dans la classe
\texttt{Calendar}, puis dans les deux sous-classes d'implémenter le type
de sortie qui va bien.

C'est l'utilisateur qui choisit la classe qui lui convient le mieux, et
de cette manière, le maximum de code est partagé entre les deux
classes~; et de plus si vous avez besoin d'une sortie au format, disons
PDF, vous pouvez envisager d'hériter de \texttt{Calendar} et de
n'implémenter que la partie spécifique au format PDF.

C'est un peu le niveau élémentaire de l'héritage.

    \hypertarget{le-module-socketserver}{%
\subparagraph{\texorpdfstring{Le module
\texttt{SocketServer}}{Le module SocketServer}}\label{le-module-socketserver}}

    Toujours dans la bibliothèque standard, le
\href{https://docs.python.org/3/library/socketserver.html}{module
\texttt{SocketServer}} fait un usage beaucoup plus sophistiqué de
l'héritage.

Le module propose une hiérarchie de classes comme ceci~:

    \begin{Shaded}
\begin{Highlighting}[frame=lines,framerule=0.6mm,rulecolor=\color{asisframecolor}]
    \OperatorTok{+------------+}
    \OperatorTok{|}\NormalTok{ BaseServer }\OperatorTok{|}
    \OperatorTok{+------------+}
          \OperatorTok{|}
\NormalTok{          v}
    \OperatorTok{+-----------+}        \OperatorTok{+------------------+}
    \OperatorTok{|}\NormalTok{ TCPServer }\OperatorTok{|------->|}\NormalTok{ UnixStreamServer }\OperatorTok{|}
    \OperatorTok{+-----------+}        \OperatorTok{+------------------+}
          \OperatorTok{|}
\NormalTok{          v}
    \OperatorTok{+-----------+}        \OperatorTok{+--------------------+}
    \OperatorTok{|}\NormalTok{ UDPServer }\OperatorTok{|------->|}\NormalTok{ UnixDatagramServer }\OperatorTok{|}
    \OperatorTok{+-----------+}        \OperatorTok{+--------------------+}
\end{Highlighting}
\end{Shaded}

    Ici encore notre propos n'est pas d'entrer dans les détails, mais
d'observer le fait que les différents \emph{niveaux d'intelligence} sont
ajoutés les uns aux les autres au fur et à mesure que l'on descend le
graphe d'héritage.

Cette hiérarchie est par ailleurs étendue par le
\href{https://docs.python.org/3/library/http.server.html}{module
\texttt{http.server}} et notamment au travers de la classe
\href{https://docs.python.org/3/library/http.server.html\#http.server.HTTPServer}{\texttt{HTTPServer}}
qui hérite de \texttt{TCPServer}.

    Pour revenir au module \texttt{SocketServer}, j'attire votre attention
dans
\href{https://docs.python.org/3/library/socketserver.html\#examples}{la
page d'exemples} sur
\href{https://docs.python.org/3/library/socketserver.html\#asynchronous-mixins}{cet
exemple en particuler}, où on crée une classe de serveurs multi-threads
(la classe \texttt{ThreadedTCPServer}) par simple héritage multiple
entre \texttt{ThreadingMixIn} et \texttt{TCPServer}. La notion de
\texttt{Mixin} est \href{http://en.wikipedia.org/wiki/Mixin}{décrite
dans cette page Wikipédia} dans laquelle vous pouvez accéder directement
\href{http://en.wikipedia.org/wiki/Mixin\#In_Python}{à la section
consacrée à Python}.


    % Add a bibliography block to the postdoc
    
    
    
