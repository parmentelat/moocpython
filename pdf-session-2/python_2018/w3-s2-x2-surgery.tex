    \hypertarget{sequence-unpacking}{%
\section{\texorpdfstring{\emph{Sequence
unpacking}}{Sequence unpacking}}\label{sequence-unpacking}}

    \hypertarget{exercice---niveau-basique}{%
\subsection{Exercice - niveau basique}\label{exercice---niveau-basique}}

    \begin{Verbatim}[commandchars=\\\{\}]
{\color{incolor}In [{\color{incolor} }]:} \PY{c+c1}{\PYZsh{} chargeons l\PYZsq{}exercice}
        \PY{k+kn}{from} \PY{n+nn}{corrections}\PY{n+nn}{.}\PY{n+nn}{exo\PYZus{}surgery} \PY{k}{import} \PY{n}{exo\PYZus{}surgery}
\end{Verbatim}


    Cet exercice consiste à écrire une fonction \texttt{surgery}, qui prend
en argument une liste, et qui retourne la \textbf{même} liste
\textbf{modifiée} comme suit~:

\begin{itemize}
	\item 
	si la liste est de taille 0 ou 1, elle n'est pas modifiée~;
	\item
	si la liste est de taille paire, on intervertit
	les deux premiers éléments de la liste~;
	\item
	si elle est de taille impaire, on intervertit les deux derniers éléments.
\end{itemize}

    \begin{Verbatim}[commandchars=\\\{\}]
{\color{incolor}In [{\color{incolor} }]:} \PY{c+c1}{\PYZsh{} voici quelques exemples de ce qui est attendu}
        \PY{n}{exo\PYZus{}surgery}\PY{o}{.}\PY{n}{example}\PY{p}{(}\PY{p}{)}
\end{Verbatim}


    \begin{Verbatim}[commandchars=\\\{\}]
{\color{incolor}In [{\color{incolor} }]:} \PY{c+c1}{\PYZsh{} écrivez votre code}
        \PY{k}{def} \PY{n+nf}{surgery}\PY{p}{(}\PY{n}{liste}\PY{p}{)}\PY{p}{:}
            \PY{l+s+s2}{\PYZdq{}}\PY{l+s+s2}{\PYZlt{}votre\PYZus{}code\PYZgt{}}\PY{l+s+s2}{\PYZdq{}}
\end{Verbatim}


    \begin{Verbatim}[commandchars=\\\{\}]
{\color{incolor}In [{\color{incolor} }]:} \PY{c+c1}{\PYZsh{} pour le vérifier, évaluez cette cellule}
        \PY{n}{exo\PYZus{}surgery}\PY{o}{.}\PY{n}{correction}\PY{p}{(}\PY{n}{surgery}\PY{p}{)}
\end{Verbatim}