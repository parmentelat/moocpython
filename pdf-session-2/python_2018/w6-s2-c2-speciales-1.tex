    
    
    
    

    

    \hypertarget{surcharge-dopuxe9rateurs-1}{%
\section{Surcharge d'opérateurs (1)}\label{surcharge-dopuxe9rateurs-1}}

    \hypertarget{compluxe9ment---niveau-intermuxe9diaire}{%
\subsection{Complément - niveau
intermédiaire}\label{compluxe9ment---niveau-intermuxe9diaire}}

    Ce complément vise à illustrer certaines des possibilités de surcharge
d'opérateurs, ou plus généralement les mécanismes disponibles pour
étendre le langage et donner un sens à des fragments de code comme~:

\begin{itemize}
\tightlist
\item
  \texttt{objet1\ +\ objet2}
\item
  \texttt{item\ in\ objet}
\item
  \texttt{objet{[}key{]}}
\item
  \texttt{objet.key}
\item
  \texttt{for\ i\ in\ objet:}
\item
  \texttt{if\ objet:}
\item
  \texttt{objet(arg1,\ arg2)} (et non pas \texttt{classe(arg1,\ arg2)})
\item
  etc..
\end{itemize}

que jusqu'ici, sauf pour la boucle \texttt{for} et pour le hachage, on
n'a expliqués que pour des objets de type prédéfini.

    Le mécanisme général pour cela consiste à définir des \textbf{méthodes
spéciales}, avec un nom en \texttt{\_\_nom\_\_}. Il existe un total de
près de 80 méthodes dans ce système de surcharges, aussi il n'est pas
question ici d'être exhaustif. Vous trouverez
\href{https://docs.python.org/3/reference/datamodel.html\#specialnames}{dans
ce document une liste complète de ces possibilités}.

    Il nous faut également signaler que les mécanismes mis en jeu ici sont
\textbf{de difficultés assez variables}. Dans le cas le plus simple il
suffit de définir une méthode sur la classe pour obtenir le résultat
(par exemple, définir \texttt{\_\_call\_\_} pour rendre un objet
callable). Mais parfois on parle d'un ensemble de méthodes qui doivent
être cohérentes, voyez par exemple les
\href{https://docs.python.org/3/reference/datamodel.html\#invoking-descriptors}{descriptors}
qui mettent en jeu les méthodes \texttt{\_\_get\_\_},
\texttt{\_\_set\_\_} et \texttt{\_\_delete\_\_}, et qui peuvent sembler
particulièrement cryptiques. On aura d'ailleurs l'occasion d'approfondir
les descriptors en semaine 9 avec les sujets avancés.

    Nous vous conseillons de commencer par des choses simples, et surtout de
n'utiliser ces techniques que lorsqu'elles apportent vraiment quelque
chose. Le constructeur et l'affichage sont pratiquement toujours
définis, mais pour tout le reste il convient d'utiliser ces traits avec
le plus grand discernement. Dans tous les cas écrivez votre code avec la
documentation sous les yeux, c'est plus prudent :)

    Nous avons essayé de présenter cette sélection par difficulté
croissante. Par ailleurs, et pour alléger la présentation, cet exposé a
été coupé en trois notebooks différents.

    \begin{center}\rule{0.5\linewidth}{\linethickness}\end{center}

    \hypertarget{rappels}{%
\subsubsection{Rappels}\label{rappels}}

    Pour rappel, on a vu dans la vidéo~:

\begin{itemize}
\tightlist
\item
  la méthode \texttt{\_\_init\_\_} pour définir un
  \textbf{constructeur}~;
\item
  la méthode \texttt{\_\_str\_\_} pour définir comment une instance
  s'imprime avec \texttt{print}.
\end{itemize}

    \hypertarget{affichage-__repr__-et-__str__}{%
\subsubsection{\texorpdfstring{Affichage~: \texttt{\_\_repr\_\_} et
\texttt{\_\_str\_\_}}{Affichage~: \_\_repr\_\_ et \_\_str\_\_}}\label{affichage-__repr__-et-__str__}}

    Nous commençons par signaler la méthode \texttt{\_\_repr\_\_} qui est
assez voisine de \texttt{\_\_str\_\_}, et qui donc doit retourner un
objet de type chaîne de caractères, sauf que~:

\begin{itemize}
\tightlist
\item
  \texttt{\_\_str\_\_} est utilisée par \texttt{print} (affichage
  orienté utilisateur du programme, priorité au confort visuel)~;
\item
  alors que \texttt{\_\_repr\_\_} est utilisée par la fonction
  \texttt{repr()} (affichage orienté programmeur, aussi peu ambigu que
  possible)~;
\item
  enfin il faut savoir que \texttt{\_\_repr\_\_} est utilisée
  \textbf{aussi} par \texttt{print} si \texttt{\_\_str\_\_} n'est pas
  définie.
\end{itemize}

    Pour cette dernière raison, on trouve dans la nature
\texttt{\_\_repr\_\_} plutôt plus souvent que \texttt{\_\_str\_\_}~;
voyez
\href{https://docs.python.org/3/reference/datamodel.html\#object.__repr__}{ce
lien} pour davantage de détails.

    \hypertarget{quand-est-utilisuxe9e-repr}{%
\subparagraph{\texorpdfstring{Quand est utilisée
\texttt{repr()}~?}{Quand est utilisée repr()~?}}\label{quand-est-utilisuxe9e-repr}}

    La fonction \texttt{repr()} est utilisée massivement dans les
informations de debugging comme les traces de pile lorsqu'une exception
est levée. Elle est aussi utilisée lorsque vous affichez un objet sans
passer par \texttt{print}, c'est-à-dire par exemple~:

    \begin{Verbatim}[commandchars=\\\{\}]
{\color{incolor}In [{\color{incolor}1}]:} \PY{k}{class} \PY{n+nc}{Foo}\PY{p}{:} 
            \PY{k}{def} \PY{n+nf}{\PYZus{}\PYZus{}repr\PYZus{}\PYZus{}}\PY{p}{(}\PY{n+nb+bp}{self}\PY{p}{)}\PY{p}{:}
                \PY{k}{return} \PY{l+s+s1}{\PYZsq{}}\PY{l+s+s1}{custom repr}\PY{l+s+s1}{\PYZsq{}}
            
        \PY{n}{foo} \PY{o}{=} \PY{n}{Foo}\PY{p}{(}\PY{p}{)}
        \PY{c+c1}{\PYZsh{} lorsque vous affichez un objet comme ceci}
        \PY{n}{foo}
        \PY{c+c1}{\PYZsh{} en fait vous utilisez repr()}
\end{Verbatim}


\begin{Verbatim}[commandchars=\\\{\}]
{\color{outcolor}Out[{\color{outcolor}1}]:} custom repr
\end{Verbatim}
            
    \hypertarget{deux-exemples}{%
\subparagraph{Deux exemples}\label{deux-exemples}}

    Voici deux exemples simples de classes~; dans le premier on n'a défini
que \texttt{\_\_repr\_\_}, dans le second on a redéfini les deux
méthodes~:

    \begin{Verbatim}[commandchars=\\\{\}]
{\color{incolor}In [{\color{incolor}2}]:} \PY{c+c1}{\PYZsh{} une classe qui ne définit que \PYZus{}\PYZus{}repr\PYZus{}\PYZus{}}
        \PY{k}{class} \PY{n+nc}{Point}\PY{p}{:}
            \PY{l+s+s2}{\PYZdq{}}\PY{l+s+s2}{première version de Point \PYZhy{} on ne définit que \PYZus{}\PYZus{}repr\PYZus{}\PYZus{}}\PY{l+s+s2}{\PYZdq{}}
            \PY{k}{def} \PY{n+nf}{\PYZus{}\PYZus{}init\PYZus{}\PYZus{}}\PY{p}{(}\PY{n+nb+bp}{self}\PY{p}{,} \PY{n}{x}\PY{p}{,} \PY{n}{y}\PY{p}{)}\PY{p}{:}
                \PY{n+nb+bp}{self}\PY{o}{.}\PY{n}{x} \PY{o}{=} \PY{n}{x}
                \PY{n+nb+bp}{self}\PY{o}{.}\PY{n}{y} \PY{o}{=} \PY{n}{y}
            \PY{k}{def} \PY{n+nf}{\PYZus{}\PYZus{}repr\PYZus{}\PYZus{}}\PY{p}{(}\PY{n+nb+bp}{self}\PY{p}{)}\PY{p}{:}
                \PY{k}{return} \PY{n}{f}\PY{l+s+s2}{\PYZdq{}}\PY{l+s+s2}{Point(}\PY{l+s+si}{\PYZob{}self.x\PYZcb{}}\PY{l+s+s2}{,}\PY{l+s+si}{\PYZob{}self.y\PYZcb{}}\PY{l+s+s2}{)}\PY{l+s+s2}{\PYZdq{}}
            
        \PY{n}{point} \PY{o}{=} \PY{n}{Point} \PY{p}{(}\PY{l+m+mi}{0}\PY{p}{,}\PY{l+m+mi}{100}\PY{p}{)}
        
        \PY{n+nb}{print}\PY{p}{(}\PY{l+s+s2}{\PYZdq{}}\PY{l+s+s2}{avec print}\PY{l+s+s2}{\PYZdq{}}\PY{p}{,} \PY{n}{point}\PY{p}{)}
        
        \PY{c+c1}{\PYZsh{} si vous affichez un objet sans passer par print}
        \PY{c+c1}{\PYZsh{} vous utilisez repr()}
        \PY{n}{point}
\end{Verbatim}


    \begin{Verbatim}[commandchars=\\\{\}]
avec print Point(0,100)

    \end{Verbatim}

\begin{Verbatim}[commandchars=\\\{\}]
{\color{outcolor}Out[{\color{outcolor}2}]:} Point(0,100)
\end{Verbatim}
            
    \begin{Verbatim}[commandchars=\\\{\}]
{\color{incolor}In [{\color{incolor}3}]:} \PY{c+c1}{\PYZsh{} la même chose mais où on redéfinit \PYZus{}\PYZus{}str\PYZus{}\PYZus{} et \PYZus{}\PYZus{}repr\PYZus{}\PYZus{}}
        \PY{k}{class} \PY{n+nc}{Point2}\PY{p}{:}
            \PY{l+s+s2}{\PYZdq{}}\PY{l+s+s2}{seconde version de Point \PYZhy{} on définit \PYZus{}\PYZus{}repr\PYZus{}\PYZus{} et \PYZus{}\PYZus{}str\PYZus{}\PYZus{}}\PY{l+s+s2}{\PYZdq{}}
            \PY{k}{def} \PY{n+nf}{\PYZus{}\PYZus{}init\PYZus{}\PYZus{}}\PY{p}{(}\PY{n+nb+bp}{self}\PY{p}{,} \PY{n}{x}\PY{p}{,} \PY{n}{y}\PY{p}{)}\PY{p}{:}
                \PY{n+nb+bp}{self}\PY{o}{.}\PY{n}{x} \PY{o}{=} \PY{n}{x}
                \PY{n+nb+bp}{self}\PY{o}{.}\PY{n}{y} \PY{o}{=} \PY{n}{y}
            \PY{k}{def} \PY{n+nf}{\PYZus{}\PYZus{}repr\PYZus{}\PYZus{}}\PY{p}{(}\PY{n+nb+bp}{self}\PY{p}{)}\PY{p}{:}
                \PY{k}{return} \PY{n}{f}\PY{l+s+s2}{\PYZdq{}}\PY{l+s+s2}{Point2(}\PY{l+s+si}{\PYZob{}self.x\PYZcb{}}\PY{l+s+s2}{,}\PY{l+s+si}{\PYZob{}self.y\PYZcb{}}\PY{l+s+s2}{)}\PY{l+s+s2}{\PYZdq{}}
            \PY{k}{def} \PY{n+nf}{\PYZus{}\PYZus{}str\PYZus{}\PYZus{}}\PY{p}{(}\PY{n+nb+bp}{self}\PY{p}{)}\PY{p}{:}
                \PY{k}{return} \PY{n}{f}\PY{l+s+s2}{\PYZdq{}}\PY{l+s+s2}{(}\PY{l+s+si}{\PYZob{}self.x\PYZcb{}}\PY{l+s+s2}{,}\PY{l+s+si}{\PYZob{}self.y\PYZcb{}}\PY{l+s+s2}{)}\PY{l+s+s2}{\PYZdq{}}
            
        \PY{n}{point2} \PY{o}{=} \PY{n}{Point2} \PY{p}{(}\PY{l+m+mi}{0}\PY{p}{,}\PY{l+m+mi}{100}\PY{p}{)}
        
        \PY{n+nb}{print}\PY{p}{(}\PY{l+s+s2}{\PYZdq{}}\PY{l+s+s2}{avec print}\PY{l+s+s2}{\PYZdq{}}\PY{p}{,} \PY{n}{point2}\PY{p}{)}
        
        \PY{c+c1}{\PYZsh{} les f\PYZhy{}strings (ou format) utilisent aussi \PYZus{}\PYZus{}str\PYZus{}\PYZus{}}
        \PY{n+nb}{print}\PY{p}{(}\PY{n}{f}\PY{l+s+s2}{\PYZdq{}}\PY{l+s+s2}{avec format }\PY{l+s+si}{\PYZob{}point2\PYZcb{}}\PY{l+s+s2}{\PYZdq{}}\PY{p}{)}
        
        \PY{c+c1}{\PYZsh{} et si enfin vous affichez un objet sans passer par print}
        \PY{c+c1}{\PYZsh{} vous utilisez repr()}
        \PY{n}{point2}
\end{Verbatim}


    \begin{Verbatim}[commandchars=\\\{\}]
avec print (0,100)
avec format (0,100)

    \end{Verbatim}

\begin{Verbatim}[commandchars=\\\{\}]
{\color{outcolor}Out[{\color{outcolor}3}]:} Point2(0,100)
\end{Verbatim}
            
    \begin{center}\rule{0.5\linewidth}{\linethickness}\end{center}

    \hypertarget{bool__}{%
\subsubsection{\texorpdfstring{\texttt{\_\_bool\_\_}}{\_\_bool\_\_}}\label{bool__}}

    Vous vous souvenez que la condition d'un test dans un \texttt{if} peut
ne pas retourner un booléen (nous avons vu cela en Semaine 4, Séquence
``Test if/elif/else et opérateurs booléens''). Nous avions noté que pour
les types prédéfinis, sont considérés comme \emph{faux} les objets~:
\texttt{None}, la liste vide, un tuple vide, etc.

Avec \texttt{\_\_bool\_\_} on peut redéfinir le comportement des objets
d'une classe vis-à-vis des conditions - ou si l'on préfère, quel doit
être le résultat de \texttt{bool(instance)}.

\textbf{Attention} pour éviter les comportements imprévus, comme on est
en train de redéfinir le comportement des conditions, il \textbf{faut}
renvoyer un \textbf{booléen} (ou à la rigueur 0 ou 1), on ne peut pas
dans ce contexte retourner d'autres types d'objet.

Nous allons \textbf{illustrer} cette méthode dans un petit moment avec
une nouvelle implémentation de la classe \textbf{\texttt{Matrix2}}.

Remarquez enfin qu'en l'absence de méthode \texttt{\_\_bool\_\_}, on
cherche aussi la méthode \texttt{\_\_len\_\_} pour déterminer le
résultat du test~; une instance de longueur nulle est alors considéré
comme \texttt{False}, en cohérence avec ce qui se passe avec les types
\emph{built-in} \texttt{list}, \texttt{dict}, \texttt{tuple}, etc.

Ce genre de \emph{protocole}, qui cherche d'abord une méthode
(\texttt{\_\_bool\_\_}), puis une autre (\texttt{\_\_len\_\_}) en cas
d'absence de la première, est relativement fréquent dans la mécanique de
surcharge des opérateurs~; c'est entre autres pourquoi la documentation
est indispensable lorsqu'on surcharge les opérateurs.

    \hypertarget{add__-et-apparentuxe9s-__mul__-__sub__-__div__-__and__-etc.}{%
\subsubsection{\texorpdfstring{\texttt{\_\_add\_\_} et apparentés
(\texttt{\_\_mul\_\_}, \texttt{\_\_sub\_\_}, \texttt{\_\_div\_\_},
\texttt{\_\_and\_\_},
etc.)}{\_\_add\_\_ et apparentés (\_\_mul\_\_, \_\_sub\_\_, \_\_div\_\_, \_\_and\_\_, etc.)}}\label{add__-et-apparentuxe9s-__mul__-__sub__-__div__-__and__-etc.}}

    On peut également redéfinir les opérateurs arithmétiques et logiques.
Dans l'exemple qui suit, nous allons l'illustrer sur l'addition de
matrices. On rappelle pour mémoire que~:

\(\left( \begin{array}{cc} a_{11} & a_{12} \\ a_{21} & a_{22}\end{array} \right) + \left( \begin{array}{cc} b_{11} & b_{12} \\ b_{21} & b_{22}\end{array} \right) = \left( \begin{array}{cc} a_{11}+b_{11} & a_{12}+b_{12} \\ a_{21}+b_{21} & a_{22}+b_{22}\end{array} \right)\)

    \hypertarget{une-nouvelle-version-de-la-classe-matrix2}{%
\subsubsection{\texorpdfstring{Une nouvelle version de la classe
\texttt{Matrix2}}{Une nouvelle version de la classe Matrix2}}\label{une-nouvelle-version-de-la-classe-matrix2}}

    Voici (encore) une nouvelle implémentation de la classe de matrices 2x2,
qui illustre cette fois~:

\begin{itemize}
\tightlist
\item
  la possibilité d'ajouter deux matrices~;
\item
  la possibilité de faire un test sur une matrice - le test sera faux si
  la matrice a tous ses coefficients nuls~;
\item
  et, bien que ce ne soit pas le sujet immédiat, cette implémentation
  illustre aussi la possibilité de construire la matrice à partir~:

  \begin{itemize}
  \tightlist
  \item
    soit des 4 coefficients, comme par exemple~:
    \texttt{Matrix2(a,\ b,\ c,\ d)}
  \item
    soit d'une séquence, comme par exemple~: \texttt{Matrix2(range(4))}
  \end{itemize}
\end{itemize}

Cette dernière possibilité va nous permettre de simplifier le code de
l'addition, comme on va le voir.

    \begin{Verbatim}[commandchars=\\\{\}]
{\color{incolor}In [{\color{incolor}4}]:} \PY{c+c1}{\PYZsh{} notre classe Matrix2 avec encore une autre implémentation}
        \PY{k}{class} \PY{n+nc}{Matrix2}\PY{p}{:}
        
            \PY{k}{def} \PY{n+nf}{\PYZus{}\PYZus{}init\PYZus{}\PYZus{}}\PY{p}{(}\PY{n+nb+bp}{self}\PY{p}{,} \PY{o}{*}\PY{n}{args}\PY{p}{)}\PY{p}{:}
                \PY{l+s+sd}{\PYZdq{}\PYZdq{}\PYZdq{}}
        \PY{l+s+sd}{        le constructeur accepte }
        \PY{l+s+sd}{        (*) soit les 4 coefficients individuellement}
        \PY{l+s+sd}{        (*) soit une liste \PYZhy{} ou + généralement une séquence \PYZhy{} des mêmes}
        \PY{l+s+sd}{        \PYZdq{}\PYZdq{}\PYZdq{}}
                \PY{c+c1}{\PYZsh{} on veut pouvoir créer l\PYZsq{}objet à partir des 4 coefficients}
                \PY{c+c1}{\PYZsh{} souvenez\PYZhy{}vous qu\PYZsq{}avec la forme *args, args est toujours un tuple}
                \PY{k}{if} \PY{n+nb}{len}\PY{p}{(}\PY{n}{args}\PY{p}{)} \PY{o}{==} \PY{l+m+mi}{4}\PY{p}{:}
                    \PY{n+nb+bp}{self}\PY{o}{.}\PY{n}{coefs} \PY{o}{=} \PY{n}{args}
                \PY{c+c1}{\PYZsh{} ou bien d\PYZsq{}une séquence de 4 coefficients}
                \PY{k}{elif} \PY{n+nb}{len}\PY{p}{(}\PY{n}{args}\PY{p}{)} \PY{o}{==} \PY{l+m+mi}{1}\PY{p}{:}
                    \PY{n+nb+bp}{self}\PY{o}{.}\PY{n}{coefs} \PY{o}{=} \PY{n+nb}{tuple}\PY{p}{(}\PY{o}{*}\PY{n}{args}\PY{p}{)}
        
            \PY{k}{def} \PY{n+nf}{\PYZus{}\PYZus{}repr\PYZus{}\PYZus{}}\PY{p}{(}\PY{n+nb+bp}{self}\PY{p}{)}\PY{p}{:}
                \PY{l+s+s2}{\PYZdq{}}\PY{l+s+s2}{l}\PY{l+s+s2}{\PYZsq{}}\PY{l+s+s2}{affichage}\PY{l+s+s2}{\PYZdq{}}
                \PY{k}{return} \PY{l+s+s2}{\PYZdq{}}\PY{l+s+s2}{[}\PY{l+s+s2}{\PYZdq{}} \PY{o}{+} \PY{l+s+s2}{\PYZdq{}}\PY{l+s+s2}{, }\PY{l+s+s2}{\PYZdq{}}\PY{o}{.}\PY{n}{join}\PY{p}{(}\PY{p}{[}\PY{n+nb}{str}\PY{p}{(}\PY{n}{c}\PY{p}{)} \PY{k}{for} \PY{n}{c} \PY{o+ow}{in} \PY{n+nb+bp}{self}\PY{o}{.}\PY{n}{coefs}\PY{p}{]}\PY{p}{)} \PY{o}{+} \PY{l+s+s2}{\PYZdq{}}\PY{l+s+s2}{]}\PY{l+s+s2}{\PYZdq{}}
        
            \PY{k}{def} \PY{n+nf}{\PYZus{}\PYZus{}add\PYZus{}\PYZus{}}\PY{p}{(}\PY{n+nb+bp}{self}\PY{p}{,} \PY{n}{other}\PY{p}{)}\PY{p}{:}
                \PY{l+s+sd}{\PYZdq{}\PYZdq{}\PYZdq{}}
        \PY{l+s+sd}{        l\PYZsq{}addition de deux matrices retourne un nouvel objet}
        \PY{l+s+sd}{        la possibilité de créer une matrice à partir}
        \PY{l+s+sd}{        d\PYZsq{}une liste rend ce code beaucoup plus facile à écrire}
        \PY{l+s+sd}{        \PYZdq{}\PYZdq{}\PYZdq{}}
                \PY{k}{return} \PY{n}{Matrix2}\PY{p}{(}\PY{p}{[}\PY{n}{a} \PY{o}{+} \PY{n}{b} \PY{k}{for} \PY{n}{a}\PY{p}{,} \PY{n}{b} \PY{o+ow}{in} \PY{n+nb}{zip}\PY{p}{(}\PY{n+nb+bp}{self}\PY{o}{.}\PY{n}{coefs}\PY{p}{,} \PY{n}{other}\PY{o}{.}\PY{n}{coefs}\PY{p}{)}\PY{p}{]}\PY{p}{)}
        
            \PY{k}{def} \PY{n+nf}{\PYZus{}\PYZus{}bool\PYZus{}\PYZus{}}\PY{p}{(}\PY{n+nb+bp}{self}\PY{p}{)}\PY{p}{:}
                \PY{l+s+sd}{\PYZdq{}\PYZdq{}\PYZdq{}}
        \PY{l+s+sd}{        on considère que la matrice est non nulle }
        \PY{l+s+sd}{        si un au moins de ses coefficients est non nul}
        \PY{l+s+sd}{        \PYZdq{}\PYZdq{}\PYZdq{}}
                \PY{c+c1}{\PYZsh{} ATTENTION le retour doit être un booléen }
                \PY{c+c1}{\PYZsh{} ou à la rigueur 0 ou 1}
                \PY{k}{for} \PY{n}{c} \PY{o+ow}{in} \PY{n+nb+bp}{self}\PY{o}{.}\PY{n}{coefs}\PY{p}{:}
                    \PY{k}{if} \PY{n}{c}\PY{p}{:}
                        \PY{k}{return} \PY{k+kc}{True}
                \PY{k}{return} \PY{k+kc}{False}
\end{Verbatim}


    On peut à présent créer deux objets, les ajouter, et vérifier que la
matrice nulle se comporte bien comme attendu~:

    \begin{Verbatim}[commandchars=\\\{\}]
{\color{incolor}In [{\color{incolor}5}]:} \PY{n}{zero}     \PY{o}{=} \PY{n}{Matrix2} \PY{p}{(}\PY{p}{[}\PY{l+m+mi}{0}\PY{p}{,}\PY{l+m+mi}{0}\PY{p}{,}\PY{l+m+mi}{0}\PY{p}{,}\PY{l+m+mi}{0}\PY{p}{]}\PY{p}{)}
        
        \PY{n}{matrice1} \PY{o}{=} \PY{n}{Matrix2} \PY{p}{(}\PY{l+m+mi}{1}\PY{p}{,}\PY{l+m+mi}{2}\PY{p}{,}\PY{l+m+mi}{3}\PY{p}{,}\PY{l+m+mi}{4}\PY{p}{)}
        \PY{n}{matrice2} \PY{o}{=} \PY{n}{Matrix2} \PY{p}{(}\PY{n+nb}{list}\PY{p}{(}\PY{n+nb}{range}\PY{p}{(}\PY{l+m+mi}{10}\PY{p}{,}\PY{l+m+mi}{50}\PY{p}{,}\PY{l+m+mi}{10}\PY{p}{)}\PY{p}{)}\PY{p}{)}
        
        \PY{n+nb}{print}\PY{p}{(}\PY{l+s+s1}{\PYZsq{}}\PY{l+s+s1}{avant matrice1}\PY{l+s+s1}{\PYZsq{}}\PY{p}{,} \PY{n}{matrice1}\PY{p}{)}
        \PY{n+nb}{print}\PY{p}{(}\PY{l+s+s1}{\PYZsq{}}\PY{l+s+s1}{avant matrice2}\PY{l+s+s1}{\PYZsq{}}\PY{p}{,} \PY{n}{matrice2}\PY{p}{)}
        
        \PY{n+nb}{print}\PY{p}{(}\PY{l+s+s1}{\PYZsq{}}\PY{l+s+s1}{somme}\PY{l+s+s1}{\PYZsq{}}\PY{p}{,} \PY{n}{matrice1} \PY{o}{+} \PY{n}{matrice2}\PY{p}{)}
        
        \PY{n+nb}{print}\PY{p}{(}\PY{l+s+s1}{\PYZsq{}}\PY{l+s+s1}{après matrice1}\PY{l+s+s1}{\PYZsq{}}\PY{p}{,} \PY{n}{matrice1}\PY{p}{)}
        \PY{n+nb}{print}\PY{p}{(}\PY{l+s+s1}{\PYZsq{}}\PY{l+s+s1}{après matrice2}\PY{l+s+s1}{\PYZsq{}}\PY{p}{,} \PY{n}{matrice2}\PY{p}{)}
        
        \PY{k}{if} \PY{n}{matrice1}\PY{p}{:} 
            \PY{n+nb}{print}\PY{p}{(}\PY{n}{matrice1}\PY{p}{,}\PY{l+s+s2}{\PYZdq{}}\PY{l+s+s2}{n}\PY{l+s+s2}{\PYZsq{}}\PY{l+s+s2}{est pas nulle}\PY{l+s+s2}{\PYZdq{}}\PY{p}{)}
        \PY{k}{if} \PY{o+ow}{not} \PY{n}{zero}\PY{p}{:} 
            \PY{n+nb}{print}\PY{p}{(}\PY{n}{zero}\PY{p}{,}\PY{l+s+s2}{\PYZdq{}}\PY{l+s+s2}{est nulle}\PY{l+s+s2}{\PYZdq{}}\PY{p}{)}
\end{Verbatim}


    \begin{Verbatim}[commandchars=\\\{\}]
avant matrice1 [1, 2, 3, 4]
avant matrice2 [10, 20, 30, 40]
somme [11, 22, 33, 44]
après matrice1 [1, 2, 3, 4]
après matrice2 [10, 20, 30, 40]
[1, 2, 3, 4] n'est pas nulle
[0, 0, 0, 0] est nulle

    \end{Verbatim}

    Voici en vrac quelques commentaires sur cet exemple.

    \hypertarget{utiliser-un-tuple}{%
\subparagraph{Utiliser un tuple}\label{utiliser-un-tuple}}

    Avant de parler de la surcharge des opérateurs \emph{per se}, vous
remarquerez que l'on range les coefficients dans un \textbf{tuple}, de
façon à ce que notre objet \texttt{Matrix2} soit indépendant de l'objet
qu'on a utilisé pour le créer (et qui peut être ensuite modifié par
l'appelant).

    \hypertarget{cruxe9er-un-nouvel-objet}{%
\subparagraph{Créer un nouvel objet}\label{cruxe9er-un-nouvel-objet}}

    Vous remarquez que l'addition \texttt{\_\_add\_\_} renvoie un
\textbf{nouvel objet}, au lieu de modifier \texttt{self} en place. C'est
la bonne façon de procéder tout simplement parce que lorsqu'on écrit~:

\begin{verbatim}
print('somme', matrice1 + matrice2)
\end{verbatim}

on ne s'attend pas du tout à ce que \texttt{matrice1} soit modifiée
après cet appel.

    \hypertarget{du-code-qui-ne-duxe9pend-que-des-4-opuxe9rations}{%
\subparagraph{Du code qui ne dépend que des 4
opérations}\label{du-code-qui-ne-duxe9pend-que-des-4-opuxe9rations}}

    Le fait d'avoir défini l'addition nous permet par exemple de bénéficier
de la fonction \emph{built-in} \texttt{sum}. En effet le code de
\texttt{sum} fait lui-même des additions, il n'y a donc aucune raison de
ne pas pouvoir l'exécuter avec en entrée une liste de matrices puisque
maintenant on sait les additionner, (mais on a dû toutefois passer à
\texttt{sum} comme élément neutre \texttt{zero})~:

    \begin{Verbatim}[commandchars=\\\{\}]
{\color{incolor}In [{\color{incolor}6}]:} \PY{n+nb}{sum}\PY{p}{(}\PY{p}{[}\PY{n}{matrice1}\PY{p}{,} \PY{n}{matrice2}\PY{p}{,} \PY{n}{matrice1}\PY{p}{]} \PY{p}{,} \PY{n}{zero}\PY{p}{)}
\end{Verbatim}


\begin{Verbatim}[commandchars=\\\{\}]
{\color{outcolor}Out[{\color{outcolor}6}]:} [12, 24, 36, 48]
\end{Verbatim}
            
    C'est un effet de bord du typage dynamique. On ne vérifie pas \emph{a
priori} que tous les arguments passés à \texttt{sum} savent faire une
addition~; \emph{a contrario}, s'ils savent s'additionner on peut
exécuter le code de \texttt{sum}.

De manière plus générale, si vous écrivez par exemple un morceau de code
qui travaille sur les éléments d'un anneau (au sens anneau des entiers
\(\mathbb{Z}\)) - imaginez un code qui factorise des polynômes - vous
pouvez espérer utiliser ce code avec n'importe quel anneau, c'est à dire
avec une classe qui implémente les 4 opérations (pourvu bien sûr que cet
ensemble soit effectivement un anneau).

    \hypertarget{on-peut-aussi-reduxe9finir-un-ordre}{%
\subparagraph{On peut aussi redéfinir un
ordre}\label{on-peut-aussi-reduxe9finir-un-ordre}}

    La place nous manque pour illustrer la possibilité, avec les opérateurs
\texttt{\_\_eq\_\_}, \texttt{\_\_ne\_\_}, \texttt{\_\_lt\_\_},
\texttt{\_\_le\_\_}, \texttt{\_\_gt\_\_}, et \texttt{\_\_ge\_\_}, de
redéfinir un ordre sur les instances d'une classe.

Signalons à cet égard qu'il existe un mécanisme ``intelligent'' qui
permet de définir un ordre à partir d'un sous-ensemble seulement de ces
méthodes, l'idée étant que si vous savez faire \texttt{\textgreater{}}
et \texttt{=}, vous savez sûrement faire tout le reste. Ce mécanisme est
\href{https://docs.python.org/3/library/functools.html\#functools.total_ordering}{documenté
ici}~; il repose sur \textbf{un décorateur} (\texttt{@total\_ordering}),
un mécanisme que nous étudierons en semaine 9, mais que vous pouvez
utiliser dès à présent.

    De manière analogue à \texttt{sum} qui fonctionne sur une liste de
matrices, si on avait défini un ordre sur les matrices, on aurait pu
alors utiliser les \emph{built-in} \texttt{min} et \texttt{max} pour
calculer une borne supérieure ou inférieure dans une séquence de
matrices.

    \hypertarget{compluxe9ment---niveau-avancuxe9}{%
\subsection{Complément - niveau
avancé}\label{compluxe9ment---niveau-avancuxe9}}

    \hypertarget{le-produit-avec-un-scalaire}{%
\subparagraph{Le produit avec un
scalaire}\label{le-produit-avec-un-scalaire}}

    On implémenterait la multiplication de deux matrices d'une façon
identique (quoique plus fastidieuse naturellement).

La multiplication d'une matrice par un scalaire (un réel ou complexe
pour fixer les idées), comme ici~:

\begin{verbatim}
matrice2 = reel * matrice1
\end{verbatim}

peut être également réalisée par surcharge de l'opérateur
\texttt{\_\_rmul\_\_}.

Il s'agit d'une astuce, destinée précisément à ce genre de situations,
où on veut étendre la classe de l'opérande de \textbf{droite}, sachant
que dans ce cas précis l'opérande de gauche est un type de base, qu'on
ne peut pas étendre (les classes \emph{built-in} sont non mutables, pour
garantir la stabilité de l'interpréteur).

Voici donc comment on s'y prendrait. Pour éviter de reproduire tout le
code de la classe, on va l'étendre à la volée.

    \begin{Verbatim}[commandchars=\\\{\}]
{\color{incolor}In [{\color{incolor}7}]:} \PY{c+c1}{\PYZsh{} remarquez que les opérandes sont apparemment inversés}
        \PY{c+c1}{\PYZsh{} dans le sens où pour evaluer }
        \PY{c+c1}{\PYZsh{}     reel * matrice}
        \PY{c+c1}{\PYZsh{} on écrit une méthode qui prend en argument}
        \PY{c+c1}{\PYZsh{}   la matrice, puis le réel}
        \PY{c+c1}{\PYZsh{} mais n\PYZsq{}oubliez pas qu\PYZsq{}on est en fait en train}
        \PY{c+c1}{\PYZsh{} d\PYZsq{}écrire une méthode sur la classe `Matrix2`}
        \PY{k}{def} \PY{n+nf}{multiplication\PYZus{}scalaire}\PY{p}{(}\PY{n+nb+bp}{self}\PY{p}{,} \PY{n}{alpha}\PY{p}{)}\PY{p}{:}
            \PY{k}{return} \PY{n}{Matrix2}\PY{p}{(}\PY{p}{[}\PY{n}{alpha} \PY{o}{*} \PY{n}{coef} \PY{k}{for} \PY{n}{coef} \PY{o+ow}{in} \PY{n+nb+bp}{self}\PY{o}{.}\PY{n}{coefs}\PY{p}{]}\PY{p}{)}
        
        \PY{c+c1}{\PYZsh{} on ajoute la méthode spéciale \PYZus{}\PYZus{}rmul\PYZus{}\PYZus{}}
        \PY{n}{Matrix2}\PY{o}{.}\PY{n+nf+fm}{\PYZus{}\PYZus{}rmul\PYZus{}\PYZus{}} \PY{o}{=} \PY{n}{multiplication\PYZus{}scalaire}
\end{Verbatim}


    \begin{Verbatim}[commandchars=\\\{\}]
{\color{incolor}In [{\color{incolor}8}]:} \PY{n}{matrice1}
\end{Verbatim}


\begin{Verbatim}[commandchars=\\\{\}]
{\color{outcolor}Out[{\color{outcolor}8}]:} [1, 2, 3, 4]
\end{Verbatim}
            
    \begin{Verbatim}[commandchars=\\\{\}]
{\color{incolor}In [{\color{incolor}9}]:} \PY{l+m+mi}{12} \PY{o}{*} \PY{n}{matrice1}
\end{Verbatim}


\begin{Verbatim}[commandchars=\\\{\}]
{\color{outcolor}Out[{\color{outcolor}9}]:} [12, 24, 36, 48]
\end{Verbatim}
            

    % Add a bibliography block to the postdoc
    
    
    
