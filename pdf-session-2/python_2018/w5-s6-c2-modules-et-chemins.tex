    
    
    
    

    

    \hypertarget{ouxf9-sont-cherchuxe9s-les-modules}{%
\section{Où sont cherchés les
modules~?}\label{ouxf9-sont-cherchuxe9s-les-modules}}

    \hypertarget{compluxe9ment---niveau-basique}{%
\subsection{Complément - niveau
basique}\label{compluxe9ment---niveau-basique}}

    Pour les débutants en informatique, le plus simple est de se souvenir
que si vous voulez uniquement charger vos propres modules ou packages,
il suffit de les placer dans le répertoire où vous lancez la commande
python. Si vous n'êtes pas sûr de cet emplacement vous pouvez le savoir
en faisant~:

    \begin{Verbatim}[commandchars=\\\{\}]
{\color{incolor}In [{\color{incolor}1}]:} \PY{k+kn}{from} \PY{n+nn}{pathlib} \PY{k}{import} \PY{n}{Path}
        \PY{n}{Path}\PY{o}{.}\PY{n}{cwd}\PY{p}{(}\PY{p}{)}
\end{Verbatim}


\begin{Verbatim}[commandchars=\\\{\}]
{\color{outcolor}Out[{\color{outcolor}1}]:} PosixPath('/Users/tparment/git/flotpython')
\end{Verbatim}
            
    \hypertarget{compluxe9ment---niveau-intermuxe9diaire}{%
\subsection{Complément - niveau
intermédiaire}\label{compluxe9ment---niveau-intermuxe9diaire}}

    Dans ce complément nous allons voir, de manière générale, comment sont
localisés (sur le disque dur) les modules que vous chargez dans python
grâce à l'instruction \texttt{import}~; nous verrons aussi où placer vos
propres fichiers pour qu'ils soient accessibles à python.

    \href{https://docs.python.org/3/tutorial/modules.html\#the-module-search-path}{Comme
expliqué ici}, lorsque vous importez le module \texttt{spam}, python
cherche dans cet ordre~:

\begin{itemize}
\tightlist
\item
  un module \emph{built-in} de nom \texttt{spam} -
  possiblement/probablement écrit en C,
\item
  ou sinon un fichier \texttt{spam.py} (ou \texttt{spam/\_\_init.py\_\_}
  s'il s'agit d'un package)~; pour le localiser on utilise la variable
  \texttt{sys.path} (c'est-à-dire l'attribut \texttt{path} dans le
  module \texttt{sys}), qui est une liste de répertoires, et qui est
  initialisée avec, dans cet ordre~:

  \begin{itemize}
  \tightlist
  \item
    le répertoire où se trouve le point d'entrée~;
  \item
    la variable d'environnement \texttt{PYTHONPATH}~;
  \item
    un certain nombre d'emplacements définis au moment de la compilation
    de python.
  \end{itemize}
\end{itemize}

    Ainsi sans action particulière de l'utilisateur, python trouve
l'intégralité de la librairie standard, ainsi que les modules et
packages installés dans le même répertoire que le fichier passé à
l'interpréteur.

    La façon dont cela se présente dans l'interpréteur des notebooks~peut
vous induire en erreur. Aussi je vous engage à exécuter plutôt, et sur
votre machine, le programme suivant~:

    \begin{verbatim}
#!/usr/bin/env python3

import sys
from pathlib import Path

def show_argv_and_path():
    print(f"le répertoire courant est {Path.cwd()}")
    print(f"le point d'entrée du programme est {sys.argv[0]}")
    print(f"la variable sys.path contient")
    for i, path in enumerate(sys.path, 1):
        print(f"{i}-ème chemin dans sys.path {path}")

show_argv_and_path()
\end{verbatim}

    En admettant que

\begin{itemize}
\tightlist
\item
  vous rangez ceci le fichier \texttt{/le/repertoire/du/script/run.py}
\item
  et que vous lancez Python depuis un répertoire différent, disons
  \texttt{/le/repertoire/ou/vous/etes}
\item
  et avec une variable \texttt{PYTHONPATH} vide;
\end{itemize}

alors vous devriez observer ceci~:

    \begin{verbatim}
$ cd /le/repertoire/ou/vous/etes/
/le/repertoire/ou/vous/etes $ python3 /le/repertoire/du/script/run.py
le répertoire courant est /le/repertoire/ou/vous/etes
le point d'entrée du programme est /le/repertoire/du/script/run.py
la variable sys.path contient
1-ème chemin dans sys.path /le/repertoire/du/script
<snip>
\end{verbatim}

\emph{le reste dépend de votre installation}

    C'est-à-dire que~:

\begin{itemize}
\tightlist
\item
  la variable \texttt{sys.argv{[}0{]}} contient le chemin complet
  \texttt{/le/repertoire/du/script/run.py},
\item
  et le premier terme dans \texttt{sys.path} contient
  \texttt{/le/repertoire/du/script!}.
\end{itemize}

    La \href{http://en.wikipedia.org/wiki/Environment_variable}{variable
d'environnement} PYTHONPATH est définie de façon à donner la possibilité
d'étendre ces listes depuis l'extérieur, et sans recompiler
l'interpréteur, ni modifier les sources. Cette possibilité s'adresse
donc à l'utilisateur final - ou à son administrateur système - plutôt
qu'au programmeur.

    En tant que programmeur par contre, vous avez la possibilité d'étendre
\texttt{sys.path} avant de faire vos \texttt{import}.

Imaginons par exemple que vous avez écrit un petit outil utilitaire qui
se compose d'un point d'entrée \texttt{main.py}, et de plusieurs modules
\texttt{spam.py} et \texttt{eggs.py}. Vous n'avez pas le temps de
packager proprement cet outil, vous voudriez pouvoir distribuer un
\emph{tar} avec ces trois fichiers python, qui puissent s'installer
n'importe où (pourvu qu'ils soient tous les trois au même endroit), et
que le point d'entrée trouve ses deux modules sans que l'utilisateur ait
à s'en soucier.

Imaginons donc ces trois fichiers installés sur machine de l'utilisateur
dans~:

    \begin{verbatim}
/usr/share/utilitaire/
                      main.py
                      spam.py
                      eggs.py
\end{verbatim}

    Si vous ne faites rien de particulier, c'est-à-dire que \texttt{main.py}
contient juste

    \begin{verbatim}
import spam, eggs
\end{verbatim}

    Alors le programme ne fonctionnera \textbf{que s'il est lancé depuis
\texttt{/usr/share/utilitaire}}, ce qui n'est pas du tout pratique.

    Pour contourner cela on peut écrire dans \texttt{main.py} quelque chose
comme~:

    \begin{verbatim}
# on récupère le répertoire où est installé le point d'entrée
from pathlib import Path

directory_installation = Path(__file__).parent

# et on l'ajoute au chemin de recherche des modules
import sys
sys.path.append(directory_installation)

# maintenant on peut importer spam et eggs de n'importe où
import spam, eggs
\end{verbatim}

    \hypertarget{distribuer-sa-propre-librairie-avec-setuptools}{%
\subsubsection{\texorpdfstring{Distribuer sa propre librairie avec
\texttt{setuptools}}{Distribuer sa propre librairie avec setuptools}}\label{distribuer-sa-propre-librairie-avec-setuptools}}

    Notez bien que l'exemple précédent est \textbf{uniquement donné à titre
d'illustration} pour décortiquer la mécanique d'utilisation de
\texttt{sys.path}.

Ce n'est pas une technique recommandée dans le cas général. On préfère
en effet de beaucoup diffuser une application python, ou une librairie,
sous forme de packaging en utilisant le
\href{https://pypi.python.org/pypi/setuptools}{module setuptools}. Il
s'agit d'un outil qui \textbf{ne fait pas partie de la librairie
standard}, et qui supplante \texttt{distutils} qui lui, fait partie de
la distribution standard mais qui est tombé en déshérence au fil du
temps.

    \texttt{setuptools} permet au programmeur d'écrire - dans un fichier
qu'on appelle traditionnellement \texttt{setup.py} - le contenu de son
application~; grâce à quoi on peut ensuite de manière unifiée~:

\begin{itemize}
\tightlist
\item
  installer l'application sur une machine à partir des sources~;
\item
  préparer un package de l'application~;
\item
  diffuser le package dans
  \href{https://pypi.python.org/pypi}{l'infrastructure PyPI}~;
\item
  installer le package depuis PyPI en utilisant
  \href{http://pip.readthedocs.org/en/latest/installing.html}{\texttt{pip3}}.
\end{itemize}

    Pour installer \texttt{setuptools}, comme d'habitude vous pouvez faire
simplement~:

\begin{verbatim}
pip3 install setuptools
\end{verbatim}


    % Add a bibliography block to the postdoc
    
    
    
