    
    
    
    

    

    \hypertarget{pruxe9cisions-sur-limportation}{%
\section{Précisions sur
l'importation}\label{pruxe9cisions-sur-limportation}}

    \hypertarget{compluxe9ment---niveau-basique}{%
\subsection{Complément - niveau
basique}\label{compluxe9ment---niveau-basique}}

    \hypertarget{importations-multiples---rechargement}{%
\subsubsection{Importations multiples -
rechargement}\label{importations-multiples---rechargement}}

    \hypertarget{un-module-nest-charguxe9-quune-fois}{%
\subparagraph{Un module n'est chargé qu'une
fois}\label{un-module-nest-charguxe9-quune-fois}}

    De manière générale, à l'intérieur d'un interpréteur python, un module
donné n'est chargé qu'une seule fois. L'idée est naturellement que si
plusieurs modules différents importent le même module, (ou si un même
module en importe un autre plusieurs fois) on ne paie le prix du
chargement du module qu'une seule fois.

Voyons cela sur un exemple simpliste, importons un module pour la
première fois~:

    \begin{Verbatim}[commandchars=\\\{\}]
{\color{incolor}In [{\color{incolor}1}]:} \PY{k+kn}{import} \PY{n+nn}{multiple\PYZus{}import}
\end{Verbatim}


    \begin{Verbatim}[commandchars=\\\{\}]
chargement de multiple\_import

    \end{Verbatim}

    Ce module est très simple, comme vous pouvez le voir

    \begin{Verbatim}[commandchars=\\\{\}]
{\color{incolor}In [{\color{incolor}2}]:} \PY{k+kn}{from} \PY{n+nn}{modtools} \PY{k}{import} \PY{n}{show\PYZus{}module}
        \PY{n}{show\PYZus{}module}\PY{p}{(}\PY{n}{multiple\PYZus{}import}\PY{p}{)}
\end{Verbatim}


    \begin{Verbatim}[commandchars=\\\{\}]
Fichier /Users/tparment/git/flotpython/modules/multiple\_import.py
----------------------------------------
1|"""
2|Ce module est conçu pour illustrer le mécanisme de
3|chargement / rechargement 
4|"""
5|
6|print("chargement de", \_\_name\_\_)

    \end{Verbatim}

    Si on le charge une deuxième fois (peu importe où, dans le même module,
un autre module, une fonction..), vous remarquez qu'il ne produit aucune
impression

    \begin{Verbatim}[commandchars=\\\{\}]
{\color{incolor}In [{\color{incolor}3}]:} \PY{k+kn}{import} \PY{n+nn}{multiple\PYZus{}import} 
\end{Verbatim}


    Ce qui confirme que le module a déjà été chargé, donc cette instruction
\texttt{import} n'a aucun effet autre qu'affecter la variable
\texttt{multiple\_import} de nouveau à l'objet module déjà chargé. En
résumé, l'instruction \texttt{import} fait l'opération d'affectation
autant de fois qu'on appelle \texttt{import}, mais elle ne charge le
module qu'une seule fois à la première importation.

    Une autre façon d'illustrer ce trait est d'importer plusieurs fois le
module \texttt{this}

    \begin{Verbatim}[commandchars=\\\{\}]
{\color{incolor}In [{\color{incolor}4}]:} \PY{c+c1}{\PYZsh{} la première fois le chargement a vraiment lieu}
        \PY{k+kn}{import} \PY{n+nn}{this}
\end{Verbatim}


    \begin{Verbatim}[commandchars=\\\{\}]
The Zen of Python, by Tim Peters

Beautiful is better than ugly.
Explicit is better than implicit.
Simple is better than complex.
Complex is better than complicated.
Flat is better than nested.
Sparse is better than dense.
Readability counts.
Special cases aren't special enough to break the rules.
Although practicality beats purity.
Errors should never pass silently.
Unless explicitly silenced.
In the face of ambiguity, refuse the temptation to guess.
There should be one-- and preferably only one --obvious way to do it.
Although that way may not be obvious at first unless you're Dutch.
Now is better than never.
Although never is often better than *right* now.
If the implementation is hard to explain, it's a bad idea.
If the implementation is easy to explain, it may be a good idea.
Namespaces are one honking great idea -- let's do more of those!

    \end{Verbatim}

    \begin{Verbatim}[commandchars=\\\{\}]
{\color{incolor}In [{\color{incolor}5}]:} \PY{c+c1}{\PYZsh{} la deuxième fois il ne se passe plus rien}
        \PY{k+kn}{import} \PY{n+nn}{this}
\end{Verbatim}


    \hypertarget{les-raisons-de-ce-choix}{%
\subparagraph{Les raisons de ce choix}\label{les-raisons-de-ce-choix}}

    Le choix de ne charger le module qu'une seule fois est motivé par
plusieurs considérations.

    \begin{itemize}
\tightlist
\item
  D'une part, cela permet à deux modules de dépendre l'un de l'autre (ou
  plus généralement à avoir des cycles de dépendances), sans avoir à
  prendre de précaution particulière.
\end{itemize}

    \begin{itemize}
\tightlist
\item
  D'autre part, naturellement, cette stratégie améliore considérablement
  les performances.
\end{itemize}

    \begin{itemize}
\tightlist
\item
  Marginalement, \texttt{import} est une instruction comme une autre, et
  vous trouverez occasionnellement un avantage à l'utiliser à
  l'intérieur d'une fonction, \textbf{sans aucun surcoût} puisque vous
  ne payez le prix de l'import qu'au premier appel et non à chaque appel
  de la fonction.
\end{itemize}

\begin{verbatim}
    def ma_fonction():
        import un_module_improbable
        ....
\end{verbatim}

Cet usage n'est pas recommandé en général, mais de temps en temps peut
s'avérer très pratique pour alléger les dépendances entre modules dans
des contextes particuliers, comme du code multi-plateformes.

    \hypertarget{les-inconvuxe9nients-de-ce-choix---la-fonction-reload}{%
\subparagraph{\texorpdfstring{Les inconvénients de ce choix - la
fonction
\texttt{reload}}{Les inconvénients de ce choix - la fonction reload}}\label{les-inconvuxe9nients-de-ce-choix---la-fonction-reload}}

    L'inconvénient majeur de cette stratégie de chargement unique est
perceptible dans l'interpréteur interactif pendant le développement.
Nous avons vu comment IDLE traite le problème en remettant
l'interpréteur dans un état vierge lorsqu'on utilise la touche F5. Mais
dans l'interpréteur ``de base'', on n'a pas cette possibilité.

    Pour cette raison, python fournit dans le module \texttt{importlib} une
fonction \texttt{reload}, qui permet comme son nom l'indique de forcer
le rechargement d'un module, comme ceci~:

    \begin{Verbatim}[commandchars=\\\{\}]
{\color{incolor}In [{\color{incolor}6}]:} \PY{k+kn}{from} \PY{n+nn}{importlib} \PY{k}{import} \PY{n}{reload}
        \PY{n}{reload}\PY{p}{(}\PY{n}{multiple\PYZus{}import}\PY{p}{)}
\end{Verbatim}


    \begin{Verbatim}[commandchars=\\\{\}]
chargement de multiple\_import

    \end{Verbatim}

\begin{Verbatim}[commandchars=\\\{\}]
{\color{outcolor}Out[{\color{outcolor}6}]:} <module 'multiple\_import' from '/Users/tparment/git/flotpython/modules/multiple\_import.py'>
\end{Verbatim}
            
    Remarquez bien que \texttt{importlib.reload} est une fonction et non une
instruction comme \texttt{import} - d'où la syntaxe avec les parenthèses
qui n'est pas celle de \texttt{import}.

Notez également que la fonction \texttt{importlib.reload} a été
introduite en python3.4, avant, il fallait utiliser la fonction
\texttt{imp.reload} qui est dépréciée depuis python3.4 mais qui existe
toujours. Évidemment, vous devez maintenant exlusivement utiliser la
fonction \texttt{importlib.reload}.

    \begin{center}\rule{0.5\linewidth}{\linethickness}\end{center}

    \textbf{NOTE} spécifique à l'environnement des \textbf{notebooks} (en
fait, à l'utilisation de ipython)~:

À l'intérieur d'un notebook, vous
\href{https://ipython.org/ipython-doc/3/config/extensions/autoreload.html}{pouvez
faire comme ceci} pour recharger le code importé automatiquement~:

    \begin{Verbatim}[commandchars=\\\{\}]
{\color{incolor}In [{\color{incolor}7}]:} \PY{c+c1}{\PYZsh{} charger le magic \PYZsq{}autoreload\PYZsq{}}
        \PY{o}{\PYZpc{}}\PY{k}{load\PYZus{}ext} autoreload
\end{Verbatim}


    \begin{Verbatim}[commandchars=\\\{\}]
{\color{incolor}In [{\color{incolor}8}]:} \PY{c+c1}{\PYZsh{} activer autoreload}
        \PY{o}{\PYZpc{}}\PY{k}{autoreload} 2
\end{Verbatim}


    À partir de cet instant, et si le code d'un module importé est modifié
par ailleurs (ce qui est difficile à simuler dans notre environnement),
alors le module en question sera effectivement rechargé lors du prochain
import. Voyez le lien ci-dessus pour plus de détails.

    \hypertarget{compluxe9ment---niveau-avancuxe9}{%
\subsection{Complément - niveau
avancé}\label{compluxe9ment---niveau-avancuxe9}}

    Revenons à python standard. Pour ceux qui sont intéressés par les
détails, signalons enfin les deux variables suivantes.

    \hypertarget{sys.modules}{%
\subsubsection{\texorpdfstring{\texttt{sys.modules}}{sys.modules}}\label{sys.modules}}

    L'interpréteur utilise cette variable pour conserver la trace des
modules actuellement chargés.

    \begin{Verbatim}[commandchars=\\\{\}]
{\color{incolor}In [{\color{incolor}9}]:} \PY{k+kn}{import} \PY{n+nn}{sys}
        \PY{l+s+s1}{\PYZsq{}}\PY{l+s+s1}{csv}\PY{l+s+s1}{\PYZsq{}} \PY{o+ow}{in} \PY{n}{sys}\PY{o}{.}\PY{n}{modules}
\end{Verbatim}


\begin{Verbatim}[commandchars=\\\{\}]
{\color{outcolor}Out[{\color{outcolor}9}]:} False
\end{Verbatim}
            
    \begin{Verbatim}[commandchars=\\\{\}]
{\color{incolor}In [{\color{incolor}10}]:} \PY{k+kn}{import} \PY{n+nn}{csv}
         \PY{l+s+s1}{\PYZsq{}}\PY{l+s+s1}{csv}\PY{l+s+s1}{\PYZsq{}} \PY{o+ow}{in} \PY{n}{sys}\PY{o}{.}\PY{n}{modules}
\end{Verbatim}


\begin{Verbatim}[commandchars=\\\{\}]
{\color{outcolor}Out[{\color{outcolor}10}]:} True
\end{Verbatim}
            
    \begin{Verbatim}[commandchars=\\\{\}]
{\color{incolor}In [{\color{incolor}11}]:} \PY{n}{csv} \PY{o+ow}{is} \PY{n}{sys}\PY{o}{.}\PY{n}{modules}\PY{p}{[}\PY{l+s+s1}{\PYZsq{}}\PY{l+s+s1}{csv}\PY{l+s+s1}{\PYZsq{}}\PY{p}{]}
\end{Verbatim}


\begin{Verbatim}[commandchars=\\\{\}]
{\color{outcolor}Out[{\color{outcolor}11}]:} True
\end{Verbatim}
            
    La
\href{https://docs.python.org/3/library/sys.html\#sys.modules}{documentation
sur \texttt{sys.modules}} indique qu'il est possible de forcer le
rechargement d'un module en l'enlevant de cette variable
\texttt{sys.modules}.

    \begin{Verbatim}[commandchars=\\\{\}]
{\color{incolor}In [{\color{incolor}12}]:} \PY{k}{del} \PY{n}{sys}\PY{o}{.}\PY{n}{modules}\PY{p}{[}\PY{l+s+s1}{\PYZsq{}}\PY{l+s+s1}{multiple\PYZus{}import}\PY{l+s+s1}{\PYZsq{}}\PY{p}{]}
         \PY{k+kn}{import} \PY{n+nn}{multiple\PYZus{}import}
\end{Verbatim}


    \begin{Verbatim}[commandchars=\\\{\}]
chargement de multiple\_import

    \end{Verbatim}

    \hypertarget{sys.builtin_module_names}{%
\subsubsection{\texorpdfstring{\texttt{sys.builtin\_module\_names}}{sys.builtin\_module\_names}}\label{sys.builtin_module_names}}

    Signalons enfin
\href{https://docs.python.org/3/library/sys.html\#sys.builtin_module_names}{la
variable \texttt{sys.builtin\_module\_names}} qui contient le nom des
modules, comme par exemple le garbage collector \texttt{gc}, qui sont
implémentés en C et font partie intégrante de l'interpréteur.

    \begin{Verbatim}[commandchars=\\\{\}]
{\color{incolor}In [{\color{incolor}13}]:} \PY{l+s+s1}{\PYZsq{}}\PY{l+s+s1}{gc}\PY{l+s+s1}{\PYZsq{}} \PY{o+ow}{in} \PY{n}{sys}\PY{o}{.}\PY{n}{builtin\PYZus{}module\PYZus{}names}
\end{Verbatim}


\begin{Verbatim}[commandchars=\\\{\}]
{\color{outcolor}Out[{\color{outcolor}13}]:} True
\end{Verbatim}
            
    \hypertarget{pour-en-savoir-plus}{%
\subsubsection{Pour en savoir plus}\label{pour-en-savoir-plus}}

    Pour aller plus loin, vous pouvez lire
\href{https://docs.python.org/3/reference/simple_stmts.html\#the-import-statement}{la
documentation sur l'instruction \texttt{import}}


    % Add a bibliography block to the postdoc
    
    
    
