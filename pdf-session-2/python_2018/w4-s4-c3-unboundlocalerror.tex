    \hypertarget{lexception-unboundlocalerror}{%
\section{\texorpdfstring{L'exception
\texttt{UnboundLocalError}}{L'exception UnboundLocalError}}\label{lexception-unboundlocalerror}}

    \hypertarget{compluxe9ment---niveau-intermuxe9diaire}{%
\subsection{Complément - niveau
intermédiaire}\label{compluxe9ment---niveau-intermuxe9diaire}}

    Nous résumons ici quelques cas simples de portée de variables.

    \hypertarget{variable-locale}{%
\subsubsection{Variable locale}\label{variable-locale}}

    Les \textbf{arguments} attendus par la fonction sont considérés comme
des variables \textbf{locales}, c'est-à-dire dans l'espace de noms de la
fonction.\\

Pour définir une autre variable locale, il suffit de la définir
(l'affecter), elle devient alors accessible en lecture~:

    \begin{Verbatim}[commandchars=\\\{\}]
{\color{incolor}In [{\color{incolor}1}]:} \PY{k}{def} \PY{n+nf}{ma\PYZus{}fonction1}\PY{p}{(}\PY{p}{)}\PY{p}{:}
            \PY{n}{variable1} \PY{o}{=} \PY{l+s+s2}{\PYZdq{}}\PY{l+s+s2}{locale}\PY{l+s+s2}{\PYZdq{}}
            \PY{n+nb}{print}\PY{p}{(}\PY{n}{variable1}\PY{p}{)}
        
        \PY{n}{ma\PYZus{}fonction1}\PY{p}{(}\PY{p}{)}
\end{Verbatim}


    \begin{Verbatim}[commandchars=\\\{\}]
locale

    \end{Verbatim}

    et ceci que l'on ait ou non une variable globale de même nom

    \begin{Verbatim}[commandchars=\\\{\}]
{\color{incolor}In [{\color{incolor}2}]:} \PY{n}{variable2} \PY{o}{=} \PY{l+s+s2}{\PYZdq{}}\PY{l+s+s2}{globale}\PY{l+s+s2}{\PYZdq{}}
        
        \PY{k}{def} \PY{n+nf}{ma\PYZus{}fonction2}\PY{p}{(}\PY{p}{)}\PY{p}{:}
            \PY{n}{variable2} \PY{o}{=} \PY{l+s+s2}{\PYZdq{}}\PY{l+s+s2}{locale}\PY{l+s+s2}{\PYZdq{}}
            \PY{n+nb}{print}\PY{p}{(}\PY{n}{variable2}\PY{p}{)}
        
        \PY{n}{ma\PYZus{}fonction2}\PY{p}{(}\PY{p}{)}
\end{Verbatim}


    \begin{Verbatim}[commandchars=\\\{\}]
locale

    \end{Verbatim}

    \hypertarget{variable-globale}{%
\subsubsection{Variable globale}\label{variable-globale}}

    On peut accéder \textbf{en lecture} à une variable globale sans
précaution particulière~:

    \begin{Verbatim}[commandchars=\\\{\}]
{\color{incolor}In [{\color{incolor}3}]:} \PY{n}{variable3} \PY{o}{=} \PY{l+s+s2}{\PYZdq{}}\PY{l+s+s2}{globale}\PY{l+s+s2}{\PYZdq{}}
        
        \PY{k}{def} \PY{n+nf}{ma\PYZus{}fonction3}\PY{p}{(}\PY{p}{)}\PY{p}{:}
            \PY{n+nb}{print}\PY{p}{(}\PY{n}{variable3}\PY{p}{)}
        
        \PY{n}{ma\PYZus{}fonction3}\PY{p}{(}\PY{p}{)}
\end{Verbatim}


    \begin{Verbatim}[commandchars=\\\{\}]
globale

    \end{Verbatim}

    \hypertarget{mais-il-faut-choisir}{%
\subsubsection{Mais il faut choisir~!}\label{mais-il-faut-choisir}}

    Par contre on ne \textbf{peut pas} faire la chose suivante dans une
fonction. On ne peut pas utiliser \textbf{d'abord} une variable comme
une variable \textbf{globale}, \textbf{puis} essayer de l'affecter
localement - ce qui signifie la déclarer comme une \textbf{locale}~:

    \begin{Verbatim}[commandchars=\\\{\}]
{\color{incolor}In [{\color{incolor}4}]:} \PY{c+c1}{\PYZsh{} cet exemple ne fonctionne pas et lève UnboundLocalError}
        \PY{n}{variable4} \PY{o}{=} \PY{l+s+s2}{\PYZdq{}}\PY{l+s+s2}{globale}\PY{l+s+s2}{\PYZdq{}}
        
        \PY{k}{def} \PY{n+nf}{ma\PYZus{}fonction4}\PY{p}{(}\PY{p}{)}\PY{p}{:}
            \PY{c+c1}{\PYZsh{} on référence la variable globale}
            \PY{n+nb}{print}\PY{p}{(}\PY{n}{variable4}\PY{p}{)}
            \PY{c+c1}{\PYZsh{} et maintenant on crée une variable locale}
            \PY{n}{variable4} \PY{o}{=} \PY{l+s+s2}{\PYZdq{}}\PY{l+s+s2}{locale}\PY{l+s+s2}{\PYZdq{}}
        
        \PY{c+c1}{\PYZsh{} on \PYZdq{}attrape\PYZdq{} l\PYZsq{}exception}
        \PY{k}{try}\PY{p}{:}
            \PY{n}{ma\PYZus{}fonction4}\PY{p}{(}\PY{p}{)}
        \PY{k}{except} \PY{n+ne}{Exception} \PY{k}{as} \PY{n}{e}\PY{p}{:}
            \PY{n+nb}{print}\PY{p}{(}\PY{n}{f}\PY{l+s+s2}{\PYZdq{}}\PY{l+s+s2}{OOPS, exception }\PY{l+s+s2}{\PYZob{}}\PY{l+s+s2}{type(e)\PYZcb{}:}\PY{l+s+se}{\PYZbs{}n}\PY{l+s+si}{\PYZob{}e\PYZcb{}}\PY{l+s+s2}{\PYZdq{}}\PY{p}{)}
\end{Verbatim}


    \begin{Verbatim}[commandchars=\\\{\}]
OOPS, exception <class 'UnboundLocalError'>:
local variable 'variable4' referenced before assignment

    \end{Verbatim}

    \hypertarget{comment-faire-alors}{%
\subsubsection{Comment faire alors~?}\label{comment-faire-alors}}

    L'intérêt de cette erreur est d'interdire de mélanger des variables
locales et globales de même nom dans une même fonction. On voit bien que
ça serait vite incompréhensible. Donc une variable dans une fonction
peut être \textbf{ou bien} locale si elle est affectée dans la fonction
\textbf{ou bien} globale, mais \textbf{pas les deux à la fois}. Si vous
avez une erreur \texttt{UnboundLocalError}, c'est qu'à un moment donné
vous avez fait cette confusion.\\

    Vous vous demandez peut-être à ce stade, mais comment fait-on alors pour
modifier une variable globale depuis une fonction~? Pour cela il faut
utiliser l'instruction \texttt{global} comme ceci~:

    \begin{Verbatim}[commandchars=\\\{\}]
{\color{incolor}In [{\color{incolor}5}]:} \PY{c+c1}{\PYZsh{} Pour résoudre ce conflit il faut explicitement}
        \PY{c+c1}{\PYZsh{} déclarer la variable  comme globale}
        \PY{n}{variable5} \PY{o}{=} \PY{l+s+s2}{\PYZdq{}}\PY{l+s+s2}{globale}\PY{l+s+s2}{\PYZdq{}}
        
        \PY{k}{def} \PY{n+nf}{ma\PYZus{}fonction5}\PY{p}{(}\PY{p}{)}\PY{p}{:}
            \PY{k}{global} \PY{n}{variable5}
            \PY{c+c1}{\PYZsh{} on référence la variable globale}
            \PY{n+nb}{print}\PY{p}{(}\PY{l+s+s2}{\PYZdq{}}\PY{l+s+s2}{dans la fonction}\PY{l+s+s2}{\PYZdq{}}\PY{p}{,} \PY{n}{variable5}\PY{p}{)}
            \PY{c+c1}{\PYZsh{} cette fois on modifie la variable globale}
            \PY{n}{variable5} \PY{o}{=} \PY{l+s+s2}{\PYZdq{}}\PY{l+s+s2}{changée localement}\PY{l+s+s2}{\PYZdq{}}
        
        \PY{n}{ma\PYZus{}fonction5}\PY{p}{(}\PY{p}{)}
        \PY{n+nb}{print}\PY{p}{(}\PY{l+s+s2}{\PYZdq{}}\PY{l+s+s2}{après la fonction}\PY{l+s+s2}{\PYZdq{}}\PY{p}{,} \PY{n}{variable5}\PY{p}{)}
\end{Verbatim}


    \begin{Verbatim}[commandchars=\\\{\}]
dans la fonction globale
après la fonction changée localement

    \end{Verbatim}

    Nous reviendrons plus longuement sur l'instruction \texttt{global} dans
la prochaine vidéo.

    \hypertarget{bonnes-pratiques}{%
\subsubsection{Bonnes pratiques}\label{bonnes-pratiques}}

    Cela étant dit, l'utilisation de variables globales est généralement
considérée comme une mauvaise pratique.\\

Le fait d'utiliser une variable globale en \emph{lecture seule} peut
rester acceptable, lorsqu'il s'agit de matérialiser une constante qu'il
est facile de changer. Mais dans une application aboutie, ces constantes
elles-mêmes peuvent être modifiées par l'utilisateur via un système de
configuration, donc on préférera passer en argument un objet
\emph{config}.\\

Et dans les cas où votre code doit recourir à l'utilisation de
l'instruction \texttt{global}, c'est très probablement que quelque chose
peut être amélioré au niveau de la conception de votre code.

Il est recommandé, au contraire, de passer en argument à une fonction
tout le contexte dont elle a besoin pour travailler~; et à l'inverse
d'utiliser le résultat d'une fonction plutôt que de modifier une
variable globale.\\