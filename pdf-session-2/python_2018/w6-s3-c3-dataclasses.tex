    
    
    
    

    

    \hypertarget{dataclasses}{%
\section{\texorpdfstring{\texttt{dataclasses}}{dataclasses}}\label{dataclasses}}

\textbf{\emph{Nouveauté de la version 3.7}}

Python 3.7 apporte un nouveauté pour simplifier la définition de classes
dites ``de données''~; ce type de classes s'applique pour des objets qui
sont essentiellement un assemblage de quelques champs de données.

    Comme cette capacité n'est disponible qu'à partir de Python 3.7 et que
le cours est basé sur Python 3.6, nous n'aurons pas la possibilité de
manipuler directement ce nouveau concept. Voici toutefois quelques
exemples pour vous donner un aperçu de ce qu'on peut faire de cette
notion.

    \hypertarget{aperuxe7u}{%
\subsubsection{Aperçu}\label{aperuxe7u}}

    La raison d'être de \texttt{dataclass} est de fournir - encore un -
moyen de définir des classes d'enregistrements.

Voici par exemple comment on pourrait définir une classe
\texttt{Personne}:

    \begin{Shaded}
\begin{Highlighting}[frame=lines,framerule=0.6mm,rulecolor=\color{asisframecolor}]
\OperatorTok{>>>} \ImportTok{from}\NormalTok{ dataclasses }\ImportTok{import}\NormalTok{ dataclass}
\OperatorTok{>>>}
\OperatorTok{>>>} \OperatorTok{@}\NormalTok{dataclass}
\NormalTok{... }\KeywordTok{class}\NormalTok{ Personne:}
\NormalTok{...     nom: }\BuiltInTok{str}
\NormalTok{...     age: }\BuiltInTok{int}
\NormalTok{...     email: }\BuiltInTok{str} \OperatorTok{=} \StringTok{""}
\NormalTok{...}
\OperatorTok{>>>}\NormalTok{ personne }\OperatorTok{=}\NormalTok{ Personne(nom}\OperatorTok{=}\StringTok{'jean'}\NormalTok{, age}\OperatorTok{=}\DecValTok{12}\NormalTok{)}
\OperatorTok{>>>}
\OperatorTok{>>>} \BuiltInTok{print}\NormalTok{(personne)}
\NormalTok{Personne(nom}\OperatorTok{=}\StringTok{'jean'}\NormalTok{, age}\OperatorTok{=}\DecValTok{12}\NormalTok{, email}\OperatorTok{=}\StringTok{''}\NormalTok{)}
\OperatorTok{>>>}
\end{Highlighting}
\end{Shaded}

    \hypertarget{instances-non-mutables}{%
\subsubsection{Instances non mutables}\label{instances-non-mutables}}

    Le décorateur \texttt{dataclass} accepte divers arguments pour choisir
le comportement de certains aspects de la classe. Reportez-vous à la
documentation pour une liste complète, mais voici un exemple qui utilise
\texttt{frozen=True} et qui illustre la possibilité de créer des
instances non mutables. Nous retrouvons ici le même scénario d'ensemble
de points que nous avons déjà rencontré plusieurs fois~:

    \begin{Shaded}
\begin{Highlighting}[frame=lines,framerule=0.6mm,rulecolor=\color{asisframecolor}]
\OperatorTok{>>>} \ImportTok{import}\NormalTok{ sys}\OperatorTok{;} \BuiltInTok{print}\NormalTok{(sys.version)}
\DecValTok{3}\NormalTok{.}\FloatTok{7.0}\NormalTok{ (default, Jun }\DecValTok{29} \DecValTok{2018}\NormalTok{, }\DecValTok{20}\NormalTok{:}\DecValTok{14}\NormalTok{:}\DecValTok{27}\NormalTok{)}
\NormalTok{[Clang }\DecValTok{9}\NormalTok{.}\FloatTok{0.0}\NormalTok{ (clang}\DecValTok{-900}\NormalTok{.}\DecValTok{0}\NormalTok{.}\FloatTok{39.2}\NormalTok{)]}
\end{Highlighting}
\end{Shaded}

    \begin{Shaded}
\begin{Highlighting}[frame=lines,framerule=0.6mm,rulecolor=\color{asisframecolor}]
\OperatorTok{>>>} \ImportTok{from}\NormalTok{ dataclasses }\ImportTok{import}\NormalTok{ dataclass}
\OperatorTok{>>>}
\OperatorTok{>>>} \OperatorTok{@}\NormalTok{dataclass(frozen}\OperatorTok{=}\VariableTok{True}\NormalTok{)}
\NormalTok{... }\KeywordTok{class}\NormalTok{ Point:}
\NormalTok{...     x: }\BuiltInTok{float}
\NormalTok{...     y: }\BuiltInTok{float}
\NormalTok{...}
\NormalTok{...     }\KeywordTok{def} \FunctionTok{__eq__}\NormalTok{(}\VariableTok{self}\NormalTok{, other):}
\NormalTok{...         }\ControlFlowTok{return} \VariableTok{self}\NormalTok{.x }\OperatorTok{==}\NormalTok{ other.x }\KeywordTok{and} \VariableTok{self}\NormalTok{.y }\OperatorTok{==}\NormalTok{ other.y}
\NormalTok{...}
\NormalTok{...     }\KeywordTok{def} \FunctionTok{__hash__}\NormalTok{(}\VariableTok{self}\NormalTok{):}
\NormalTok{...         }\ControlFlowTok{return}\NormalTok{ (}\DecValTok{11} \OperatorTok{*} \VariableTok{self}\NormalTok{.x }\OperatorTok{+} \VariableTok{self}\NormalTok{.y) }\OperatorTok{//} \DecValTok{16}
\NormalTok{...}
\OperatorTok{>>>}\NormalTok{ p1, p2, p3 }\OperatorTok{=}\NormalTok{ Point(}\DecValTok{1}\NormalTok{, }\DecValTok{1}\NormalTok{), Point(}\DecValTok{1}\NormalTok{, }\DecValTok{1}\NormalTok{), Point(}\DecValTok{1}\NormalTok{, }\DecValTok{1}\NormalTok{)}
\OperatorTok{>>>}
\OperatorTok{>>>}\NormalTok{ s }\OperatorTok{=}\NormalTok{ \{p1, p2\}}
\OperatorTok{>>>} \BuiltInTok{len}\NormalTok{(s)}
\DecValTok{1}
\OperatorTok{>>>}
\OperatorTok{>>>}\NormalTok{ p3 }\KeywordTok{in}\NormalTok{ s}
\VariableTok{True}
\OperatorTok{>>>}
\OperatorTok{>>>} \ControlFlowTok{try}\NormalTok{:}
\NormalTok{...     p1.x }\OperatorTok{=} \DecValTok{10}
\NormalTok{... }\ControlFlowTok{except} \PreprocessorTok{Exception} \ImportTok{as}\NormalTok{ e:}
\NormalTok{...     }\BuiltInTok{print}\NormalTok{(}\SpecialStringTok{f"OOPS }\SpecialCharTok{\{}\BuiltInTok{type}\NormalTok{(e)}\SpecialCharTok{\}}\SpecialStringTok{"}\NormalTok{)}
\NormalTok{...}
\NormalTok{OOPS }\OperatorTok{<}\KeywordTok{class} \StringTok{'dataclasses.FrozenInstanceError'}\OperatorTok{>}
\end{Highlighting}
\end{Shaded}

    \hypertarget{pour-aller-plus-loin}{%
\subsubsection{Pour aller plus loin}\label{pour-aller-plus-loin}}

    Vous pouvez vous rapporter

\begin{itemize}
\tightlist
\item
  \href{https://www.python.org/dev/peps/pep-0557/}{au PEP 557} qui a
  abouti au consensus, et
\item
  \href{https://docs.python.org/3/library/dataclasses.html}{à la
  documentation officielle du module}.
\end{itemize}


    % Add a bibliography block to the postdoc
    
    
    
