    \hypertarget{fusionner-des-donnuxe9es}{%
\section{Fusionner des données}\label{fusionner-des-donnuxe9es}}

    \hypertarget{exercices}{%
\subsection{Exercices}\label{exercices}}

    Cet exercice vient en deux versions, une de niveau basique et une de
niveau intermédiaire.\\

La version basique est une application de la technique d'indexation que
l'on a vue dans le complément ``Gérer des enregistrements''. On peut
très bien faire les deux versions dans l'ordre, une fois qu'on a fait la
version basique on est en principe un peu plus avancé pour aborder la
version intermédiaire.

    \hypertarget{contexte}{%
\subsubsection{Contexte}\label{contexte}}

    Nous allons commencer à utiliser des données un peu plus réalistes. Il
s'agit de données obtenues auprès de
\href{https://www.marinetraffic.com}{MarineTraffic} - et légèrement
simplifiées pour les besoins de l'exercice. Ce site expose les
coordonnées géographiques de bateaux observées en mer au travers d'un
réseau de collecte de type \emph{crowdsourcing}.\\

    De manière à optimiser le volume de données à transférer, l'API de
MarineTraffic offre deux modes pour obtenir les données~:

\begin{itemize}
	\item 
	\textbf{mode étendu}~: chaque mesure (bateau x position x temps) est accompagnée de
	tous les détails du bateau (\texttt{id}, nom, pays de rattachement, etc.)~;
	\item
	\textbf{mode abrégé}~: chaque mesure est uniquement attachée à
	l'\texttt{id} du bateau.	
\end{itemize}

En effet, chaque bateau possède un identifiant unique qui est un entier,
que l'on note \texttt{id}.

    \hypertarget{chargement-des-donnuxe9es}{%
\subsubsection{Chargement des données}\label{chargement-des-donnuxe9es}}

    Commençons par charger les données de l'exercice~:

    \begin{Verbatim}[commandchars=\\\{\}]
{\color{incolor}In [{\color{incolor} }]:} \PY{k+kn}{from} \PY{n+nn}{corrections}\PY{n+nn}{.}\PY{n+nn}{exo\PYZus{}marine\PYZus{}dict} \PY{k}{import} \PY{n}{extended}\PY{p}{,} \PY{n}{abbreviated}
\end{Verbatim}


    \hypertarget{format-des-donnuxe9es}{%
\subsubsection{Format des données}\label{format-des-donnuxe9es}}

    Le format de ces données est relativement simple, il s'agit dans les
deux cas d'une liste d'entrées - une par bateau.\\

Chaque entrée à son tour est une liste qui contient~:

\begin{verbatim}
mode étendu: [id, latitude, longitude, date_heure, nom_bateau, code_pays, ...]
mode abrégé: [id, latitude, longitude, date_heure]
\end{verbatim}

sachant que les entrées après le code pays dans le format étendu ne nous
intéressent pas pour cet exercice.

    \begin{Verbatim}[commandchars=\\\{\}]
{\color{incolor}In [{\color{incolor} }]:} \PY{c+c1}{\PYZsh{} une entrée étendue est une liste qui ressemble à ceci}
        \PY{n}{sample\PYZus{}extended\PYZus{}entry} \PY{o}{=} \PY{n}{extended}\PY{p}{[}\PY{l+m+mi}{3}\PY{p}{]}
        \PY{n+nb}{print}\PY{p}{(}\PY{n}{sample\PYZus{}extended\PYZus{}entry}\PY{p}{)}
\end{Verbatim}


    \begin{Verbatim}[commandchars=\\\{\}]
{\color{incolor}In [{\color{incolor} }]:} \PY{c+c1}{\PYZsh{} une entrée abrégée ressemble à ceci}
        \PY{n}{sample\PYZus{}abbreviated\PYZus{}entry} \PY{o}{=} \PY{n}{abbreviated}\PY{p}{[}\PY{l+m+mi}{0}\PY{p}{]}
        \PY{n+nb}{print}\PY{p}{(}\PY{n}{sample\PYZus{}abbreviated\PYZus{}entry}\PY{p}{)}
\end{Verbatim}


    On précise également que les deux listes \texttt{extended} et
\texttt{abbreviated}~:

\begin{itemize}
	\item 
	possèdent exactement \textbf{le même nombre} d'entrées~;
	\item
	et correspondent \textbf{aux mêmes bateaux}~;
	\item
	mais naturellement \textbf{à des moments différents}~;
	\item
	et \textbf{pas forcément dans le même ordre}.
\end{itemize}

    \hypertarget{exercice---niveau-basique}{%
\subsubsection{Exercice - niveau
basique}\label{exercice---niveau-basique}}

    \begin{Verbatim}[commandchars=\\\{\}]
{\color{incolor}In [{\color{incolor} }]:} \PY{c+c1}{\PYZsh{} chargement de l\PYZsq{}exercice}
        \PY{k+kn}{from} \PY{n+nn}{corrections}\PY{n+nn}{.}\PY{n+nn}{exo\PYZus{}marine\PYZus{}dict} \PY{k}{import} \PY{n}{exo\PYZus{}index}
\end{Verbatim}


    \hypertarget{but-de-lexercice}{%
\subparagraph{But de l'exercice}\label{but-de-lexercice}}

    On vous demande d'écrire une fonction \texttt{index} qui calcule, à
partir de la liste des données étendues, un dictionnaire qui est~:

\begin{itemize}
	\item 
	indexé par l'\texttt{id} de chaque bateau~;
	\item
	et qui a pour valeur la liste qui décrit le bateau correspondant.
\end{itemize}

    De manière plus imagée, si~:

\begin{Shaded}
\begin{Highlighting}[]
\NormalTok{extended }\OperatorTok{=}\NormalTok{ [ bateau1, bateau2, ... ]}
\end{Highlighting}
\end{Shaded}

Et si~:

\begin{Shaded}
\begin{Highlighting}[]
\NormalTok{bateau1 }\OperatorTok{=}\NormalTok{ [ id1, latitude, ... ]}
\end{Highlighting}
\end{Shaded}

On doit obtenir comme résultat de \texttt{index} un dictionnaire~:

\begin{Shaded}
\begin{Highlighting}[]
\NormalTok{\{}
\NormalTok{    id1 }\OperatorTok{->}\NormalTok{ [ id_bateau1, latitude, ... ],}
\NormalTok{    id2 ...}
\NormalTok{\}}
\end{Highlighting}
\end{Shaded}

Bref, on veut pouvoir retrouver les différents éléments de la liste
\texttt{extended} par accès direct, en ne faisant qu'une seule recherche
dans l'index.

    \begin{Verbatim}[commandchars=\\\{\}]
{\color{incolor}In [{\color{incolor} }]:} \PY{c+c1}{\PYZsh{} le résultat attendu}
        \PY{n}{result\PYZus{}index} \PY{o}{=} \PY{n}{exo\PYZus{}index}\PY{o}{.}\PY{n}{resultat}\PY{p}{(}\PY{n}{extended}\PY{p}{)}
        
        \PY{c+c1}{\PYZsh{} on en profite pour illustrer le module pprint}
        \PY{k+kn}{from} \PY{n+nn}{pprint} \PY{k}{import} \PY{n}{pprint}
        
        \PY{c+c1}{\PYZsh{} à quoi ressemble le résultat pour un bateau au hasard}
        \PY{k}{for} \PY{n}{key}\PY{p}{,} \PY{n}{value} \PY{o+ow}{in} \PY{n}{result\PYZus{}index}\PY{o}{.}\PY{n}{items}\PY{p}{(}\PY{p}{)}\PY{p}{:}
            \PY{n+nb}{print}\PY{p}{(}\PY{l+s+s2}{\PYZdq{}}\PY{l+s+s2}{==== clé}\PY{l+s+s2}{\PYZdq{}}\PY{p}{)}
            \PY{n}{pprint}\PY{p}{(}\PY{n}{key}\PY{p}{)}
            \PY{n+nb}{print}\PY{p}{(}\PY{l+s+s2}{\PYZdq{}}\PY{l+s+s2}{==== valeur}\PY{l+s+s2}{\PYZdq{}}\PY{p}{)}
            \PY{n}{pprint}\PY{p}{(}\PY{n}{value}\PY{p}{)}
            \PY{k}{break}
\end{Verbatim}


    Remarquez ci-dessus l'utilisation d'un utilitaire parfois pratique~: le
\href{https://docs.python.org/3/library/pprint.html}{module
\texttt{pprint} pour pretty-printer}.

    \hypertarget{votre-code}{%
\subparagraph{Votre code}\label{votre-code}}

    \begin{Verbatim}[commandchars=\\\{\}]
{\color{incolor}In [{\color{incolor} }]:} \PY{k}{def} \PY{n+nf}{index}\PY{p}{(}\PY{n}{extended}\PY{p}{)}\PY{p}{:}
            \PY{l+s+s2}{\PYZdq{}}\PY{l+s+s2}{\PYZlt{}votre\PYZus{}code\PYZgt{}}\PY{l+s+s2}{\PYZdq{}}
\end{Verbatim}


    \hypertarget{validation}{%
\subparagraph{Validation}\label{validation}}

    \begin{Verbatim}[commandchars=\\\{\}]
{\color{incolor}In [{\color{incolor} }]:} \PY{n}{exo\PYZus{}index}\PY{o}{.}\PY{n}{correction}\PY{p}{(}\PY{n}{index}\PY{p}{,} \PY{n}{abbreviated}\PY{p}{)}
\end{Verbatim}


    Vous remarquerez d'ailleurs que la seule chose que l'on utilise dans cet
exercice, c'est que l'id des bateaux arrive en première position (dans
la liste qui matérialise le bateau), aussi votre code doit marcher à
l'identique avec les bateaux étendus~:

    \begin{Verbatim}[commandchars=\\\{\}]
{\color{incolor}In [{\color{incolor} }]:} \PY{n}{exo\PYZus{}index}\PY{o}{.}\PY{n}{correction}\PY{p}{(}\PY{n}{index}\PY{p}{,} \PY{n}{extended}\PY{p}{)}
\end{Verbatim}


    \hypertarget{exercice---niveau-intermuxe9diaire}{%
\subsubsection{Exercice - niveau
intermédiaire}\label{exercice---niveau-intermuxe9diaire}}

    \begin{Verbatim}[commandchars=\\\{\}]
{\color{incolor}In [{\color{incolor} }]:} \PY{c+c1}{\PYZsh{} chargement de l\PYZsq{}exercice}
        \PY{k+kn}{from} \PY{n+nn}{corrections}\PY{n+nn}{.}\PY{n+nn}{exo\PYZus{}marine\PYZus{}dict} \PY{k}{import} \PY{n}{exo\PYZus{}merge}
\end{Verbatim}


    \hypertarget{but-de-lexercice}{%
\subparagraph{But de l'exercice}\label{but-de-lexercice}}

    On vous demande d'écrire une fonction \texttt{merge} qui fasse une
consolidation des données, de façon à obtenir en sortie un
dictionnaire~:

\begin{Shaded}
\begin{Highlighting}[]
\BuiltInTok{id} \OperatorTok{->}\NormalTok{ [nom_bateau, code_pays, position_etendu, position_abrege]}
\end{Highlighting}
\end{Shaded}

dans lequel les deux objets \texttt{position} sont tous les deux des
tuples de la forme~:

\begin{Shaded}
\begin{Highlighting}[]
\NormalTok{(latitude, longitude, date_heure)}
\end{Highlighting}
\end{Shaded}

    Voici par exemple un couple clé-valeur dans le résultat attendu~:

    \begin{Verbatim}[commandchars=\\\{\}]
{\color{incolor}In [{\color{incolor} }]:} \PY{c+c1}{\PYZsh{} le résultat attendu}
        \PY{n}{result\PYZus{}merge} \PY{o}{=} \PY{n}{exo\PYZus{}merge}\PY{o}{.}\PY{n}{resultat}\PY{p}{(}\PY{n}{extended}\PY{p}{,} \PY{n}{abbreviated}\PY{p}{)}
        
        \PY{c+c1}{\PYZsh{} à quoi ressemble le résultat pour un bateau au hasard}
        \PY{k+kn}{from} \PY{n+nn}{pprint} \PY{k}{import} \PY{n}{pprint}
        \PY{k}{for} \PY{n}{key\PYZus{}value} \PY{o+ow}{in} \PY{n}{result\PYZus{}merge}\PY{o}{.}\PY{n}{items}\PY{p}{(}\PY{p}{)}\PY{p}{:}
            \PY{n}{pprint}\PY{p}{(}\PY{n}{key\PYZus{}value}\PY{p}{)}
            \PY{k}{break}
\end{Verbatim}


    \hypertarget{votre-code}{%
\subparagraph{Votre code}\label{votre-code}}

    \begin{Verbatim}[commandchars=\\\{\}]
{\color{incolor}In [{\color{incolor} }]:} \PY{k}{def} \PY{n+nf}{merge}\PY{p}{(}\PY{n}{extended}\PY{p}{,} \PY{n}{abbreviated}\PY{p}{)}\PY{p}{:}
            \PY{l+s+s2}{\PYZdq{}}\PY{l+s+s2}{votre code}\PY{l+s+s2}{\PYZdq{}}
\end{Verbatim}


    \hypertarget{validation}{%
\subparagraph{Validation}\label{validation}}

    \begin{Verbatim}[commandchars=\\\{\}]
{\color{incolor}In [{\color{incolor} }]:} \PY{n}{exo\PYZus{}merge}\PY{o}{.}\PY{n}{correction}\PY{p}{(}\PY{n}{merge}\PY{p}{,} \PY{n}{extended}\PY{p}{,} \PY{n}{abbreviated}\PY{p}{)}
\end{Verbatim}

\newpage
    \hypertarget{les-fichiers-de-donnuxe9es-complets}{%
\subsubsection{Les fichiers de données
complets}\label{les-fichiers-de-donnuxe9es-complets}}

    Signalons enfin pour ceux qui sont intéressés que les données chargées
dans cet exercice sont disponibles au format JSON - qui est précisément
celui exposé par marinetraffic.\\

Nous avons beaucoup simplifié les données d'entrée pour vous permettre
une mise au point plus facile. Si vous voulez vous amuser à charger des
données un peu plus significatives, sachez que~:

\begin{itemize}
\tightlist
\item
  vous avez accès aux fichiers de données plus complets~:

  \begin{itemize}
  \tightlist
  \item
    \url{data/marine-e1-ext.json}
  \item
    \url{data/marine-e1-abb.json}
  \end{itemize}
\item
  pour charger ces fichiers, qui sont donc au
  \href{http://en.wikipedia.org/wiki/JSON}{format JSON}, la connaissance
  intime de ce format n'est pas nécessaire, on peut tout simplement
  utiliser le \href{https://docs.python.org/3/library/json.html}{module
  \texttt{json}}. Voici le code utilisé dans l'exercice pour charger ces
  JSON en mémoire~; il utilise des notions que nous verrons dans les
  semaines à venir~:
\end{itemize}

    \begin{Verbatim}[commandchars=\\\{\}]
{\color{incolor}In [{\color{incolor} }]:} \PY{c+c1}{\PYZsh{} load data from files}
        \PY{k+kn}{import} \PY{n+nn}{json}
        
        \PY{k}{with} \PY{n+nb}{open}\PY{p}{(}\PY{l+s+s2}{\PYZdq{}}\PY{l+s+s2}{data/marine\PYZhy{}e1\PYZhy{}ext.json}\PY{l+s+s2}{\PYZdq{}}\PY{p}{,} \PY{n}{encoding}\PY{o}{=}\PY{l+s+s2}{\PYZdq{}}\PY{l+s+s2}{utf\PYZhy{}8}\PY{l+s+s2}{\PYZdq{}}\PY{p}{)} \PY{k}{as} \PY{n}{feed}\PY{p}{:}
            \PY{n}{extended\PYZus{}full} \PY{o}{=} \PY{n}{json}\PY{o}{.}\PY{n}{load}\PY{p}{(}\PY{n}{feed}\PY{p}{)}
        
        \PY{k}{with} \PY{n+nb}{open}\PY{p}{(}\PY{l+s+s2}{\PYZdq{}}\PY{l+s+s2}{data/marine\PYZhy{}e1\PYZhy{}abb.json}\PY{l+s+s2}{\PYZdq{}}\PY{p}{,} \PY{n}{encoding}\PY{o}{=}\PY{l+s+s2}{\PYZdq{}}\PY{l+s+s2}{utf\PYZhy{}8}\PY{l+s+s2}{\PYZdq{}}\PY{p}{)} \PY{k}{as} \PY{n}{feed}\PY{p}{:}
            \PY{n}{abbreviated\PYZus{}full} \PY{o}{=} \PY{n}{json}\PY{o}{.}\PY{n}{load}\PY{p}{(}\PY{n}{feed}\PY{p}{)}
\end{Verbatim}


    Une fois que vous avez un code qui fonctionne vous pouvez le lancer sur
ces données plus copieuses en faisant~:

    \begin{Verbatim}[commandchars=\\\{\}]
{\color{incolor}In [{\color{incolor} }]:} \PY{n}{exo\PYZus{}merge}\PY{o}{.}\PY{n}{correction}\PY{p}{(}\PY{n}{merge}\PY{p}{,} \PY{n}{extended\PYZus{}full}\PY{p}{,} \PY{n}{abbreviated\PYZus{}full}\PY{p}{)}
\end{Verbatim}